%% Generated by Sphinx.
\def\sphinxdocclass{report}
\documentclass[letterpaper,10pt,english]{sphinxmanual}
\ifdefined\pdfpxdimen
   \let\sphinxpxdimen\pdfpxdimen\else\newdimen\sphinxpxdimen
\fi \sphinxpxdimen=.75bp\relax
\ifdefined\pdfimageresolution
    \pdfimageresolution= \numexpr \dimexpr1in\relax/\sphinxpxdimen\relax
\fi
%% let collapsible pdf bookmarks panel have high depth per default
\PassOptionsToPackage{bookmarksdepth=5}{hyperref}


\PassOptionsToPackage{warn}{textcomp}
\usepackage[utf8]{inputenc}
\ifdefined\DeclareUnicodeCharacter
% support both utf8 and utf8x syntaxes
  \ifdefined\DeclareUnicodeCharacterAsOptional
    \def\sphinxDUC#1{\DeclareUnicodeCharacter{"#1}}
  \else
    \let\sphinxDUC\DeclareUnicodeCharacter
  \fi
  \sphinxDUC{00A0}{\nobreakspace}
  \sphinxDUC{2500}{\sphinxunichar{2500}}
  \sphinxDUC{2502}{\sphinxunichar{2502}}
  \sphinxDUC{2514}{\sphinxunichar{2514}}
  \sphinxDUC{251C}{\sphinxunichar{251C}}
  \sphinxDUC{2572}{\textbackslash}
\fi
\usepackage{cmap}
\usepackage[T1]{fontenc}
\usepackage{amsmath,amssymb,amstext}
\usepackage{babel}



\usepackage{tgtermes}
\usepackage{tgheros}
\renewcommand{\ttdefault}{txtt}



\usepackage[Bjarne]{fncychap}
\usepackage{sphinx}

\fvset{fontsize=auto}
\usepackage{geometry}


% Include hyperref last.
\usepackage{hyperref}
% Fix anchor placement for figures with captions.
\usepackage{hypcap}% it must be loaded after hyperref.
% Set up styles of URL: it should be placed after hyperref.
\urlstyle{same}


\usepackage{sphinxmessages}
\setcounter{tocdepth}{1}



\title{dsptoolbox}
\date{Dec 29, 2022}
\release{0.0.2}
\author{Nicolas Franco\sphinxhyphen{}Gomez}
\newcommand{\sphinxlogo}{\vbox{}}
\renewcommand{\releasename}{Release}
\makeindex
\begin{document}

\ifdefined\shorthandoff
  \ifnum\catcode`\=\string=\active\shorthandoff{=}\fi
  \ifnum\catcode`\"=\active\shorthandoff{"}\fi
\fi

\pagestyle{empty}
\sphinxmaketitle
\pagestyle{plain}
\sphinxtableofcontents
\pagestyle{normal}
\phantomsection\label{\detokenize{index::doc}}


\sphinxstepscope


\chapter{Readme}
\label{\detokenize{readme:readme}}\label{\detokenize{readme::doc}}
\sphinxAtStartPar
This is a toolbox in form of a python package that handles algorithms to be used in dsp (digital signal processing) projects.

\sphinxAtStartPar
It is a personal project that is under active development and it will take some time until it reaches a certain level of maturity.
If you find some implementations interesting or useful, please feel free to use it for your projects and expand or change
functionalities.


\section{Getting Started}
\label{\detokenize{readme:getting-started}}
\sphinxAtStartPar
Check out the \sphinxhref{https://github.com/nico-franco-gomez/dsptoolbox/tree/main/examples}{examples} for some basic examples of the dsptoolbox package
and refer to the documentation for the complete description of classes and methods.

\begin{sphinxadmonition}{note}{Note:}
\sphinxAtStartPar
The documentation at \sphinxhref{http://dsptoolbox.readthedocs.io/}{read the docs} is available but not working completely without errors
because of a bug in one of the modules needed to build the package. This will be fixed in the future.
Until then, please refer to the other available documentation.
\end{sphinxadmonition}


\section{Installation}
\label{\detokenize{readme:installation}}
\sphinxAtStartPar
Use pip to install dsptoolbox

\begin{sphinxVerbatim}[commandchars=\\\{\}]
\PYG{g+gp}{\PYGZdl{} }pip install dsptoolbox
\end{sphinxVerbatim}

\sphinxAtStartPar
(Requires Python 3.10 or higher)

\sphinxstepscope


\chapter{Classes (dsptoolbox.classes)}
\label{\detokenize{classes:classes-dsptoolbox-classes}}\label{\detokenize{classes::doc}}
\sphinxAtStartPar
The main classes are listed here. For some filterbanks, special classes are used but the api works almost equally
as for the main FilterBank class.


\section{Signal}
\label{\detokenize{classes:module-dsptoolbox.classes.signal_class}}\label{\detokenize{classes:signal}}\index{module@\spxentry{module}!dsptoolbox.classes.signal\_class@\spxentry{dsptoolbox.classes.signal\_class}}\index{dsptoolbox.classes.signal\_class@\spxentry{dsptoolbox.classes.signal\_class}!module@\spxentry{module}}
\sphinxAtStartPar
Signal class
\index{Signal (class in dsptoolbox.classes.signal\_class)@\spxentry{Signal}\spxextra{class in dsptoolbox.classes.signal\_class}}

\begin{fulllineitems}
\phantomsection\label{\detokenize{classes:dsptoolbox.classes.signal_class.Signal}}
\pysigstartsignatures
\pysiglinewithargsret{\sphinxbfcode{\sphinxupquote{class\DUrole{w}{  }}}\sphinxcode{\sphinxupquote{dsptoolbox.classes.signal\_class.}}\sphinxbfcode{\sphinxupquote{Signal}}}{\emph{\DUrole{n}{path}\DUrole{o}{=}\DUrole{default_value}{None}}, \emph{\DUrole{n}{time\_data}\DUrole{o}{=}\DUrole{default_value}{None}}, \emph{\DUrole{n}{sampling\_rate\_hz}\DUrole{p}{:}\DUrole{w}{  }\DUrole{n}{\sphinxhref{https://docs.python.org/3/library/functions.html\#int}{int}}\DUrole{w}{  }\DUrole{o}{=}\DUrole{w}{  }\DUrole{default_value}{48000}}, \emph{\DUrole{n}{signal\_type}\DUrole{p}{:}\DUrole{w}{  }\DUrole{n}{\sphinxhref{https://docs.python.org/3/library/stdtypes.html\#str}{str}}\DUrole{w}{  }\DUrole{o}{=}\DUrole{w}{  }\DUrole{default_value}{\textquotesingle{}general\textquotesingle{}}}, \emph{\DUrole{n}{signal\_id}\DUrole{p}{:}\DUrole{w}{  }\DUrole{n}{\sphinxhref{https://docs.python.org/3/library/stdtypes.html\#str}{str}}\DUrole{w}{  }\DUrole{o}{=}\DUrole{w}{  }\DUrole{default_value}{\textquotesingle{}\textquotesingle{}}}}{}
\pysigstopsignatures
\sphinxAtStartPar
Bases: \sphinxhref{https://docs.python.org/3/library/functions.html\#object}{\sphinxcode{\sphinxupquote{object}}}

\sphinxAtStartPar
Class for general signals (time series). Most of the methods and
supported computations are focused on audio signals, but some features
might be generalizable to all kind of time series.
\begin{quote}\begin{description}
\sphinxlineitem{Attributes}\begin{description}
\sphinxlineitem{\sphinxstylestrong{number\_of\_channels}}
\sphinxlineitem{\sphinxstylestrong{sampling\_rate\_hz}}
\sphinxlineitem{\sphinxstylestrong{signal\_id}}
\sphinxlineitem{\sphinxstylestrong{signal\_type}}
\sphinxlineitem{\sphinxstylestrong{time\_data}}
\end{description}

\end{description}\end{quote}
\subsubsection*{Methods}


\begin{savenotes}\sphinxattablestart
\sphinxthistablewithglobalstyle
\sphinxthistablewithnovlinesstyle
\centering
\begin{tabulary}{\linewidth}[t]{\X{1}{2}\X{1}{2}}
\sphinxtoprule
\sphinxtableatstartofbodyhook
\sphinxAtStartPar
{\hyperref[\detokenize{classes:dsptoolbox.classes.signal_class.Signal.add_channel}]{\sphinxcrossref{\sphinxcode{\sphinxupquote{add\_channel}}}}}({[}path, new\_time\_data, ...{]})
&
\sphinxAtStartPar
Adds new channels to this signal object.
\\
\sphinxhline
\sphinxAtStartPar
{\hyperref[\detokenize{classes:dsptoolbox.classes.signal_class.Signal.copy}]{\sphinxcrossref{\sphinxcode{\sphinxupquote{copy}}}}}()
&
\sphinxAtStartPar
Returns a copy of the object.
\\
\sphinxhline
\sphinxAtStartPar
{\hyperref[\detokenize{classes:dsptoolbox.classes.signal_class.Signal.get_coherence}]{\sphinxcrossref{\sphinxcode{\sphinxupquote{get\_coherence}}}}}()
&
\sphinxAtStartPar
Returns the coherence matrix.
\\
\sphinxhline
\sphinxAtStartPar
{\hyperref[\detokenize{classes:dsptoolbox.classes.signal_class.Signal.get_csm}]{\sphinxcrossref{\sphinxcode{\sphinxupquote{get\_csm}}}}}({[}force\_computation{]})
&
\sphinxAtStartPar
Get Cross spectral matrix for all channels with the shape (frequencies, channels, channels)
\\
\sphinxhline
\sphinxAtStartPar
{\hyperref[\detokenize{classes:dsptoolbox.classes.signal_class.Signal.get_spectrogram}]{\sphinxcrossref{\sphinxcode{\sphinxupquote{get\_spectrogram}}}}}({[}channel\_number, ...{]})
&
\sphinxAtStartPar
Returns a matrix containing the STFT of a specific channel.
\\
\sphinxhline
\sphinxAtStartPar
{\hyperref[\detokenize{classes:dsptoolbox.classes.signal_class.Signal.get_spectrum}]{\sphinxcrossref{\sphinxcode{\sphinxupquote{get\_spectrum}}}}}({[}force\_computation{]})
&
\sphinxAtStartPar
Returns spectrum.
\\
\sphinxhline
\sphinxAtStartPar
{\hyperref[\detokenize{classes:dsptoolbox.classes.signal_class.Signal.get_time_vector}]{\sphinxcrossref{\sphinxcode{\sphinxupquote{get\_time\_vector}}}}}()
&
\sphinxAtStartPar
Returns the time vector associated with the signal.
\\
\sphinxhline
\sphinxAtStartPar
{\hyperref[\detokenize{classes:dsptoolbox.classes.signal_class.Signal.plot_coherence}]{\sphinxcrossref{\sphinxcode{\sphinxupquote{plot\_coherence}}}}}({[}returns{]})
&
\sphinxAtStartPar
Plots coherence measurements if there are any.
\\
\sphinxhline
\sphinxAtStartPar
{\hyperref[\detokenize{classes:dsptoolbox.classes.signal_class.Signal.plot_csm}]{\sphinxcrossref{\sphinxcode{\sphinxupquote{plot\_csm}}}}}({[}range\_hz, logx, with\_phase, returns{]})
&
\sphinxAtStartPar
Plots the cross spectral matrix of the multichannel signal.
\\
\sphinxhline
\sphinxAtStartPar
{\hyperref[\detokenize{classes:dsptoolbox.classes.signal_class.Signal.plot_group_delay}]{\sphinxcrossref{\sphinxcode{\sphinxupquote{plot\_group\_delay}}}}}({[}range\_hz, returns{]})
&
\sphinxAtStartPar
Plots group delay of each channel.
\\
\sphinxhline
\sphinxAtStartPar
{\hyperref[\detokenize{classes:dsptoolbox.classes.signal_class.Signal.plot_magnitude}]{\sphinxcrossref{\sphinxcode{\sphinxupquote{plot\_magnitude}}}}}({[}range\_hz, normalize, ...{]})
&
\sphinxAtStartPar
Plots magnitude spectrum.
\\
\sphinxhline
\sphinxAtStartPar
{\hyperref[\detokenize{classes:dsptoolbox.classes.signal_class.Signal.plot_phase}]{\sphinxcrossref{\sphinxcode{\sphinxupquote{plot\_phase}}}}}({[}range\_hz, unwrap, returns{]})
&
\sphinxAtStartPar
Plots phase of the frequency response, only available if the method for the spectrum parameters is not welch.
\\
\sphinxhline
\sphinxAtStartPar
{\hyperref[\detokenize{classes:dsptoolbox.classes.signal_class.Signal.plot_spectrogram}]{\sphinxcrossref{\sphinxcode{\sphinxupquote{plot\_spectrogram}}}}}({[}channel\_number, logfreqs, ...{]})
&
\sphinxAtStartPar
Plots STFT matrix of the given channel.
\\
\sphinxhline
\sphinxAtStartPar
{\hyperref[\detokenize{classes:dsptoolbox.classes.signal_class.Signal.plot_time}]{\sphinxcrossref{\sphinxcode{\sphinxupquote{plot\_time}}}}}({[}returns{]})
&
\sphinxAtStartPar
Plots time signals.
\\
\sphinxhline
\sphinxAtStartPar
{\hyperref[\detokenize{classes:dsptoolbox.classes.signal_class.Signal.remove_channel}]{\sphinxcrossref{\sphinxcode{\sphinxupquote{remove\_channel}}}}}({[}channel\_number{]})
&
\sphinxAtStartPar
Removes a channel.
\\
\sphinxhline
\sphinxAtStartPar
{\hyperref[\detokenize{classes:dsptoolbox.classes.signal_class.Signal.save_signal}]{\sphinxcrossref{\sphinxcode{\sphinxupquote{save\_signal}}}}}({[}path, mode{]})
&
\sphinxAtStartPar
Saves the Signal object as wav, flac or pickle.
\\
\sphinxhline
\sphinxAtStartPar
{\hyperref[\detokenize{classes:dsptoolbox.classes.signal_class.Signal.set_coherence}]{\sphinxcrossref{\sphinxcode{\sphinxupquote{set\_coherence}}}}}(coherence)
&
\sphinxAtStartPar
Sets the coherence measurements of the transfer function.
\\
\sphinxhline
\sphinxAtStartPar
{\hyperref[\detokenize{classes:dsptoolbox.classes.signal_class.Signal.set_csm_parameters}]{\sphinxcrossref{\sphinxcode{\sphinxupquote{set\_csm\_parameters}}}}}({[}window\_length\_samples, ...{]})
&
\sphinxAtStartPar
Sets all necessary parameters for the computation of the CSM.
\\
\sphinxhline
\sphinxAtStartPar
{\hyperref[\detokenize{classes:dsptoolbox.classes.signal_class.Signal.set_spectrogram_parameters}]{\sphinxcrossref{\sphinxcode{\sphinxupquote{set\_spectrogram\_parameters}}}}}({[}channel\_number, ...{]})
&
\sphinxAtStartPar
Sets all necessary parameters for the computation of the spectrogram.
\\
\sphinxhline
\sphinxAtStartPar
{\hyperref[\detokenize{classes:dsptoolbox.classes.signal_class.Signal.set_spectrum_parameters}]{\sphinxcrossref{\sphinxcode{\sphinxupquote{set\_spectrum\_parameters}}}}}({[}method, smoothe, ...{]})
&
\sphinxAtStartPar
Sets all necessary parameters for the computation of the spectrum.
\\
\sphinxhline
\sphinxAtStartPar
{\hyperref[\detokenize{classes:dsptoolbox.classes.signal_class.Signal.set_window}]{\sphinxcrossref{\sphinxcode{\sphinxupquote{set\_window}}}}}(window)
&
\sphinxAtStartPar
Sets the window used for the IR.
\\
\sphinxhline
\sphinxAtStartPar
{\hyperref[\detokenize{classes:dsptoolbox.classes.signal_class.Signal.show_info}]{\sphinxcrossref{\sphinxcode{\sphinxupquote{show\_info}}}}}()
&
\sphinxAtStartPar
Prints all the signal information to the console.
\\
\sphinxhline
\sphinxAtStartPar
{\hyperref[\detokenize{classes:dsptoolbox.classes.signal_class.Signal.swap_channels}]{\sphinxcrossref{\sphinxcode{\sphinxupquote{swap\_channels}}}}}(new\_order)
&
\sphinxAtStartPar
Rearranges the channels in the new given order.
\\
\sphinxbottomrule
\end{tabulary}
\sphinxtableafterendhook\par
\sphinxattableend\end{savenotes}
\index{\_\_init\_\_() (dsptoolbox.classes.signal\_class.Signal method)@\spxentry{\_\_init\_\_()}\spxextra{dsptoolbox.classes.signal\_class.Signal method}}

\begin{fulllineitems}
\phantomsection\label{\detokenize{classes:dsptoolbox.classes.signal_class.Signal.__init__}}
\pysigstartsignatures
\pysiglinewithargsret{\sphinxbfcode{\sphinxupquote{\_\_init\_\_}}}{\emph{\DUrole{n}{path}\DUrole{o}{=}\DUrole{default_value}{None}}, \emph{\DUrole{n}{time\_data}\DUrole{o}{=}\DUrole{default_value}{None}}, \emph{\DUrole{n}{sampling\_rate\_hz}\DUrole{p}{:}\DUrole{w}{  }\DUrole{n}{\sphinxhref{https://docs.python.org/3/library/functions.html\#int}{int}}\DUrole{w}{  }\DUrole{o}{=}\DUrole{w}{  }\DUrole{default_value}{48000}}, \emph{\DUrole{n}{signal\_type}\DUrole{p}{:}\DUrole{w}{  }\DUrole{n}{\sphinxhref{https://docs.python.org/3/library/stdtypes.html\#str}{str}}\DUrole{w}{  }\DUrole{o}{=}\DUrole{w}{  }\DUrole{default_value}{\textquotesingle{}general\textquotesingle{}}}, \emph{\DUrole{n}{signal\_id}\DUrole{p}{:}\DUrole{w}{  }\DUrole{n}{\sphinxhref{https://docs.python.org/3/library/stdtypes.html\#str}{str}}\DUrole{w}{  }\DUrole{o}{=}\DUrole{w}{  }\DUrole{default_value}{\textquotesingle{}\textquotesingle{}}}}{}
\pysigstopsignatures
\sphinxAtStartPar
Signal class that saves mainly time data for being used with all the
methods.
\begin{quote}\begin{description}
\sphinxlineitem{Parameters}\begin{description}
\sphinxlineitem{\sphinxstylestrong{path}}{[}str, optional{]}
\sphinxAtStartPar
A path to audio files. Reading is done with soundfile. Wave and
Flac audio files are accepted.
Default: \sphinxtitleref{None}.

\sphinxlineitem{\sphinxstylestrong{time\_data}}{[}array\sphinxhyphen{}like, \sphinxtitleref{np.ndarray}, optional{]}
\sphinxAtStartPar
Time data of the signal. It is saved as a matrix with the form
(time samples, channel number). Default: \sphinxtitleref{None}.

\sphinxlineitem{\sphinxstylestrong{sampling\_rate\_hz}}{[}int, optional{]}
\sphinxAtStartPar
Sampling rate of the signal in Hz. Default: 48000.

\sphinxlineitem{\sphinxstylestrong{signal\_type}}{[}str, optional{]}
\sphinxAtStartPar
A generic signal type. Some functionalities are only unlocked for
impulse responses with \sphinxtitleref{‘ir’}, \sphinxtitleref{‘h1’}, \sphinxtitleref{‘h2’}, \sphinxtitleref{‘h3’} or \sphinxtitleref{‘rir’}.
Default: \sphinxtitleref{‘general’}.

\sphinxlineitem{\sphinxstylestrong{signal\_id}}{[}str, optional{]}
\sphinxAtStartPar
An even more generic signal id that can be used by the user.
Default: \sphinxtitleref{‘’}.

\end{description}

\end{description}\end{quote}
\subsubsection*{Methods}


\begin{savenotes}\sphinxattablestart
\sphinxthistablewithglobalstyle
\centering
\begin{tabulary}{\linewidth}[t]{|T|T|}
\sphinxtoprule
\sphinxtableatstartofbodyhook
\sphinxAtStartPar
\sphinxstylestrong{Time data:}
&
\sphinxAtStartPar
add\_channel, remove\_channel, swap\_channels.
\\
\sphinxhline
\sphinxAtStartPar
\sphinxstylestrong{Spectrum:}
&
\sphinxAtStartPar
set\_spectrum\_parameters, get\_spectrum.
\\
\sphinxhline
\sphinxAtStartPar
\sphinxstylestrong{Cross spectral matrix:}
&
\sphinxAtStartPar
set\_csm\_parameters, get\_csm.
\\
\sphinxhline
\sphinxAtStartPar
\sphinxstylestrong{Spectrogram:}
&
\sphinxAtStartPar
set\_spectrogram\_parameters, get\_spectrogram.
\\
\sphinxhline
\sphinxAtStartPar
\sphinxstylestrong{Plots:}
&
\sphinxAtStartPar
plot\_magnitude, plot\_time, plot\_spectrogram, plot\_phase, plot\_csm.
\\
\sphinxhline
\sphinxAtStartPar
\sphinxstylestrong{General:}
&
\sphinxAtStartPar
save\_signal.
\\
\sphinxhline
\sphinxAtStartPar
\sphinxstylestrong{Only for \textasciigrave{}signal\_type in (‘rir’, ‘ir’, ‘h1’, ‘h2’, ‘h3’)\textasciigrave{}:}
&
\sphinxAtStartPar
set\_window, set\_coherence, plot\_group\_delay, plot\_coherence.
\\
\sphinxbottomrule
\end{tabulary}
\sphinxtableafterendhook\par
\sphinxattableend\end{savenotes}

\end{fulllineitems}

\index{add\_channel() (dsptoolbox.classes.signal\_class.Signal method)@\spxentry{add\_channel()}\spxextra{dsptoolbox.classes.signal\_class.Signal method}}

\begin{fulllineitems}
\phantomsection\label{\detokenize{classes:dsptoolbox.classes.signal_class.Signal.add_channel}}
\pysigstartsignatures
\pysiglinewithargsret{\sphinxbfcode{\sphinxupquote{add\_channel}}}{\emph{\DUrole{n}{path}\DUrole{p}{:}\DUrole{w}{  }\DUrole{n}{\sphinxhref{https://docs.python.org/3/library/typing.html\#typing.Optional}{Optional}\DUrole{p}{{[}}\sphinxhref{https://docs.python.org/3/library/stdtypes.html\#str}{str}\DUrole{p}{{]}}}\DUrole{w}{  }\DUrole{o}{=}\DUrole{w}{  }\DUrole{default_value}{None}}, \emph{\DUrole{n}{new\_time\_data}\DUrole{p}{:}\DUrole{w}{  }\DUrole{n}{\sphinxhref{https://docs.python.org/3/library/typing.html\#typing.Optional}{Optional}\DUrole{p}{{[}}ndarray\DUrole{p}{{]}}}\DUrole{w}{  }\DUrole{o}{=}\DUrole{w}{  }\DUrole{default_value}{None}}, \emph{\DUrole{n}{sampling\_rate\_hz}\DUrole{p}{:}\DUrole{w}{  }\DUrole{n}{\sphinxhref{https://docs.python.org/3/library/typing.html\#typing.Optional}{Optional}\DUrole{p}{{[}}\sphinxhref{https://docs.python.org/3/library/functions.html\#int}{int}\DUrole{p}{{]}}}\DUrole{w}{  }\DUrole{o}{=}\DUrole{w}{  }\DUrole{default_value}{None}}, \emph{\DUrole{n}{padding\_trimming}\DUrole{p}{:}\DUrole{w}{  }\DUrole{n}{\sphinxhref{https://docs.python.org/3/library/functions.html\#bool}{bool}}\DUrole{w}{  }\DUrole{o}{=}\DUrole{w}{  }\DUrole{default_value}{True}}}{}
\pysigstopsignatures
\sphinxAtStartPar
Adds new channels to this signal object.
\begin{quote}\begin{description}
\sphinxlineitem{Parameters}\begin{description}
\sphinxlineitem{\sphinxstylestrong{path}}{[}str, optional{]}
\sphinxAtStartPar
Path to the file containing new channel information.

\sphinxlineitem{\sphinxstylestrong{new\_time\_data}}{[}\sphinxtitleref{np.ndarray}, optional{]}
\sphinxAtStartPar
np.array with new channel data.

\sphinxlineitem{\sphinxstylestrong{sampling\_rate\_hz}}{[}int, optional{]}
\sphinxAtStartPar
Sampling rate for the new data

\sphinxlineitem{\sphinxstylestrong{padding\_trimming}}{[}bool, optional{]}
\sphinxAtStartPar
Activates padding or trimming at the end of signal in case the
new data does not match previous data. Default: \sphinxtitleref{True}.

\end{description}

\end{description}\end{quote}

\end{fulllineitems}

\index{copy() (dsptoolbox.classes.signal\_class.Signal method)@\spxentry{copy()}\spxextra{dsptoolbox.classes.signal\_class.Signal method}}

\begin{fulllineitems}
\phantomsection\label{\detokenize{classes:dsptoolbox.classes.signal_class.Signal.copy}}
\pysigstartsignatures
\pysiglinewithargsret{\sphinxbfcode{\sphinxupquote{copy}}}{}{}
\pysigstopsignatures
\sphinxAtStartPar
Returns a copy of the object.
\begin{quote}\begin{description}
\sphinxlineitem{Returns}\begin{description}
\sphinxlineitem{\sphinxstylestrong{new\_sig}}{[}\sphinxtitleref{Signal}{]}
\sphinxAtStartPar
Copy of Signal.

\end{description}

\end{description}\end{quote}

\end{fulllineitems}

\index{get\_coherence() (dsptoolbox.classes.signal\_class.Signal method)@\spxentry{get\_coherence()}\spxextra{dsptoolbox.classes.signal\_class.Signal method}}

\begin{fulllineitems}
\phantomsection\label{\detokenize{classes:dsptoolbox.classes.signal_class.Signal.get_coherence}}
\pysigstartsignatures
\pysiglinewithargsret{\sphinxbfcode{\sphinxupquote{get\_coherence}}}{}{}
\pysigstopsignatures
\sphinxAtStartPar
Returns the coherence matrix.
\begin{quote}\begin{description}
\sphinxlineitem{Returns}\begin{description}
\sphinxlineitem{\sphinxstylestrong{f}}{[}\sphinxtitleref{np.ndarray}{]}
\sphinxAtStartPar
Frequency vector.

\sphinxlineitem{\sphinxstylestrong{coherence}}{[}\sphinxtitleref{np.ndarray}{]}
\sphinxAtStartPar
Coherence matrix.

\end{description}

\end{description}\end{quote}

\end{fulllineitems}

\index{get\_csm() (dsptoolbox.classes.signal\_class.Signal method)@\spxentry{get\_csm()}\spxextra{dsptoolbox.classes.signal\_class.Signal method}}

\begin{fulllineitems}
\phantomsection\label{\detokenize{classes:dsptoolbox.classes.signal_class.Signal.get_csm}}
\pysigstartsignatures
\pysiglinewithargsret{\sphinxbfcode{\sphinxupquote{get\_csm}}}{\emph{\DUrole{n}{force\_computation}\DUrole{o}{=}\DUrole{default_value}{False}}}{}
\pysigstopsignatures
\sphinxAtStartPar
Get Cross spectral matrix for all channels with the shape
(frequencies, channels, channels)
\begin{quote}\begin{description}
\sphinxlineitem{Returns}\begin{description}
\sphinxlineitem{\sphinxstylestrong{f\_csm}}{[}\sphinxtitleref{np.ndarray}{]}
\sphinxAtStartPar
Frequency vector.

\sphinxlineitem{\sphinxstylestrong{csm}}{[}\sphinxtitleref{np.ndarray}{]}
\sphinxAtStartPar
Cross spectral matrix with shape (frequency, channels, channels).

\end{description}

\end{description}\end{quote}

\end{fulllineitems}

\index{get\_spectrogram() (dsptoolbox.classes.signal\_class.Signal method)@\spxentry{get\_spectrogram()}\spxextra{dsptoolbox.classes.signal\_class.Signal method}}

\begin{fulllineitems}
\phantomsection\label{\detokenize{classes:dsptoolbox.classes.signal_class.Signal.get_spectrogram}}
\pysigstartsignatures
\pysiglinewithargsret{\sphinxbfcode{\sphinxupquote{get\_spectrogram}}}{\emph{\DUrole{n}{channel\_number}\DUrole{p}{:}\DUrole{w}{  }\DUrole{n}{\sphinxhref{https://docs.python.org/3/library/functions.html\#int}{int}}\DUrole{w}{  }\DUrole{o}{=}\DUrole{w}{  }\DUrole{default_value}{0}}, \emph{\DUrole{n}{force\_computation}\DUrole{p}{:}\DUrole{w}{  }\DUrole{n}{\sphinxhref{https://docs.python.org/3/library/functions.html\#bool}{bool}}\DUrole{w}{  }\DUrole{o}{=}\DUrole{w}{  }\DUrole{default_value}{False}}}{}
\pysigstopsignatures
\sphinxAtStartPar
Returns a matrix containing the STFT of a specific channel.
\begin{quote}\begin{description}
\sphinxlineitem{Parameters}\begin{description}
\sphinxlineitem{\sphinxstylestrong{channel\_number}}{[}int, optional{]}
\sphinxAtStartPar
Channel number for which to compute the STFT. Default: 0.

\sphinxlineitem{\sphinxstylestrong{force\_computation}}{[}bool, optional{]}
\sphinxAtStartPar
Forces new computation of the STFT. Default: False.

\end{description}

\sphinxlineitem{Returns}\begin{description}
\sphinxlineitem{\sphinxstylestrong{t\_s}}{[}\sphinxtitleref{np.ndarray}{]}
\sphinxAtStartPar
Time vector.

\sphinxlineitem{\sphinxstylestrong{f\_hz}}{[}\sphinxtitleref{np.ndarray}{]}
\sphinxAtStartPar
Frequency vector.

\sphinxlineitem{\sphinxstylestrong{spectrogram}}{[}\sphinxtitleref{np.ndarray}{]}
\sphinxAtStartPar
Spectrogram.

\end{description}

\end{description}\end{quote}

\end{fulllineitems}

\index{get\_spectrum() (dsptoolbox.classes.signal\_class.Signal method)@\spxentry{get\_spectrum()}\spxextra{dsptoolbox.classes.signal\_class.Signal method}}

\begin{fulllineitems}
\phantomsection\label{\detokenize{classes:dsptoolbox.classes.signal_class.Signal.get_spectrum}}
\pysigstartsignatures
\pysiglinewithargsret{\sphinxbfcode{\sphinxupquote{get\_spectrum}}}{\emph{\DUrole{n}{force\_computation}\DUrole{o}{=}\DUrole{default_value}{False}}}{}
\pysigstopsignatures
\sphinxAtStartPar
Returns spectrum.
\begin{quote}\begin{description}
\sphinxlineitem{Parameters}\begin{description}
\sphinxlineitem{\sphinxstylestrong{force\_computation}}{[}bool, optional{]}
\sphinxAtStartPar
Forces spectrum computation.

\end{description}

\sphinxlineitem{Returns}\begin{description}
\sphinxlineitem{\sphinxstylestrong{spectrum\_freqs}}{[}\sphinxtitleref{np.ndarray}{]}
\sphinxAtStartPar
Frequency vector.

\sphinxlineitem{\sphinxstylestrong{spectrum}}{[}\sphinxtitleref{np.ndarray}{]}
\sphinxAtStartPar
Spectrum matrix for each channel.

\end{description}

\end{description}\end{quote}

\end{fulllineitems}

\index{get\_time\_vector() (dsptoolbox.classes.signal\_class.Signal method)@\spxentry{get\_time\_vector()}\spxextra{dsptoolbox.classes.signal\_class.Signal method}}

\begin{fulllineitems}
\phantomsection\label{\detokenize{classes:dsptoolbox.classes.signal_class.Signal.get_time_vector}}
\pysigstartsignatures
\pysiglinewithargsret{\sphinxbfcode{\sphinxupquote{get\_time\_vector}}}{}{}
\pysigstopsignatures
\sphinxAtStartPar
Returns the time vector associated with the signal.
\begin{quote}\begin{description}
\sphinxlineitem{Returns}\begin{description}
\sphinxlineitem{\sphinxstylestrong{time\_vector\_s}}{[}\sphinxtitleref{np.ndarray}{]}
\sphinxAtStartPar
Time vector in seconds.

\end{description}

\end{description}\end{quote}

\end{fulllineitems}

\index{number\_of\_channels (dsptoolbox.classes.signal\_class.Signal property)@\spxentry{number\_of\_channels}\spxextra{dsptoolbox.classes.signal\_class.Signal property}}

\begin{fulllineitems}
\phantomsection\label{\detokenize{classes:dsptoolbox.classes.signal_class.Signal.number_of_channels}}
\pysigstartsignatures
\pysigline{\sphinxbfcode{\sphinxupquote{property\DUrole{w}{  }}}\sphinxbfcode{\sphinxupquote{number\_of\_channels}}}
\pysigstopsignatures
\end{fulllineitems}

\index{plot\_coherence() (dsptoolbox.classes.signal\_class.Signal method)@\spxentry{plot\_coherence()}\spxextra{dsptoolbox.classes.signal\_class.Signal method}}

\begin{fulllineitems}
\phantomsection\label{\detokenize{classes:dsptoolbox.classes.signal_class.Signal.plot_coherence}}
\pysigstartsignatures
\pysiglinewithargsret{\sphinxbfcode{\sphinxupquote{plot\_coherence}}}{\emph{\DUrole{n}{returns}\DUrole{p}{:}\DUrole{w}{  }\DUrole{n}{\sphinxhref{https://docs.python.org/3/library/functions.html\#bool}{bool}}\DUrole{w}{  }\DUrole{o}{=}\DUrole{w}{  }\DUrole{default_value}{False}}}{}
\pysigstopsignatures
\sphinxAtStartPar
Plots coherence measurements if there are any.
\begin{quote}\begin{description}
\sphinxlineitem{Parameters}\begin{description}
\sphinxlineitem{\sphinxstylestrong{returns}}{[}bool, optional{]}
\sphinxAtStartPar
When \sphinxtitleref{True}, the plot’s figure and axis are returned.
Default: \sphinxtitleref{False}.

\end{description}

\sphinxlineitem{Returns}\begin{description}
\sphinxlineitem{fig, ax}
\sphinxAtStartPar
Only returned if \sphinxtitleref{returns=True}.

\end{description}

\end{description}\end{quote}

\end{fulllineitems}

\index{plot\_csm() (dsptoolbox.classes.signal\_class.Signal method)@\spxentry{plot\_csm()}\spxextra{dsptoolbox.classes.signal\_class.Signal method}}

\begin{fulllineitems}
\phantomsection\label{\detokenize{classes:dsptoolbox.classes.signal_class.Signal.plot_csm}}
\pysigstartsignatures
\pysiglinewithargsret{\sphinxbfcode{\sphinxupquote{plot\_csm}}}{\emph{\DUrole{n}{range\_hz}\DUrole{o}{=}\DUrole{default_value}{{[}20, 20000.0{]}}}, \emph{\DUrole{n}{logx}\DUrole{p}{:}\DUrole{w}{  }\DUrole{n}{\sphinxhref{https://docs.python.org/3/library/functions.html\#bool}{bool}}\DUrole{w}{  }\DUrole{o}{=}\DUrole{w}{  }\DUrole{default_value}{True}}, \emph{\DUrole{n}{with\_phase}\DUrole{p}{:}\DUrole{w}{  }\DUrole{n}{\sphinxhref{https://docs.python.org/3/library/functions.html\#bool}{bool}}\DUrole{w}{  }\DUrole{o}{=}\DUrole{w}{  }\DUrole{default_value}{True}}, \emph{\DUrole{n}{returns}\DUrole{p}{:}\DUrole{w}{  }\DUrole{n}{\sphinxhref{https://docs.python.org/3/library/functions.html\#bool}{bool}}\DUrole{w}{  }\DUrole{o}{=}\DUrole{w}{  }\DUrole{default_value}{False}}}{}
\pysigstopsignatures
\sphinxAtStartPar
Plots the cross spectral matrix of the multichannel signal.
\begin{quote}\begin{description}
\sphinxlineitem{Parameters}\begin{description}
\sphinxlineitem{\sphinxstylestrong{range\_hz}}{[}array\sphinxhyphen{}like with length 2, optional{]}
\sphinxAtStartPar
Range of Hz to be showed. Default: {[}20, 20e3{]}.

\sphinxlineitem{\sphinxstylestrong{logx}}{[}bool, optional{]}
\sphinxAtStartPar
Logarithmic x axis. Default: \sphinxtitleref{True}.

\sphinxlineitem{\sphinxstylestrong{with\_phase}}{[}bool, optional{]}
\sphinxAtStartPar
When \sphinxtitleref{True}, the unwrapped phase is also plotted. Default: \sphinxtitleref{True}.

\sphinxlineitem{\sphinxstylestrong{returns}}{[}bool, optional{]}
\sphinxAtStartPar
Gives back figure and axis objects. Default: \sphinxtitleref{False}.

\end{description}

\sphinxlineitem{Returns}\begin{description}
\sphinxlineitem{fig, ax if \sphinxtitleref{returns = True}.}
\end{description}

\end{description}\end{quote}

\end{fulllineitems}

\index{plot\_group\_delay() (dsptoolbox.classes.signal\_class.Signal method)@\spxentry{plot\_group\_delay()}\spxextra{dsptoolbox.classes.signal\_class.Signal method}}

\begin{fulllineitems}
\phantomsection\label{\detokenize{classes:dsptoolbox.classes.signal_class.Signal.plot_group_delay}}
\pysigstartsignatures
\pysiglinewithargsret{\sphinxbfcode{\sphinxupquote{plot\_group\_delay}}}{\emph{\DUrole{n}{range\_hz}\DUrole{o}{=}\DUrole{default_value}{{[}20, 20000{]}}}, \emph{\DUrole{n}{returns}\DUrole{p}{:}\DUrole{w}{  }\DUrole{n}{\sphinxhref{https://docs.python.org/3/library/functions.html\#bool}{bool}}\DUrole{w}{  }\DUrole{o}{=}\DUrole{w}{  }\DUrole{default_value}{False}}}{}
\pysigstopsignatures
\sphinxAtStartPar
Plots group delay of each channel.
Only works if \sphinxtitleref{signal\_type in (‘ir’, ‘h1’, ‘h2’, ‘h3’, ‘rir’, chirp,
noise, dirac)}.
\begin{quote}\begin{description}
\sphinxlineitem{Parameters}\begin{description}
\sphinxlineitem{\sphinxstylestrong{range\_hz}}{[}array\sphinxhyphen{}like with length 2, optional{]}
\sphinxAtStartPar
Range of frequencies for which to show group delay.
Default: {[}20, 20e3{]}.

\sphinxlineitem{\sphinxstylestrong{returns}}{[}bool, optional{]}
\sphinxAtStartPar
When \sphinxtitleref{True}, the plot’s figure and axis are returned.
Default: \sphinxtitleref{False}.

\end{description}

\sphinxlineitem{Returns}\begin{description}
\sphinxlineitem{fig, ax}
\sphinxAtStartPar
Only returned if \sphinxtitleref{returns=True}.

\end{description}

\end{description}\end{quote}

\end{fulllineitems}

\index{plot\_magnitude() (dsptoolbox.classes.signal\_class.Signal method)@\spxentry{plot\_magnitude()}\spxextra{dsptoolbox.classes.signal\_class.Signal method}}

\begin{fulllineitems}
\phantomsection\label{\detokenize{classes:dsptoolbox.classes.signal_class.Signal.plot_magnitude}}
\pysigstartsignatures
\pysiglinewithargsret{\sphinxbfcode{\sphinxupquote{plot\_magnitude}}}{\emph{\DUrole{n}{range\_hz}\DUrole{o}{=}\DUrole{default_value}{{[}20, 20000.0{]}}}, \emph{\DUrole{n}{normalize}\DUrole{p}{:}\DUrole{w}{  }\DUrole{n}{\sphinxhref{https://docs.python.org/3/library/stdtypes.html\#str}{str}}\DUrole{w}{  }\DUrole{o}{=}\DUrole{w}{  }\DUrole{default_value}{\textquotesingle{}1k\textquotesingle{}}}, \emph{\DUrole{n}{range\_db}\DUrole{o}{=}\DUrole{default_value}{None}}, \emph{\DUrole{n}{smoothe}\DUrole{p}{:}\DUrole{w}{  }\DUrole{n}{\sphinxhref{https://docs.python.org/3/library/functions.html\#int}{int}}\DUrole{w}{  }\DUrole{o}{=}\DUrole{w}{  }\DUrole{default_value}{0}}, \emph{\DUrole{n}{show\_info\_box}\DUrole{p}{:}\DUrole{w}{  }\DUrole{n}{\sphinxhref{https://docs.python.org/3/library/functions.html\#bool}{bool}}\DUrole{w}{  }\DUrole{o}{=}\DUrole{w}{  }\DUrole{default_value}{False}}, \emph{\DUrole{n}{returns}\DUrole{p}{:}\DUrole{w}{  }\DUrole{n}{\sphinxhref{https://docs.python.org/3/library/functions.html\#bool}{bool}}\DUrole{w}{  }\DUrole{o}{=}\DUrole{w}{  }\DUrole{default_value}{False}}}{}
\pysigstopsignatures
\sphinxAtStartPar
Plots magnitude spectrum.
Change parameters of spectrum with set\_spectrum\_parameters.
NOTE: Smoothing is only applied on the plot data.
\begin{quote}\begin{description}
\sphinxlineitem{Parameters}\begin{description}
\sphinxlineitem{\sphinxstylestrong{range\_hz}}{[}array\sphinxhyphen{}like with length 2, optional{]}
\sphinxAtStartPar
Range for which to plot the magnitude response.
Default: {[}20, 20000{]}.

\sphinxlineitem{\sphinxstylestrong{normalize}}{[}str, optional{]}
\sphinxAtStartPar
Mode for normalization, supported are \sphinxtitleref{‘1k’} for normalization
with value at frequency 1 kHz or \sphinxtitleref{‘max’} for normalization with
maximal value. Use \sphinxtitleref{None} for no normalization. Default: \sphinxtitleref{‘1k’}.

\sphinxlineitem{\sphinxstylestrong{range\_db}}{[}array\sphinxhyphen{}like with length 2, optional{]}
\sphinxAtStartPar
Range in dB for which to plot the magnitude response.
Default: \sphinxtitleref{None}.

\sphinxlineitem{\sphinxstylestrong{smoothe}}{[}int, optional{]}
\sphinxAtStartPar
Smoothing across the (1/smoothe) octave band.
Default: 0 (no smoothing).

\sphinxlineitem{\sphinxstylestrong{show\_info\_box}}{[}bool, optional{]}
\sphinxAtStartPar
Plots a info box regarding spectrum parameters and plot parameters.
If it is str, it overwrites the standard message.
Default: \sphinxtitleref{False}.

\sphinxlineitem{\sphinxstylestrong{returns}}{[}bool, optional{]}
\sphinxAtStartPar
When \sphinxtitleref{True} figure and axis are returned. Default: \sphinxtitleref{False}.

\end{description}

\sphinxlineitem{Returns}\begin{description}
\sphinxlineitem{fig, ax}
\sphinxAtStartPar
Only returned if \sphinxtitleref{returns=True}.

\end{description}

\end{description}\end{quote}

\end{fulllineitems}

\index{plot\_phase() (dsptoolbox.classes.signal\_class.Signal method)@\spxentry{plot\_phase()}\spxextra{dsptoolbox.classes.signal\_class.Signal method}}

\begin{fulllineitems}
\phantomsection\label{\detokenize{classes:dsptoolbox.classes.signal_class.Signal.plot_phase}}
\pysigstartsignatures
\pysiglinewithargsret{\sphinxbfcode{\sphinxupquote{plot\_phase}}}{\emph{\DUrole{n}{range\_hz}\DUrole{o}{=}\DUrole{default_value}{{[}20, 20000.0{]}}}, \emph{\DUrole{n}{unwrap}\DUrole{p}{:}\DUrole{w}{  }\DUrole{n}{\sphinxhref{https://docs.python.org/3/library/functions.html\#bool}{bool}}\DUrole{w}{  }\DUrole{o}{=}\DUrole{w}{  }\DUrole{default_value}{False}}, \emph{\DUrole{n}{returns}\DUrole{p}{:}\DUrole{w}{  }\DUrole{n}{\sphinxhref{https://docs.python.org/3/library/functions.html\#bool}{bool}}\DUrole{w}{  }\DUrole{o}{=}\DUrole{w}{  }\DUrole{default_value}{False}}}{}
\pysigstopsignatures
\sphinxAtStartPar
Plots phase of the frequency response, only available if the method
for the spectrum parameters is not welch.
\begin{quote}\begin{description}
\sphinxlineitem{Parameters}\begin{description}
\sphinxlineitem{\sphinxstylestrong{range\_hz}}{[}array\sphinxhyphen{}like with length 2, optional{]}
\sphinxAtStartPar
Range of frequencies for which to show group delay.
Default: {[}20, 20e3{]}.

\sphinxlineitem{\sphinxstylestrong{unwrap}}{[}bool, optional{]}
\sphinxAtStartPar
When \sphinxtitleref{True}, the unwrapped phase is plotted. Default: \sphinxtitleref{False}.

\sphinxlineitem{\sphinxstylestrong{returns}}{[}bool, optional{]}
\sphinxAtStartPar
When \sphinxtitleref{True}, the plot’s figure and axis are returned.
Default: \sphinxtitleref{False}.

\end{description}

\sphinxlineitem{Returns}\begin{description}
\sphinxlineitem{fig, ax}
\sphinxAtStartPar
Only returned if \sphinxtitleref{returns=True}.

\end{description}

\end{description}\end{quote}

\end{fulllineitems}

\index{plot\_spectrogram() (dsptoolbox.classes.signal\_class.Signal method)@\spxentry{plot\_spectrogram()}\spxextra{dsptoolbox.classes.signal\_class.Signal method}}

\begin{fulllineitems}
\phantomsection\label{\detokenize{classes:dsptoolbox.classes.signal_class.Signal.plot_spectrogram}}
\pysigstartsignatures
\pysiglinewithargsret{\sphinxbfcode{\sphinxupquote{plot\_spectrogram}}}{\emph{\DUrole{n}{channel\_number}\DUrole{p}{:}\DUrole{w}{  }\DUrole{n}{\sphinxhref{https://docs.python.org/3/library/functions.html\#int}{int}}\DUrole{w}{  }\DUrole{o}{=}\DUrole{w}{  }\DUrole{default_value}{0}}, \emph{\DUrole{n}{logfreqs}\DUrole{p}{:}\DUrole{w}{  }\DUrole{n}{\sphinxhref{https://docs.python.org/3/library/functions.html\#bool}{bool}}\DUrole{w}{  }\DUrole{o}{=}\DUrole{w}{  }\DUrole{default_value}{True}}, \emph{\DUrole{n}{returns}\DUrole{p}{:}\DUrole{w}{  }\DUrole{n}{\sphinxhref{https://docs.python.org/3/library/functions.html\#bool}{bool}}\DUrole{w}{  }\DUrole{o}{=}\DUrole{w}{  }\DUrole{default_value}{False}}}{}
\pysigstopsignatures
\sphinxAtStartPar
Plots STFT matrix of the given channel.
\begin{quote}\begin{description}
\sphinxlineitem{Parameters}\begin{description}
\sphinxlineitem{\sphinxstylestrong{channel\_number}}{[}int, optional{]}
\sphinxAtStartPar
Selected channel to plot spectrogram. Default: 0 (first).

\sphinxlineitem{\sphinxstylestrong{logfreqs}}{[}bool, optional{]}
\sphinxAtStartPar
When \sphinxtitleref{True}, frequency axis is plotted logarithmically.
Default: \sphinxtitleref{True}.

\sphinxlineitem{\sphinxstylestrong{returns}}{[}bool, optional{]}
\sphinxAtStartPar
When \sphinxtitleref{True}, the plot’s figure and axis are returned.
Default: \sphinxtitleref{False}.

\end{description}

\sphinxlineitem{Returns}\begin{description}
\sphinxlineitem{fig, ax}
\sphinxAtStartPar
Only returned if \sphinxtitleref{returns=True}.

\end{description}

\end{description}\end{quote}

\end{fulllineitems}

\index{plot\_time() (dsptoolbox.classes.signal\_class.Signal method)@\spxentry{plot\_time()}\spxextra{dsptoolbox.classes.signal\_class.Signal method}}

\begin{fulllineitems}
\phantomsection\label{\detokenize{classes:dsptoolbox.classes.signal_class.Signal.plot_time}}
\pysigstartsignatures
\pysiglinewithargsret{\sphinxbfcode{\sphinxupquote{plot\_time}}}{\emph{\DUrole{n}{returns}\DUrole{p}{:}\DUrole{w}{  }\DUrole{n}{\sphinxhref{https://docs.python.org/3/library/functions.html\#bool}{bool}}\DUrole{w}{  }\DUrole{o}{=}\DUrole{w}{  }\DUrole{default_value}{False}}}{}
\pysigstopsignatures
\sphinxAtStartPar
Plots time signals.
\begin{quote}\begin{description}
\sphinxlineitem{Parameters}\begin{description}
\sphinxlineitem{\sphinxstylestrong{returns}}{[}bool, optional{]}
\sphinxAtStartPar
When \sphinxtitleref{True}, the plot’s figure and axis are returned.
Default: \sphinxtitleref{False}.

\end{description}

\sphinxlineitem{Returns}\begin{description}
\sphinxlineitem{fig, ax}
\sphinxAtStartPar
Only returned if \sphinxtitleref{returns=True}.

\end{description}

\end{description}\end{quote}

\end{fulllineitems}

\index{remove\_channel() (dsptoolbox.classes.signal\_class.Signal method)@\spxentry{remove\_channel()}\spxextra{dsptoolbox.classes.signal\_class.Signal method}}

\begin{fulllineitems}
\phantomsection\label{\detokenize{classes:dsptoolbox.classes.signal_class.Signal.remove_channel}}
\pysigstartsignatures
\pysiglinewithargsret{\sphinxbfcode{\sphinxupquote{remove\_channel}}}{\emph{\DUrole{n}{channel\_number}\DUrole{p}{:}\DUrole{w}{  }\DUrole{n}{\sphinxhref{https://docs.python.org/3/library/functions.html\#int}{int}}\DUrole{w}{  }\DUrole{o}{=}\DUrole{w}{  }\DUrole{default_value}{\sphinxhyphen{}1}}}{}
\pysigstopsignatures
\sphinxAtStartPar
Removes a channel.
\begin{quote}\begin{description}
\sphinxlineitem{Parameters}\begin{description}
\sphinxlineitem{\sphinxstylestrong{channel\_number}}{[}int, optional{]}
\sphinxAtStartPar
Channel number to be removed. Default: \sphinxhyphen{}1 (last).

\end{description}

\end{description}\end{quote}

\end{fulllineitems}

\index{sampling\_rate\_hz (dsptoolbox.classes.signal\_class.Signal property)@\spxentry{sampling\_rate\_hz}\spxextra{dsptoolbox.classes.signal\_class.Signal property}}

\begin{fulllineitems}
\phantomsection\label{\detokenize{classes:dsptoolbox.classes.signal_class.Signal.sampling_rate_hz}}
\pysigstartsignatures
\pysigline{\sphinxbfcode{\sphinxupquote{property\DUrole{w}{  }}}\sphinxbfcode{\sphinxupquote{sampling\_rate\_hz}}}
\pysigstopsignatures
\end{fulllineitems}

\index{save\_signal() (dsptoolbox.classes.signal\_class.Signal method)@\spxentry{save\_signal()}\spxextra{dsptoolbox.classes.signal\_class.Signal method}}

\begin{fulllineitems}
\phantomsection\label{\detokenize{classes:dsptoolbox.classes.signal_class.Signal.save_signal}}
\pysigstartsignatures
\pysiglinewithargsret{\sphinxbfcode{\sphinxupquote{save\_signal}}}{\emph{\DUrole{n}{path}\DUrole{p}{:}\DUrole{w}{  }\DUrole{n}{\sphinxhref{https://docs.python.org/3/library/stdtypes.html\#str}{str}}\DUrole{w}{  }\DUrole{o}{=}\DUrole{w}{  }\DUrole{default_value}{\textquotesingle{}signal\textquotesingle{}}}, \emph{\DUrole{n}{mode}\DUrole{p}{:}\DUrole{w}{  }\DUrole{n}{\sphinxhref{https://docs.python.org/3/library/stdtypes.html\#str}{str}}\DUrole{w}{  }\DUrole{o}{=}\DUrole{w}{  }\DUrole{default_value}{\textquotesingle{}wav\textquotesingle{}}}}{}
\pysigstopsignatures
\sphinxAtStartPar
Saves the Signal object as wav, flac or pickle.
\begin{quote}\begin{description}
\sphinxlineitem{Parameters}\begin{description}
\sphinxlineitem{\sphinxstylestrong{path}}{[}str, optional{]}
\sphinxAtStartPar
Path for the signal to be saved. Use only folders/name
(without format). Default: \sphinxtitleref{‘signal’}
(local folder, object named signal).

\sphinxlineitem{\sphinxstylestrong{mode}}{[}str, optional{]}
\sphinxAtStartPar
Mode of saving. Available modes are \sphinxtitleref{‘wav’}, \sphinxtitleref{‘flac’}, \sphinxtitleref{‘pickle’}.
Default: \sphinxtitleref{‘wav’}.

\end{description}

\end{description}\end{quote}

\end{fulllineitems}

\index{set\_coherence() (dsptoolbox.classes.signal\_class.Signal method)@\spxentry{set\_coherence()}\spxextra{dsptoolbox.classes.signal\_class.Signal method}}

\begin{fulllineitems}
\phantomsection\label{\detokenize{classes:dsptoolbox.classes.signal_class.Signal.set_coherence}}
\pysigstartsignatures
\pysiglinewithargsret{\sphinxbfcode{\sphinxupquote{set\_coherence}}}{\emph{\DUrole{n}{coherence}\DUrole{p}{:}\DUrole{w}{  }\DUrole{n}{ndarray}}}{}
\pysigstopsignatures
\sphinxAtStartPar
Sets the coherence measurements of the transfer function.
It only works for \sphinxtitleref{signal\_type = (‘ir’, ‘h1’, ‘h2’, ‘h3’, ‘rir’)}.
\begin{quote}\begin{description}
\sphinxlineitem{Parameters}\begin{description}
\sphinxlineitem{\sphinxstylestrong{coherence}}{[}\sphinxtitleref{np.ndarray}{]}
\sphinxAtStartPar
Coherence matrix.

\end{description}

\end{description}\end{quote}

\end{fulllineitems}

\index{set\_csm\_parameters() (dsptoolbox.classes.signal\_class.Signal method)@\spxentry{set\_csm\_parameters()}\spxextra{dsptoolbox.classes.signal\_class.Signal method}}

\begin{fulllineitems}
\phantomsection\label{\detokenize{classes:dsptoolbox.classes.signal_class.Signal.set_csm_parameters}}
\pysigstartsignatures
\pysiglinewithargsret{\sphinxbfcode{\sphinxupquote{set\_csm\_parameters}}}{\emph{\DUrole{n}{window\_length\_samples}\DUrole{p}{:}\DUrole{w}{  }\DUrole{n}{\sphinxhref{https://docs.python.org/3/library/functions.html\#int}{int}}\DUrole{w}{  }\DUrole{o}{=}\DUrole{w}{  }\DUrole{default_value}{1024}}, \emph{\DUrole{n}{window\_type}\DUrole{o}{=}\DUrole{default_value}{\textquotesingle{}hann\textquotesingle{}}}, \emph{\DUrole{n}{overlap\_percent}\DUrole{o}{=}\DUrole{default_value}{75}}, \emph{\DUrole{n}{detrend}\DUrole{o}{=}\DUrole{default_value}{True}}, \emph{\DUrole{n}{average}\DUrole{o}{=}\DUrole{default_value}{\textquotesingle{}mean\textquotesingle{}}}, \emph{\DUrole{n}{scaling}\DUrole{o}{=}\DUrole{default_value}{\textquotesingle{}power\textquotesingle{}}}}{}
\pysigstopsignatures
\sphinxAtStartPar
Sets all necessary parameters for the computation of the CSM.
\begin{quote}\begin{description}
\sphinxlineitem{Parameters}\begin{description}
\sphinxlineitem{\sphinxstylestrong{window\_length\_samples}}{[}int, optional{]}
\sphinxAtStartPar
Window size. Default: 1024.

\sphinxlineitem{\sphinxstylestrong{overlap\_percent}}{[}float, optional{]}
\sphinxAtStartPar
Overlap in percent. Default: 75.

\sphinxlineitem{\sphinxstylestrong{detrend}}{[}bool, optional{]}
\sphinxAtStartPar
Detrending (subtracting mean). Default: True.

\sphinxlineitem{\sphinxstylestrong{average}}{[}str, optional{]}
\sphinxAtStartPar
Averaging method. Choose from \sphinxtitleref{‘mean’} or \sphinxtitleref{‘median’}.
Default: \sphinxtitleref{‘mean’}.

\sphinxlineitem{\sphinxstylestrong{scaling}}{[}str, optional{]}
\sphinxAtStartPar
Type of scaling. ‘\sphinxtitleref{power’} or \sphinxtitleref{‘spectrum’}. Default: \sphinxtitleref{‘power’}.

\end{description}

\end{description}\end{quote}

\end{fulllineitems}

\index{set\_spectrogram\_parameters() (dsptoolbox.classes.signal\_class.Signal method)@\spxentry{set\_spectrogram\_parameters()}\spxextra{dsptoolbox.classes.signal\_class.Signal method}}

\begin{fulllineitems}
\phantomsection\label{\detokenize{classes:dsptoolbox.classes.signal_class.Signal.set_spectrogram_parameters}}
\pysigstartsignatures
\pysiglinewithargsret{\sphinxbfcode{\sphinxupquote{set\_spectrogram\_parameters}}}{\emph{\DUrole{n}{channel\_number}\DUrole{p}{:}\DUrole{w}{  }\DUrole{n}{\sphinxhref{https://docs.python.org/3/library/functions.html\#int}{int}}\DUrole{w}{  }\DUrole{o}{=}\DUrole{w}{  }\DUrole{default_value}{0}}, \emph{\DUrole{n}{window\_length\_samples}\DUrole{p}{:}\DUrole{w}{  }\DUrole{n}{\sphinxhref{https://docs.python.org/3/library/functions.html\#int}{int}}\DUrole{w}{  }\DUrole{o}{=}\DUrole{w}{  }\DUrole{default_value}{1024}}, \emph{\DUrole{n}{window\_type}\DUrole{p}{:}\DUrole{w}{  }\DUrole{n}{\sphinxhref{https://docs.python.org/3/library/stdtypes.html\#str}{str}}\DUrole{w}{  }\DUrole{o}{=}\DUrole{w}{  }\DUrole{default_value}{\textquotesingle{}hann\textquotesingle{}}}, \emph{\DUrole{n}{overlap\_percent}\DUrole{o}{=}\DUrole{default_value}{75}}, \emph{\DUrole{n}{detrend}\DUrole{p}{:}\DUrole{w}{  }\DUrole{n}{\sphinxhref{https://docs.python.org/3/library/functions.html\#bool}{bool}}\DUrole{w}{  }\DUrole{o}{=}\DUrole{w}{  }\DUrole{default_value}{True}}, \emph{\DUrole{n}{padding}\DUrole{p}{:}\DUrole{w}{  }\DUrole{n}{\sphinxhref{https://docs.python.org/3/library/functions.html\#bool}{bool}}\DUrole{w}{  }\DUrole{o}{=}\DUrole{w}{  }\DUrole{default_value}{True}}, \emph{\DUrole{n}{scaling}\DUrole{p}{:}\DUrole{w}{  }\DUrole{n}{\sphinxhref{https://docs.python.org/3/library/functions.html\#bool}{bool}}\DUrole{w}{  }\DUrole{o}{=}\DUrole{w}{  }\DUrole{default_value}{False}}}{}
\pysigstopsignatures
\sphinxAtStartPar
Sets all necessary parameters for the computation of the
spectrogram.
\begin{quote}\begin{description}
\sphinxlineitem{Parameters}\begin{description}
\sphinxlineitem{\sphinxstylestrong{window\_length\_samples}}{[}int, optional{]}
\sphinxAtStartPar
Window size. Default: 1024.

\sphinxlineitem{\sphinxstylestrong{overlap\_percent}}{[}float, optional{]}
\sphinxAtStartPar
Overlap in percent. Default: 75.

\sphinxlineitem{\sphinxstylestrong{detrend}}{[}bool, optional{]}
\sphinxAtStartPar
Detrending (subtracting mean). Default: True.

\sphinxlineitem{\sphinxstylestrong{padding}}{[}bool, optional{]}
\sphinxAtStartPar
Padding signal in the beginning and end to center it.
Default: True.

\sphinxlineitem{\sphinxstylestrong{scaling}}{[}bool, optional{]}
\sphinxAtStartPar
Scaling or not after np.fft. Default: False.

\end{description}

\end{description}\end{quote}

\end{fulllineitems}

\index{set\_spectrum\_parameters() (dsptoolbox.classes.signal\_class.Signal method)@\spxentry{set\_spectrum\_parameters()}\spxextra{dsptoolbox.classes.signal\_class.Signal method}}

\begin{fulllineitems}
\phantomsection\label{\detokenize{classes:dsptoolbox.classes.signal_class.Signal.set_spectrum_parameters}}
\pysigstartsignatures
\pysiglinewithargsret{\sphinxbfcode{\sphinxupquote{set\_spectrum\_parameters}}}{\emph{\DUrole{n}{method}\DUrole{o}{=}\DUrole{default_value}{\textquotesingle{}welch\textquotesingle{}}}, \emph{\DUrole{n}{smoothe}\DUrole{p}{:}\DUrole{w}{  }\DUrole{n}{\sphinxhref{https://docs.python.org/3/library/functions.html\#int}{int}}\DUrole{w}{  }\DUrole{o}{=}\DUrole{w}{  }\DUrole{default_value}{0}}, \emph{\DUrole{n}{window\_length\_samples}\DUrole{p}{:}\DUrole{w}{  }\DUrole{n}{\sphinxhref{https://docs.python.org/3/library/functions.html\#int}{int}}\DUrole{w}{  }\DUrole{o}{=}\DUrole{w}{  }\DUrole{default_value}{1024}}, \emph{\DUrole{n}{window\_type}\DUrole{o}{=}\DUrole{default_value}{\textquotesingle{}hann\textquotesingle{}}}, \emph{\DUrole{n}{overlap\_percent}\DUrole{o}{=}\DUrole{default_value}{50}}, \emph{\DUrole{n}{detrend}\DUrole{o}{=}\DUrole{default_value}{True}}, \emph{\DUrole{n}{average}\DUrole{o}{=}\DUrole{default_value}{\textquotesingle{}mean\textquotesingle{}}}, \emph{\DUrole{n}{scaling}\DUrole{o}{=}\DUrole{default_value}{\textquotesingle{}power\textquotesingle{}}}}{}
\pysigstopsignatures
\sphinxAtStartPar
Sets all necessary parameters for the computation of the spectrum.
\begin{quote}\begin{description}
\sphinxlineitem{Parameters}\begin{description}
\sphinxlineitem{\sphinxstylestrong{method}}{[}str, optional{]}
\sphinxAtStartPar
\sphinxtitleref{‘welch’} or \sphinxtitleref{‘standard’}. Default: \sphinxtitleref{‘welch’}.

\sphinxlineitem{\sphinxstylestrong{smoothe}}{[}int, optional{]}
\sphinxAtStartPar
Smoothing across (1/smoothe) octave bands using a hamming
window. Smoothes magnitude AND phase. For accesing the smoothing
algorithm, refer to
\sphinxtitleref{dsptoolbox.\_general\_helpers.\_fractional\_octave\_smoothing()}.
If smoothing is applied here, \sphinxtitleref{Signal.get\_spectrum()} returns
the smoothed spectrum as well.
Default: 0 (no smoothing).

\sphinxlineitem{\sphinxstylestrong{window\_length\_samples}}{[}int, optional{]}
\sphinxAtStartPar
Window size. Default: 1024.

\sphinxlineitem{\sphinxstylestrong{window\_type}}{[}str,optional{]}
\sphinxAtStartPar
Choose type of window from the options in scipy.windows.
Default: \sphinxtitleref{‘hann’}.

\sphinxlineitem{\sphinxstylestrong{overlap\_percent}}{[}float, optional{]}
\sphinxAtStartPar
Overlap in percent. Default: 50.

\sphinxlineitem{\sphinxstylestrong{detrend}}{[}bool, optional{]}
\sphinxAtStartPar
Detrending (subtracting mean). Default: True.

\sphinxlineitem{\sphinxstylestrong{average}}{[}str, optional{]}
\sphinxAtStartPar
Averaging method. Choose from \sphinxtitleref{‘mean’} or \sphinxtitleref{‘median’}.
Default: \sphinxtitleref{‘mean’}.

\sphinxlineitem{\sphinxstylestrong{scaling}}{[}str, optional{]}
\sphinxAtStartPar
Type of scaling. ‘\sphinxtitleref{power’} or \sphinxtitleref{‘spectrum’}. Default: \sphinxtitleref{‘power’}.

\end{description}

\end{description}\end{quote}

\end{fulllineitems}

\index{set\_window() (dsptoolbox.classes.signal\_class.Signal method)@\spxentry{set\_window()}\spxextra{dsptoolbox.classes.signal\_class.Signal method}}

\begin{fulllineitems}
\phantomsection\label{\detokenize{classes:dsptoolbox.classes.signal_class.Signal.set_window}}
\pysigstartsignatures
\pysiglinewithargsret{\sphinxbfcode{\sphinxupquote{set\_window}}}{\emph{\DUrole{n}{window}}}{}
\pysigstopsignatures
\sphinxAtStartPar
Sets the window used for the IR. It only works for
\sphinxtitleref{signal\_type in (‘ir’, ‘h1’, ‘h2’, ‘h3’, ‘rir’)}.
\begin{quote}\begin{description}
\sphinxlineitem{Parameters}\begin{description}
\sphinxlineitem{\sphinxstylestrong{window}}{[}\sphinxtitleref{np.ndarray}{]}
\sphinxAtStartPar
Window used for the IR.

\end{description}

\end{description}\end{quote}

\end{fulllineitems}

\index{show\_info() (dsptoolbox.classes.signal\_class.Signal method)@\spxentry{show\_info()}\spxextra{dsptoolbox.classes.signal\_class.Signal method}}

\begin{fulllineitems}
\phantomsection\label{\detokenize{classes:dsptoolbox.classes.signal_class.Signal.show_info}}
\pysigstartsignatures
\pysiglinewithargsret{\sphinxbfcode{\sphinxupquote{show\_info}}}{}{}
\pysigstopsignatures
\sphinxAtStartPar
Prints all the signal information to the console.

\end{fulllineitems}

\index{signal\_id (dsptoolbox.classes.signal\_class.Signal property)@\spxentry{signal\_id}\spxextra{dsptoolbox.classes.signal\_class.Signal property}}

\begin{fulllineitems}
\phantomsection\label{\detokenize{classes:dsptoolbox.classes.signal_class.Signal.signal_id}}
\pysigstartsignatures
\pysigline{\sphinxbfcode{\sphinxupquote{property\DUrole{w}{  }}}\sphinxbfcode{\sphinxupquote{signal\_id}}}
\pysigstopsignatures
\end{fulllineitems}

\index{signal\_type (dsptoolbox.classes.signal\_class.Signal property)@\spxentry{signal\_type}\spxextra{dsptoolbox.classes.signal\_class.Signal property}}

\begin{fulllineitems}
\phantomsection\label{\detokenize{classes:dsptoolbox.classes.signal_class.Signal.signal_type}}
\pysigstartsignatures
\pysigline{\sphinxbfcode{\sphinxupquote{property\DUrole{w}{  }}}\sphinxbfcode{\sphinxupquote{signal\_type}}}
\pysigstopsignatures
\end{fulllineitems}

\index{swap\_channels() (dsptoolbox.classes.signal\_class.Signal method)@\spxentry{swap\_channels()}\spxextra{dsptoolbox.classes.signal\_class.Signal method}}

\begin{fulllineitems}
\phantomsection\label{\detokenize{classes:dsptoolbox.classes.signal_class.Signal.swap_channels}}
\pysigstartsignatures
\pysiglinewithargsret{\sphinxbfcode{\sphinxupquote{swap\_channels}}}{\emph{\DUrole{n}{new\_order}}}{}
\pysigstopsignatures
\sphinxAtStartPar
Rearranges the channels in the new given order.
\begin{quote}\begin{description}
\sphinxlineitem{Parameters}\begin{description}
\sphinxlineitem{\sphinxstylestrong{new\_order}}{[}array\sphinxhyphen{}like{]}
\sphinxAtStartPar
New rearrangement of channels.

\end{description}

\end{description}\end{quote}

\end{fulllineitems}

\index{time\_data (dsptoolbox.classes.signal\_class.Signal property)@\spxentry{time\_data}\spxextra{dsptoolbox.classes.signal\_class.Signal property}}

\begin{fulllineitems}
\phantomsection\label{\detokenize{classes:dsptoolbox.classes.signal_class.Signal.time_data}}
\pysigstartsignatures
\pysigline{\sphinxbfcode{\sphinxupquote{property\DUrole{w}{  }}}\sphinxbfcode{\sphinxupquote{time\_data}}}
\pysigstopsignatures
\end{fulllineitems}


\end{fulllineitems}



\section{MultiBandSignal}
\label{\detokenize{classes:module-dsptoolbox.classes.multibandsignal}}\label{\detokenize{classes:multibandsignal}}\index{module@\spxentry{module}!dsptoolbox.classes.multibandsignal@\spxentry{dsptoolbox.classes.multibandsignal}}\index{dsptoolbox.classes.multibandsignal@\spxentry{dsptoolbox.classes.multibandsignal}!module@\spxentry{module}}\index{MultiBandSignal (class in dsptoolbox.classes.multibandsignal)@\spxentry{MultiBandSignal}\spxextra{class in dsptoolbox.classes.multibandsignal}}

\begin{fulllineitems}
\phantomsection\label{\detokenize{classes:dsptoolbox.classes.multibandsignal.MultiBandSignal}}
\pysigstartsignatures
\pysiglinewithargsret{\sphinxbfcode{\sphinxupquote{class\DUrole{w}{  }}}\sphinxcode{\sphinxupquote{dsptoolbox.classes.multibandsignal.}}\sphinxbfcode{\sphinxupquote{MultiBandSignal}}}{\emph{\DUrole{n}{bands}\DUrole{o}{=}\DUrole{default_value}{None}}, \emph{\DUrole{n}{same\_sampling\_rate}\DUrole{p}{:}\DUrole{w}{  }\DUrole{n}{\sphinxhref{https://docs.python.org/3/library/functions.html\#bool}{bool}}\DUrole{w}{  }\DUrole{o}{=}\DUrole{w}{  }\DUrole{default_value}{True}}, \emph{\DUrole{n}{info}\DUrole{p}{:}\DUrole{w}{  }\DUrole{n}{\sphinxhref{https://docs.python.org/3/library/stdtypes.html\#dict}{dict}}\DUrole{w}{  }\DUrole{o}{=}\DUrole{w}{  }\DUrole{default_value}{\{\}}}}{}
\pysigstopsignatures
\sphinxAtStartPar
Bases: \sphinxhref{https://docs.python.org/3/library/functions.html\#object}{\sphinxcode{\sphinxupquote{object}}}

\sphinxAtStartPar
The \sphinxtitleref{MultiBandSignal} class contains multiple Signal objects which are
to be interpreted as frequency bands or the same signal. Since every
signal has also multiple channels, the object resembles somewhat a
3D\sphinxhyphen{}Matrix representation of a signal.

\sphinxAtStartPar
The \sphinxtitleref{MultiBandSignal} can contain multirate system if the attribute
\sphinxtitleref{same\_sampling\_rate} is set to \sphinxtitleref{False}. A dictionary also can carry
all kinds of metadata that might characterize the signals.
\begin{quote}\begin{description}
\sphinxlineitem{Attributes}\begin{description}
\sphinxlineitem{\sphinxstylestrong{bands}}
\sphinxlineitem{\sphinxstylestrong{same\_sampling\_rate}}
\sphinxlineitem{\sphinxstylestrong{sampling\_rate\_hz}}
\end{description}

\end{description}\end{quote}
\subsubsection*{Methods}


\begin{savenotes}\sphinxattablestart
\sphinxthistablewithglobalstyle
\sphinxthistablewithnovlinesstyle
\centering
\begin{tabulary}{\linewidth}[t]{\X{1}{2}\X{1}{2}}
\sphinxtoprule
\sphinxtableatstartofbodyhook
\sphinxAtStartPar
{\hyperref[\detokenize{classes:dsptoolbox.classes.multibandsignal.MultiBandSignal.add_band}]{\sphinxcrossref{\sphinxcode{\sphinxupquote{add\_band}}}}}(sig{[}, index{]})
&
\sphinxAtStartPar
Adds a new band to the \sphinxtitleref{MultiBandSignal}.
\\
\sphinxhline
\sphinxAtStartPar
{\hyperref[\detokenize{classes:dsptoolbox.classes.multibandsignal.MultiBandSignal.collapse}]{\sphinxcrossref{\sphinxcode{\sphinxupquote{collapse}}}}}()
&
\sphinxAtStartPar
Collapses MultiBandSignal by summing all of its bands and returning one Signal.
\\
\sphinxhline
\sphinxAtStartPar
{\hyperref[\detokenize{classes:dsptoolbox.classes.multibandsignal.MultiBandSignal.copy}]{\sphinxcrossref{\sphinxcode{\sphinxupquote{copy}}}}}()
&
\sphinxAtStartPar
Returns a copy of the object.
\\
\sphinxhline
\sphinxAtStartPar
{\hyperref[\detokenize{classes:dsptoolbox.classes.multibandsignal.MultiBandSignal.get_all_bands}]{\sphinxcrossref{\sphinxcode{\sphinxupquote{get\_all\_bands}}}}}({[}channel{]})
&
\sphinxAtStartPar
Returns a signal with all bands as channels.
\\
\sphinxhline
\sphinxAtStartPar
{\hyperref[\detokenize{classes:dsptoolbox.classes.multibandsignal.MultiBandSignal.remove_band}]{\sphinxcrossref{\sphinxcode{\sphinxupquote{remove\_band}}}}}({[}index, return\_band{]})
&
\sphinxAtStartPar
Removes a band from the \sphinxtitleref{MultiBandSignal}.
\\
\sphinxhline
\sphinxAtStartPar
{\hyperref[\detokenize{classes:dsptoolbox.classes.multibandsignal.MultiBandSignal.save_signal}]{\sphinxcrossref{\sphinxcode{\sphinxupquote{save\_signal}}}}}({[}path{]})
&
\sphinxAtStartPar
Saves the \sphinxtitleref{MultiBandSignal} object as a pickle.
\\
\sphinxhline
\sphinxAtStartPar
{\hyperref[\detokenize{classes:dsptoolbox.classes.multibandsignal.MultiBandSignal.show_info}]{\sphinxcrossref{\sphinxcode{\sphinxupquote{show\_info}}}}}({[}show\_band\_info{]})
&
\sphinxAtStartPar
Show information about the \sphinxtitleref{MultiBandSignal}.
\\
\sphinxbottomrule
\end{tabulary}
\sphinxtableafterendhook\par
\sphinxattableend\end{savenotes}
\index{\_\_init\_\_() (dsptoolbox.classes.multibandsignal.MultiBandSignal method)@\spxentry{\_\_init\_\_()}\spxextra{dsptoolbox.classes.multibandsignal.MultiBandSignal method}}

\begin{fulllineitems}
\phantomsection\label{\detokenize{classes:dsptoolbox.classes.multibandsignal.MultiBandSignal.__init__}}
\pysigstartsignatures
\pysiglinewithargsret{\sphinxbfcode{\sphinxupquote{\_\_init\_\_}}}{\emph{\DUrole{n}{bands}\DUrole{o}{=}\DUrole{default_value}{None}}, \emph{\DUrole{n}{same\_sampling\_rate}\DUrole{p}{:}\DUrole{w}{  }\DUrole{n}{\sphinxhref{https://docs.python.org/3/library/functions.html\#bool}{bool}}\DUrole{w}{  }\DUrole{o}{=}\DUrole{w}{  }\DUrole{default_value}{True}}, \emph{\DUrole{n}{info}\DUrole{p}{:}\DUrole{w}{  }\DUrole{n}{\sphinxhref{https://docs.python.org/3/library/stdtypes.html\#dict}{dict}}\DUrole{w}{  }\DUrole{o}{=}\DUrole{w}{  }\DUrole{default_value}{\{\}}}}{}
\pysigstopsignatures
\sphinxAtStartPar
\sphinxtitleref{MultiBandSignal} contains a composite band list where each index
is a Signal object with the same number of channels. For multirate
systems, the parameter \sphinxtitleref{same\_sampling\_rate} has to be set to \sphinxtitleref{False}.
\begin{quote}\begin{description}
\sphinxlineitem{Parameters}\begin{description}
\sphinxlineitem{\sphinxstylestrong{bands}}{[}list or tuple, optional{]}
\sphinxAtStartPar
List or tuple containing different Signal objects. All of them
should be associated to the same Signal. This means that the
channel numbers have to match. Set to \sphinxtitleref{None} for initializing the
object. Default: \sphinxtitleref{None}.

\sphinxlineitem{\sphinxstylestrong{same\_sampling\_rate}}{[}bool, optional{]}
\sphinxAtStartPar
When \sphinxtitleref{True}, every Signal should have the same sampling rate.
Set to \sphinxtitleref{False} for a multirate system. Default: \sphinxtitleref{True}.

\sphinxlineitem{\sphinxstylestrong{info}}{[}dict, optional{]}
\sphinxAtStartPar
A dictionary with generic information about the \sphinxtitleref{MultiBandSignal}
can be passed. Default: \sphinxtitleref{None}.

\end{description}

\end{description}\end{quote}

\end{fulllineitems}

\index{add\_band() (dsptoolbox.classes.multibandsignal.MultiBandSignal method)@\spxentry{add\_band()}\spxextra{dsptoolbox.classes.multibandsignal.MultiBandSignal method}}

\begin{fulllineitems}
\phantomsection\label{\detokenize{classes:dsptoolbox.classes.multibandsignal.MultiBandSignal.add_band}}
\pysigstartsignatures
\pysiglinewithargsret{\sphinxbfcode{\sphinxupquote{add\_band}}}{\emph{\DUrole{n}{sig}\DUrole{p}{:}\DUrole{w}{  }\DUrole{n}{{\hyperref[\detokenize{classes:dsptoolbox.classes.signal_class.Signal}]{\sphinxcrossref{Signal}}}}}, \emph{\DUrole{n}{index}\DUrole{p}{:}\DUrole{w}{  }\DUrole{n}{\sphinxhref{https://docs.python.org/3/library/functions.html\#int}{int}}\DUrole{w}{  }\DUrole{o}{=}\DUrole{w}{  }\DUrole{default_value}{\sphinxhyphen{}1}}}{}
\pysigstopsignatures
\sphinxAtStartPar
Adds a new band to the \sphinxtitleref{MultiBandSignal}.
\begin{quote}\begin{description}
\sphinxlineitem{Parameters}\begin{description}
\sphinxlineitem{\sphinxstylestrong{sig}}{[}\sphinxtitleref{Signal}{]}
\sphinxAtStartPar
Signal to be added.

\sphinxlineitem{\sphinxstylestrong{index}}{[}int, optional{]}
\sphinxAtStartPar
Index at which to insert the new Signal. Default: \sphinxhyphen{}1.

\end{description}

\end{description}\end{quote}

\end{fulllineitems}

\index{bands (dsptoolbox.classes.multibandsignal.MultiBandSignal property)@\spxentry{bands}\spxextra{dsptoolbox.classes.multibandsignal.MultiBandSignal property}}

\begin{fulllineitems}
\phantomsection\label{\detokenize{classes:dsptoolbox.classes.multibandsignal.MultiBandSignal.bands}}
\pysigstartsignatures
\pysigline{\sphinxbfcode{\sphinxupquote{property\DUrole{w}{  }}}\sphinxbfcode{\sphinxupquote{bands}}}
\pysigstopsignatures
\end{fulllineitems}

\index{collapse() (dsptoolbox.classes.multibandsignal.MultiBandSignal method)@\spxentry{collapse()}\spxextra{dsptoolbox.classes.multibandsignal.MultiBandSignal method}}

\begin{fulllineitems}
\phantomsection\label{\detokenize{classes:dsptoolbox.classes.multibandsignal.MultiBandSignal.collapse}}
\pysigstartsignatures
\pysiglinewithargsret{\sphinxbfcode{\sphinxupquote{collapse}}}{}{}
\pysigstopsignatures
\sphinxAtStartPar
Collapses MultiBandSignal by summing all of its bands and returning
one Signal.
\begin{quote}\begin{description}
\sphinxlineitem{Returns}\begin{description}
\sphinxlineitem{\sphinxstylestrong{new\_sig}}{[}\sphinxtitleref{Signal}{]}
\sphinxAtStartPar
Collapsed Signal.

\end{description}

\end{description}\end{quote}

\end{fulllineitems}

\index{copy() (dsptoolbox.classes.multibandsignal.MultiBandSignal method)@\spxentry{copy()}\spxextra{dsptoolbox.classes.multibandsignal.MultiBandSignal method}}

\begin{fulllineitems}
\phantomsection\label{\detokenize{classes:dsptoolbox.classes.multibandsignal.MultiBandSignal.copy}}
\pysigstartsignatures
\pysiglinewithargsret{\sphinxbfcode{\sphinxupquote{copy}}}{}{}
\pysigstopsignatures
\sphinxAtStartPar
Returns a copy of the object.
\begin{quote}\begin{description}
\sphinxlineitem{Returns}\begin{description}
\sphinxlineitem{\sphinxstylestrong{new\_sig}}{[}\sphinxtitleref{MultiBandSignal}{]}
\sphinxAtStartPar
Copy of Signal.

\end{description}

\end{description}\end{quote}

\end{fulllineitems}

\index{get\_all\_bands() (dsptoolbox.classes.multibandsignal.MultiBandSignal method)@\spxentry{get\_all\_bands()}\spxextra{dsptoolbox.classes.multibandsignal.MultiBandSignal method}}

\begin{fulllineitems}
\phantomsection\label{\detokenize{classes:dsptoolbox.classes.multibandsignal.MultiBandSignal.get_all_bands}}
\pysigstartsignatures
\pysiglinewithargsret{\sphinxbfcode{\sphinxupquote{get\_all\_bands}}}{\emph{\DUrole{n}{channel}\DUrole{p}{:}\DUrole{w}{  }\DUrole{n}{\sphinxhref{https://docs.python.org/3/library/functions.html\#int}{int}}\DUrole{w}{  }\DUrole{o}{=}\DUrole{w}{  }\DUrole{default_value}{0}}}{}
\pysigstopsignatures
\sphinxAtStartPar
Returns a signal with all bands as channels. Done for an specified
channel.
\begin{quote}\begin{description}
\sphinxlineitem{Parameters}\begin{description}
\sphinxlineitem{\sphinxstylestrong{channel}}{[}int, optional{]}
\sphinxAtStartPar
Channel to choose from the band signals.

\end{description}

\sphinxlineitem{Returns}\begin{description}
\sphinxlineitem{\sphinxstylestrong{sig}}{[}\sphinxtitleref{Signal} or list of \sphinxtitleref{np.ndarray} and dict{]}
\sphinxAtStartPar
Multichannel signal with all the bands. If the MultiBandSignal
does not have the same sampling rate for all signals, a list with
the time data vectors and a dictionary containing their sampling
rates with the key ‘sampling\_rates’ are returned.

\end{description}

\end{description}\end{quote}

\end{fulllineitems}

\index{remove\_band() (dsptoolbox.classes.multibandsignal.MultiBandSignal method)@\spxentry{remove\_band()}\spxextra{dsptoolbox.classes.multibandsignal.MultiBandSignal method}}

\begin{fulllineitems}
\phantomsection\label{\detokenize{classes:dsptoolbox.classes.multibandsignal.MultiBandSignal.remove_band}}
\pysigstartsignatures
\pysiglinewithargsret{\sphinxbfcode{\sphinxupquote{remove\_band}}}{\emph{\DUrole{n}{index}\DUrole{p}{:}\DUrole{w}{  }\DUrole{n}{\sphinxhref{https://docs.python.org/3/library/functions.html\#int}{int}}\DUrole{w}{  }\DUrole{o}{=}\DUrole{w}{  }\DUrole{default_value}{\sphinxhyphen{}1}}, \emph{\DUrole{n}{return\_band}\DUrole{p}{:}\DUrole{w}{  }\DUrole{n}{\sphinxhref{https://docs.python.org/3/library/functions.html\#bool}{bool}}\DUrole{w}{  }\DUrole{o}{=}\DUrole{w}{  }\DUrole{default_value}{False}}}{}
\pysigstopsignatures
\sphinxAtStartPar
Removes a band from the \sphinxtitleref{MultiBandSignal}.
\begin{quote}\begin{description}
\sphinxlineitem{Parameters}\begin{description}
\sphinxlineitem{\sphinxstylestrong{index}}{[}int, optional{]}
\sphinxAtStartPar
This is the index from the bands list at which the band
will be erased. When \sphinxhyphen{}1, last band is erased.
Default: \sphinxhyphen{}1.

\sphinxlineitem{\sphinxstylestrong{return\_band}}{[}bool, optional{]}
\sphinxAtStartPar
When \sphinxtitleref{True}, the erased band is returned. Default: \sphinxtitleref{False}.

\end{description}

\end{description}\end{quote}

\end{fulllineitems}

\index{same\_sampling\_rate (dsptoolbox.classes.multibandsignal.MultiBandSignal property)@\spxentry{same\_sampling\_rate}\spxextra{dsptoolbox.classes.multibandsignal.MultiBandSignal property}}

\begin{fulllineitems}
\phantomsection\label{\detokenize{classes:dsptoolbox.classes.multibandsignal.MultiBandSignal.same_sampling_rate}}
\pysigstartsignatures
\pysigline{\sphinxbfcode{\sphinxupquote{property\DUrole{w}{  }}}\sphinxbfcode{\sphinxupquote{same\_sampling\_rate}}}
\pysigstopsignatures
\end{fulllineitems}

\index{sampling\_rate\_hz (dsptoolbox.classes.multibandsignal.MultiBandSignal property)@\spxentry{sampling\_rate\_hz}\spxextra{dsptoolbox.classes.multibandsignal.MultiBandSignal property}}

\begin{fulllineitems}
\phantomsection\label{\detokenize{classes:dsptoolbox.classes.multibandsignal.MultiBandSignal.sampling_rate_hz}}
\pysigstartsignatures
\pysigline{\sphinxbfcode{\sphinxupquote{property\DUrole{w}{  }}}\sphinxbfcode{\sphinxupquote{sampling\_rate\_hz}}}
\pysigstopsignatures
\end{fulllineitems}

\index{save\_signal() (dsptoolbox.classes.multibandsignal.MultiBandSignal method)@\spxentry{save\_signal()}\spxextra{dsptoolbox.classes.multibandsignal.MultiBandSignal method}}

\begin{fulllineitems}
\phantomsection\label{\detokenize{classes:dsptoolbox.classes.multibandsignal.MultiBandSignal.save_signal}}
\pysigstartsignatures
\pysiglinewithargsret{\sphinxbfcode{\sphinxupquote{save\_signal}}}{\emph{\DUrole{n}{path}\DUrole{p}{:}\DUrole{w}{  }\DUrole{n}{\sphinxhref{https://docs.python.org/3/library/stdtypes.html\#str}{str}}\DUrole{w}{  }\DUrole{o}{=}\DUrole{w}{  }\DUrole{default_value}{\textquotesingle{}multibandsignal\textquotesingle{}}}}{}
\pysigstopsignatures
\sphinxAtStartPar
Saves the \sphinxtitleref{MultiBandSignal} object as a pickle.
\begin{quote}\begin{description}
\sphinxlineitem{Parameters}\begin{description}
\sphinxlineitem{\sphinxstylestrong{path}}{[}str, optional{]}
\sphinxAtStartPar
Path for the signal to be saved. Use only folder/folder/name
(without format). Default: \sphinxtitleref{‘multibandsignal’}
(local folder, object named multibandsignal).

\end{description}

\end{description}\end{quote}

\end{fulllineitems}

\index{show\_info() (dsptoolbox.classes.multibandsignal.MultiBandSignal method)@\spxentry{show\_info()}\spxextra{dsptoolbox.classes.multibandsignal.MultiBandSignal method}}

\begin{fulllineitems}
\phantomsection\label{\detokenize{classes:dsptoolbox.classes.multibandsignal.MultiBandSignal.show_info}}
\pysigstartsignatures
\pysiglinewithargsret{\sphinxbfcode{\sphinxupquote{show\_info}}}{\emph{\DUrole{n}{show\_band\_info}\DUrole{p}{:}\DUrole{w}{  }\DUrole{n}{\sphinxhref{https://docs.python.org/3/library/functions.html\#bool}{bool}}\DUrole{w}{  }\DUrole{o}{=}\DUrole{w}{  }\DUrole{default_value}{False}}}{}
\pysigstopsignatures
\sphinxAtStartPar
Show information about the \sphinxtitleref{MultiBandSignal}.
\begin{quote}\begin{description}
\sphinxlineitem{Parameters}\begin{description}
\sphinxlineitem{\sphinxstylestrong{show\_band\_info}}{[}bool, optional{]}
\sphinxAtStartPar
When \sphinxtitleref{True}, a longer message is printed with all available
information regarding each \sphinxtitleref{Signal} in the \sphinxtitleref{MultiBandSignal}.
Default: \sphinxtitleref{True}.

\end{description}

\end{description}\end{quote}

\end{fulllineitems}


\end{fulllineitems}



\section{Filter}
\label{\detokenize{classes:module-dsptoolbox.classes.filter_class}}\label{\detokenize{classes:filter}}\index{module@\spxentry{module}!dsptoolbox.classes.filter\_class@\spxentry{dsptoolbox.classes.filter\_class}}\index{dsptoolbox.classes.filter\_class@\spxentry{dsptoolbox.classes.filter\_class}!module@\spxentry{module}}
\sphinxAtStartPar
Contains Filter classes
\index{Filter (class in dsptoolbox.classes.filter\_class)@\spxentry{Filter}\spxextra{class in dsptoolbox.classes.filter\_class}}

\begin{fulllineitems}
\phantomsection\label{\detokenize{classes:dsptoolbox.classes.filter_class.Filter}}
\pysigstartsignatures
\pysiglinewithargsret{\sphinxbfcode{\sphinxupquote{class\DUrole{w}{  }}}\sphinxcode{\sphinxupquote{dsptoolbox.classes.filter\_class.}}\sphinxbfcode{\sphinxupquote{Filter}}}{\emph{\DUrole{n}{filter\_type}\DUrole{p}{:}\DUrole{w}{  }\DUrole{n}{\sphinxhref{https://docs.python.org/3/library/stdtypes.html\#str}{str}}\DUrole{w}{  }\DUrole{o}{=}\DUrole{w}{  }\DUrole{default_value}{\textquotesingle{}biquad\textquotesingle{}}}, \emph{\DUrole{n}{filter\_configuration}\DUrole{p}{:}\DUrole{w}{  }\DUrole{n}{\sphinxhref{https://docs.python.org/3/library/typing.html\#typing.Optional}{Optional}\DUrole{p}{{[}}\sphinxhref{https://docs.python.org/3/library/stdtypes.html\#dict}{dict}\DUrole{p}{{]}}}\DUrole{w}{  }\DUrole{o}{=}\DUrole{w}{  }\DUrole{default_value}{None}}, \emph{\DUrole{n}{sampling\_rate\_hz}\DUrole{p}{:}\DUrole{w}{  }\DUrole{n}{\sphinxhref{https://docs.python.org/3/library/functions.html\#int}{int}}\DUrole{w}{  }\DUrole{o}{=}\DUrole{w}{  }\DUrole{default_value}{48000}}}{}
\pysigstopsignatures
\sphinxAtStartPar
Bases: \sphinxhref{https://docs.python.org/3/library/functions.html\#object}{\sphinxcode{\sphinxupquote{object}}}

\sphinxAtStartPar
Class for creating and storing linear digital filters with all their
metadata.
\begin{quote}\begin{description}
\sphinxlineitem{Attributes}\begin{description}
\sphinxlineitem{\sphinxstylestrong{sampling\_rate\_hz}}
\end{description}

\end{description}\end{quote}
\subsubsection*{Methods}


\begin{savenotes}\sphinxattablestart
\sphinxthistablewithglobalstyle
\sphinxthistablewithnovlinesstyle
\centering
\begin{tabulary}{\linewidth}[t]{\X{1}{2}\X{1}{2}}
\sphinxtoprule
\sphinxtableatstartofbodyhook
\sphinxAtStartPar
{\hyperref[\detokenize{classes:dsptoolbox.classes.filter_class.Filter.copy}]{\sphinxcrossref{\sphinxcode{\sphinxupquote{copy}}}}}()
&
\sphinxAtStartPar
Returns a copy of the object.
\\
\sphinxhline
\sphinxAtStartPar
{\hyperref[\detokenize{classes:dsptoolbox.classes.filter_class.Filter.filter_signal}]{\sphinxcrossref{\sphinxcode{\sphinxupquote{filter\_signal}}}}}(signal{[}, channel, ...{]})
&
\sphinxAtStartPar
Takes in a \sphinxtitleref{Signal} object and filters selected channels.
\\
\sphinxhline
\sphinxAtStartPar
{\hyperref[\detokenize{classes:dsptoolbox.classes.filter_class.Filter.get_coefficients}]{\sphinxcrossref{\sphinxcode{\sphinxupquote{get\_coefficients}}}}}({[}mode{]})
&
\sphinxAtStartPar
Returns the filter coefficients.
\\
\sphinxhline
\sphinxAtStartPar
{\hyperref[\detokenize{classes:dsptoolbox.classes.filter_class.Filter.get_filter_metadata}]{\sphinxcrossref{\sphinxcode{\sphinxupquote{get\_filter\_metadata}}}}}()
&
\sphinxAtStartPar
Returns filter metadata.
\\
\sphinxhline
\sphinxAtStartPar
{\hyperref[\detokenize{classes:dsptoolbox.classes.filter_class.Filter.get_ir}]{\sphinxcrossref{\sphinxcode{\sphinxupquote{get\_ir}}}}}({[}length\_samples{]})
&
\sphinxAtStartPar
Gets an impulse response of the filter with given length.
\\
\sphinxhline
\sphinxAtStartPar
{\hyperref[\detokenize{classes:dsptoolbox.classes.filter_class.Filter.initialize_zi}]{\sphinxcrossref{\sphinxcode{\sphinxupquote{initialize\_zi}}}}}({[}number\_of\_channels{]})
&
\sphinxAtStartPar
Initializes zi for steady\sphinxhyphen{}state filtering.
\\
\sphinxhline
\sphinxAtStartPar
{\hyperref[\detokenize{classes:dsptoolbox.classes.filter_class.Filter.plot_group_delay}]{\sphinxcrossref{\sphinxcode{\sphinxupquote{plot\_group\_delay}}}}}({[}length\_samples, range\_hz, ...{]})
&
\sphinxAtStartPar
Plots group delay of the filter.
\\
\sphinxhline
\sphinxAtStartPar
{\hyperref[\detokenize{classes:dsptoolbox.classes.filter_class.Filter.plot_magnitude}]{\sphinxcrossref{\sphinxcode{\sphinxupquote{plot\_magnitude}}}}}({[}length\_samples, range\_hz, ...{]})
&
\sphinxAtStartPar
Plots magnitude spectrum.
\\
\sphinxhline
\sphinxAtStartPar
{\hyperref[\detokenize{classes:dsptoolbox.classes.filter_class.Filter.plot_phase}]{\sphinxcrossref{\sphinxcode{\sphinxupquote{plot\_phase}}}}}({[}length\_samples, range\_hz, ...{]})
&
\sphinxAtStartPar
Plots phase spectrum.
\\
\sphinxhline
\sphinxAtStartPar
{\hyperref[\detokenize{classes:dsptoolbox.classes.filter_class.Filter.plot_zp}]{\sphinxcrossref{\sphinxcode{\sphinxupquote{plot\_zp}}}}}({[}show\_info\_box, returns{]})
&
\sphinxAtStartPar
Plots zeros and poles with the unit circle.
\\
\sphinxhline
\sphinxAtStartPar
{\hyperref[\detokenize{classes:dsptoolbox.classes.filter_class.Filter.save_filter}]{\sphinxcrossref{\sphinxcode{\sphinxupquote{save\_filter}}}}}({[}path{]})
&
\sphinxAtStartPar
Saves the Filter object as a pickle.
\\
\sphinxhline
\sphinxAtStartPar
{\hyperref[\detokenize{classes:dsptoolbox.classes.filter_class.Filter.set_filter_parameters}]{\sphinxcrossref{\sphinxcode{\sphinxupquote{set\_filter\_parameters}}}}}(filter\_type, ...)
&
\sphinxAtStartPar

\\
\sphinxhline
\sphinxAtStartPar
{\hyperref[\detokenize{classes:dsptoolbox.classes.filter_class.Filter.show_info}]{\sphinxcrossref{\sphinxcode{\sphinxupquote{show\_info}}}}}()
&
\sphinxAtStartPar
Prints all the filter parameters to the console.
\\
\sphinxbottomrule
\end{tabulary}
\sphinxtableafterendhook\par
\sphinxattableend\end{savenotes}
\index{\_\_init\_\_() (dsptoolbox.classes.filter\_class.Filter method)@\spxentry{\_\_init\_\_()}\spxextra{dsptoolbox.classes.filter\_class.Filter method}}

\begin{fulllineitems}
\phantomsection\label{\detokenize{classes:dsptoolbox.classes.filter_class.Filter.__init__}}
\pysigstartsignatures
\pysiglinewithargsret{\sphinxbfcode{\sphinxupquote{\_\_init\_\_}}}{\emph{\DUrole{n}{filter\_type}\DUrole{p}{:}\DUrole{w}{  }\DUrole{n}{\sphinxhref{https://docs.python.org/3/library/stdtypes.html\#str}{str}}\DUrole{w}{  }\DUrole{o}{=}\DUrole{w}{  }\DUrole{default_value}{\textquotesingle{}biquad\textquotesingle{}}}, \emph{\DUrole{n}{filter\_configuration}\DUrole{p}{:}\DUrole{w}{  }\DUrole{n}{\sphinxhref{https://docs.python.org/3/library/typing.html\#typing.Optional}{Optional}\DUrole{p}{{[}}\sphinxhref{https://docs.python.org/3/library/stdtypes.html\#dict}{dict}\DUrole{p}{{]}}}\DUrole{w}{  }\DUrole{o}{=}\DUrole{w}{  }\DUrole{default_value}{None}}, \emph{\DUrole{n}{sampling\_rate\_hz}\DUrole{p}{:}\DUrole{w}{  }\DUrole{n}{\sphinxhref{https://docs.python.org/3/library/functions.html\#int}{int}}\DUrole{w}{  }\DUrole{o}{=}\DUrole{w}{  }\DUrole{default_value}{48000}}}{}
\pysigstopsignatures
\sphinxAtStartPar
The Filter class contains all parameters and metadata needed for
using a digital filter.
\begin{quote}\begin{description}
\sphinxlineitem{Parameters}\begin{description}
\sphinxlineitem{\sphinxstylestrong{filter\_type}}{[}str, optional{]}
\sphinxAtStartPar
String defining the filter type. Options are \sphinxtitleref{iir}, \sphinxtitleref{fir},
\sphinxtitleref{biquad} or \sphinxtitleref{other}. Default: creates a dummy biquad bell filter
with no gain.

\sphinxlineitem{\sphinxstylestrong{filter\_configuration}}{[}dict, optional{]}
\sphinxAtStartPar
Dictionary containing configuration for the filter.
Default: some dummy parameters.

\sphinxlineitem{\sphinxstylestrong{sampling\_rate\_hz}}{[}int, optional{]}
\sphinxAtStartPar
Sampling rate in Hz for the digital filter. Default: 48000.

\end{description}

\end{description}\end{quote}
\subsubsection*{Notes}
\begin{description}
\sphinxlineitem{For \sphinxtitleref{iir}:}
\sphinxAtStartPar
order, freqs, type\_of\_pass, filter\_design\_method,
filter\_id (optional).
freqs (float, array\sphinxhyphen{}like): array with len 2 when ‘bandpass’
\begin{quote}

\sphinxAtStartPar
or ‘bandstop’.
\end{quote}

\sphinxAtStartPar
type\_of\_pass (str): ‘bandpass’, ‘lowpass’, ‘highpass’, ‘bandstop’.
filter\_design\_method (str): ‘butter’, ‘bessel’, ‘ellip’, ‘cheby1’,
\begin{quote}

\sphinxAtStartPar
‘cheby2’.
\end{quote}

\sphinxlineitem{For \sphinxtitleref{fir}:}
\sphinxAtStartPar
order, freqs, type\_of\_pass, filter\_design\_method (optional),
width (optional, necessary for ‘kaiser’), filter\_id (optional).
filter\_design\_method (str): Window to be used. Default: ‘hamming’.
\begin{quote}

\sphinxAtStartPar
Supported types are: ‘boxcar’, ‘triang’, ‘blackman’, ‘hamming’,
‘hann’, ‘bartlett’, ‘flattop’, ‘parzen’, ‘bohman’,
‘blackmanharris’, ‘nuttall’, ‘barthann’, ‘cosine’,
‘exponential’, ‘tukey’, ‘taylor’.
\end{quote}
\begin{description}
\sphinxlineitem{width (float): estimated width of transition region in Hz for}
\sphinxAtStartPar
kaiser window. Default: \sphinxtitleref{None}.

\end{description}

\sphinxAtStartPar
type\_of\_pass (str): ‘bandpass’, ‘lowpass’, ‘highpass’, ‘bandstop’.

\sphinxlineitem{For \sphinxtitleref{biquad}:}
\sphinxAtStartPar
eq\_type, freqs, gain, q, filter\_id (optional).
gain (float): in dB.
eq\_type (int or str): 0 = Bell/Peaking, 1 = Lowpass, 2 = Highpass,
\begin{quote}

\sphinxAtStartPar
3 = Bandpass skirt, 4 = Bandpass peak, 5 = Notch, 6 = Allpass,
7 = Lowshelf, 8 = Highshelf.
\end{quote}

\sphinxlineitem{For \sphinxtitleref{other} or \sphinxtitleref{general}:}
\sphinxAtStartPar
ba or sos or zpk, filter\_id (optional).

\end{description}
\subsubsection*{Methods}


\begin{savenotes}\sphinxattablestart
\sphinxthistablewithglobalstyle
\centering
\begin{tabulary}{\linewidth}[t]{|T|T|}
\sphinxtoprule
\sphinxtableatstartofbodyhook
\sphinxAtStartPar
\sphinxstylestrong{General}
&
\sphinxAtStartPar
set\_filter\_parameters, get\_filter\_metadata, get\_ir.
\\
\sphinxhline
\sphinxAtStartPar
\sphinxstylestrong{Plots or prints}
&
\sphinxAtStartPar
show\_filter\_parameters, plot\_magnitude, plot\_group\_delay, plot\_phase, plot\_zp.
\\
\sphinxhline
\sphinxAtStartPar
\sphinxstylestrong{Filtering}
&
\sphinxAtStartPar
filter\_signal.
\\
\sphinxbottomrule
\end{tabulary}
\sphinxtableafterendhook\par
\sphinxattableend\end{savenotes}

\end{fulllineitems}

\index{copy() (dsptoolbox.classes.filter\_class.Filter method)@\spxentry{copy()}\spxextra{dsptoolbox.classes.filter\_class.Filter method}}

\begin{fulllineitems}
\phantomsection\label{\detokenize{classes:dsptoolbox.classes.filter_class.Filter.copy}}
\pysigstartsignatures
\pysiglinewithargsret{\sphinxbfcode{\sphinxupquote{copy}}}{}{}
\pysigstopsignatures
\sphinxAtStartPar
Returns a copy of the object.
\begin{quote}\begin{description}
\sphinxlineitem{Returns}\begin{description}
\sphinxlineitem{\sphinxstylestrong{new\_sig}}{[}\sphinxtitleref{Filter}{]}
\sphinxAtStartPar
Copy of filter.

\end{description}

\end{description}\end{quote}

\end{fulllineitems}

\index{filter\_signal() (dsptoolbox.classes.filter\_class.Filter method)@\spxentry{filter\_signal()}\spxextra{dsptoolbox.classes.filter\_class.Filter method}}

\begin{fulllineitems}
\phantomsection\label{\detokenize{classes:dsptoolbox.classes.filter_class.Filter.filter_signal}}
\pysigstartsignatures
\pysiglinewithargsret{\sphinxbfcode{\sphinxupquote{filter\_signal}}}{\emph{\DUrole{n}{signal}\DUrole{p}{:}\DUrole{w}{  }\DUrole{n}{{\hyperref[\detokenize{classes:dsptoolbox.classes.signal_class.Signal}]{\sphinxcrossref{Signal}}}}}, \emph{\DUrole{n}{channel}\DUrole{o}{=}\DUrole{default_value}{None}}, \emph{\DUrole{n}{activate\_zi}\DUrole{p}{:}\DUrole{w}{  }\DUrole{n}{\sphinxhref{https://docs.python.org/3/library/functions.html\#bool}{bool}}\DUrole{w}{  }\DUrole{o}{=}\DUrole{w}{  }\DUrole{default_value}{False}}, \emph{\DUrole{n}{zero\_phase}\DUrole{p}{:}\DUrole{w}{  }\DUrole{n}{\sphinxhref{https://docs.python.org/3/library/functions.html\#bool}{bool}}\DUrole{w}{  }\DUrole{o}{=}\DUrole{w}{  }\DUrole{default_value}{False}}}{}
\pysigstopsignatures
\sphinxAtStartPar
Takes in a \sphinxtitleref{Signal} object and filters selected channels. Exports a
new \sphinxtitleref{Signal} object.
\begin{quote}\begin{description}
\sphinxlineitem{Parameters}\begin{description}
\sphinxlineitem{\sphinxstylestrong{signal}}{[}\sphinxtitleref{Signal}{]}
\sphinxAtStartPar
Signal to be filtered.

\sphinxlineitem{\sphinxstylestrong{channel}}{[}int or array\sphinxhyphen{}like, optional{]}
\sphinxAtStartPar
Channel or array of channels to be filtered. When \sphinxtitleref{None}, all
channels are filtered. Default: \sphinxtitleref{None}.

\sphinxlineitem{\sphinxstylestrong{activate\_zi}}{[}int, optional{]}
\sphinxAtStartPar
Gives the zi to update the filter values. Default: \sphinxtitleref{False}.

\sphinxlineitem{\sphinxstylestrong{zero\_phase}}{[}bool, optional{]}
\sphinxAtStartPar
Uses zero\sphinxhyphen{}phase filtering on signal. Be aware that the filter
is applied twice in this case. Default: \sphinxtitleref{False}.

\end{description}

\sphinxlineitem{Returns}\begin{description}
\sphinxlineitem{\sphinxstylestrong{new\_signal}}{[}\sphinxtitleref{Signal}{]}
\sphinxAtStartPar
New Signal object.

\end{description}

\end{description}\end{quote}

\end{fulllineitems}

\index{get\_coefficients() (dsptoolbox.classes.filter\_class.Filter method)@\spxentry{get\_coefficients()}\spxextra{dsptoolbox.classes.filter\_class.Filter method}}

\begin{fulllineitems}
\phantomsection\label{\detokenize{classes:dsptoolbox.classes.filter_class.Filter.get_coefficients}}
\pysigstartsignatures
\pysiglinewithargsret{\sphinxbfcode{\sphinxupquote{get\_coefficients}}}{\emph{\DUrole{n}{mode}\DUrole{o}{=}\DUrole{default_value}{\textquotesingle{}sos\textquotesingle{}}}}{}
\pysigstopsignatures
\sphinxAtStartPar
Returns the filter coefficients.
\begin{quote}\begin{description}
\sphinxlineitem{Parameters}\begin{description}
\sphinxlineitem{\sphinxstylestrong{mode}}{[}str, optional{]}
\sphinxAtStartPar
Type of filter coefficients to be returned. Choose from \sphinxtitleref{‘sos’},
\sphinxtitleref{‘ba’} or \sphinxtitleref{‘zpk’}. Default: \sphinxtitleref{‘sos’}.

\end{description}

\sphinxlineitem{Returns}\begin{description}
\sphinxlineitem{\sphinxstylestrong{coefficients}}{[}array\sphinxhyphen{}like{]}
\sphinxAtStartPar
Array with filter parameters.

\end{description}

\end{description}\end{quote}

\end{fulllineitems}

\index{get\_filter\_metadata() (dsptoolbox.classes.filter\_class.Filter method)@\spxentry{get\_filter\_metadata()}\spxextra{dsptoolbox.classes.filter\_class.Filter method}}

\begin{fulllineitems}
\phantomsection\label{\detokenize{classes:dsptoolbox.classes.filter_class.Filter.get_filter_metadata}}
\pysigstartsignatures
\pysiglinewithargsret{\sphinxbfcode{\sphinxupquote{get\_filter\_metadata}}}{}{}
\pysigstopsignatures
\sphinxAtStartPar
Returns filter metadata.
\begin{quote}\begin{description}
\sphinxlineitem{Returns}\begin{description}
\sphinxlineitem{\sphinxstylestrong{info}}{[}dict{]}
\sphinxAtStartPar
Dictionary containing all filter metadata.

\end{description}

\end{description}\end{quote}

\end{fulllineitems}

\index{get\_ir() (dsptoolbox.classes.filter\_class.Filter method)@\spxentry{get\_ir()}\spxextra{dsptoolbox.classes.filter\_class.Filter method}}

\begin{fulllineitems}
\phantomsection\label{\detokenize{classes:dsptoolbox.classes.filter_class.Filter.get_ir}}
\pysigstartsignatures
\pysiglinewithargsret{\sphinxbfcode{\sphinxupquote{get\_ir}}}{\emph{\DUrole{n}{length\_samples}\DUrole{p}{:}\DUrole{w}{  }\DUrole{n}{\sphinxhref{https://docs.python.org/3/library/functions.html\#int}{int}}\DUrole{w}{  }\DUrole{o}{=}\DUrole{w}{  }\DUrole{default_value}{512}}}{}
\pysigstopsignatures
\sphinxAtStartPar
Gets an impulse response of the filter with given length.
\begin{quote}\begin{description}
\sphinxlineitem{Parameters}\begin{description}
\sphinxlineitem{\sphinxstylestrong{length\_samples}}{[}int, optional{]}
\sphinxAtStartPar
Length for the impulse response in samples. Default: 512.

\end{description}

\sphinxlineitem{Returns}\begin{description}
\sphinxlineitem{\sphinxstylestrong{ir\_filt}}{[}\sphinxtitleref{Signal}{]}
\sphinxAtStartPar
Impulse response of the filter.

\end{description}

\end{description}\end{quote}

\end{fulllineitems}

\index{initialize\_zi() (dsptoolbox.classes.filter\_class.Filter method)@\spxentry{initialize\_zi()}\spxextra{dsptoolbox.classes.filter\_class.Filter method}}

\begin{fulllineitems}
\phantomsection\label{\detokenize{classes:dsptoolbox.classes.filter_class.Filter.initialize_zi}}
\pysigstartsignatures
\pysiglinewithargsret{\sphinxbfcode{\sphinxupquote{initialize\_zi}}}{\emph{\DUrole{n}{number\_of\_channels}\DUrole{p}{:}\DUrole{w}{  }\DUrole{n}{\sphinxhref{https://docs.python.org/3/library/functions.html\#int}{int}}\DUrole{w}{  }\DUrole{o}{=}\DUrole{w}{  }\DUrole{default_value}{1}}}{}
\pysigstopsignatures
\sphinxAtStartPar
Initializes zi for steady\sphinxhyphen{}state filtering. The number of parallel
zi’s can be defined externally.
\begin{quote}\begin{description}
\sphinxlineitem{Parameters}\begin{description}
\sphinxlineitem{\sphinxstylestrong{number\_of\_channels}}{[}int, optional{]}
\sphinxAtStartPar
Number of channels is needed for the number of filter’s zi’s.
Default: 1.

\end{description}

\end{description}\end{quote}

\end{fulllineitems}

\index{plot\_group\_delay() (dsptoolbox.classes.filter\_class.Filter method)@\spxentry{plot\_group\_delay()}\spxextra{dsptoolbox.classes.filter\_class.Filter method}}

\begin{fulllineitems}
\phantomsection\label{\detokenize{classes:dsptoolbox.classes.filter_class.Filter.plot_group_delay}}
\pysigstartsignatures
\pysiglinewithargsret{\sphinxbfcode{\sphinxupquote{plot\_group\_delay}}}{\emph{\DUrole{n}{length\_samples}\DUrole{p}{:}\DUrole{w}{  }\DUrole{n}{\sphinxhref{https://docs.python.org/3/library/functions.html\#int}{int}}\DUrole{w}{  }\DUrole{o}{=}\DUrole{w}{  }\DUrole{default_value}{512}}, \emph{\DUrole{n}{range\_hz}\DUrole{o}{=}\DUrole{default_value}{{[}20, 20000.0{]}}}, \emph{\DUrole{n}{show\_info\_box}\DUrole{p}{:}\DUrole{w}{  }\DUrole{n}{\sphinxhref{https://docs.python.org/3/library/functions.html\#bool}{bool}}\DUrole{w}{  }\DUrole{o}{=}\DUrole{w}{  }\DUrole{default_value}{False}}, \emph{\DUrole{n}{returns}\DUrole{p}{:}\DUrole{w}{  }\DUrole{n}{\sphinxhref{https://docs.python.org/3/library/functions.html\#bool}{bool}}\DUrole{w}{  }\DUrole{o}{=}\DUrole{w}{  }\DUrole{default_value}{True}}}{}
\pysigstopsignatures
\sphinxAtStartPar
Plots group delay of the filter. Different methods are used for
FIR or IIR filters.
\begin{quote}\begin{description}
\sphinxlineitem{Parameters}\begin{description}
\sphinxlineitem{\sphinxstylestrong{length\_samples}}{[}int, optional{]}
\sphinxAtStartPar
Length of ir for magnitude plot. Default: 512.

\sphinxlineitem{\sphinxstylestrong{range\_hz}}{[}array\sphinxhyphen{}like with length 2, optional{]}
\sphinxAtStartPar
Range for which to plot the magnitude response.
Default: {[}20, 20000{]}.

\sphinxlineitem{\sphinxstylestrong{show\_info\_box}}{[}bool, optional{]}
\sphinxAtStartPar
Shows an information box on the plot. Default: \sphinxtitleref{False}.

\sphinxlineitem{\sphinxstylestrong{returns}}{[}bool, optional{]}
\sphinxAtStartPar
When \sphinxtitleref{True} figure and axis are returned. Default: \sphinxtitleref{False}.

\end{description}

\sphinxlineitem{Returns}\begin{description}
\sphinxlineitem{fig, ax}
\sphinxAtStartPar
Returned only when \sphinxtitleref{returns=True}.

\end{description}

\end{description}\end{quote}

\end{fulllineitems}

\index{plot\_magnitude() (dsptoolbox.classes.filter\_class.Filter method)@\spxentry{plot\_magnitude()}\spxextra{dsptoolbox.classes.filter\_class.Filter method}}

\begin{fulllineitems}
\phantomsection\label{\detokenize{classes:dsptoolbox.classes.filter_class.Filter.plot_magnitude}}
\pysigstartsignatures
\pysiglinewithargsret{\sphinxbfcode{\sphinxupquote{plot\_magnitude}}}{\emph{\DUrole{n}{length\_samples}\DUrole{p}{:}\DUrole{w}{  }\DUrole{n}{\sphinxhref{https://docs.python.org/3/library/functions.html\#int}{int}}\DUrole{w}{  }\DUrole{o}{=}\DUrole{w}{  }\DUrole{default_value}{512}}, \emph{\DUrole{n}{range\_hz}\DUrole{o}{=}\DUrole{default_value}{{[}20, 20000.0{]}}}, \emph{\DUrole{n}{normalize}\DUrole{p}{:}\DUrole{w}{  }\DUrole{n}{\sphinxhref{https://docs.python.org/3/library/typing.html\#typing.Optional}{Optional}\DUrole{p}{{[}}\sphinxhref{https://docs.python.org/3/library/stdtypes.html\#str}{str}\DUrole{p}{{]}}}\DUrole{w}{  }\DUrole{o}{=}\DUrole{w}{  }\DUrole{default_value}{None}}, \emph{\DUrole{n}{show\_info\_box}\DUrole{p}{:}\DUrole{w}{  }\DUrole{n}{\sphinxhref{https://docs.python.org/3/library/functions.html\#bool}{bool}}\DUrole{w}{  }\DUrole{o}{=}\DUrole{w}{  }\DUrole{default_value}{True}}, \emph{\DUrole{n}{returns}\DUrole{p}{:}\DUrole{w}{  }\DUrole{n}{\sphinxhref{https://docs.python.org/3/library/functions.html\#bool}{bool}}\DUrole{w}{  }\DUrole{o}{=}\DUrole{w}{  }\DUrole{default_value}{False}}}{}
\pysigstopsignatures
\sphinxAtStartPar
Plots magnitude spectrum.
Change parameters of spectrum with set\_spectrum\_parameters.
\begin{quote}\begin{description}
\sphinxlineitem{Parameters}\begin{description}
\sphinxlineitem{\sphinxstylestrong{length\_samples}}{[}int, optional{]}
\sphinxAtStartPar
Length of ir for magnitude plot. Default: 512.

\sphinxlineitem{\sphinxstylestrong{range\_hz}}{[}array\sphinxhyphen{}like with length 2, optional{]}
\sphinxAtStartPar
Range for which to plot the magnitude response.
Default: {[}20, 20000{]}.

\sphinxlineitem{\sphinxstylestrong{normalize}}{[}str, optional{]}
\sphinxAtStartPar
Mode for normalization, supported are \sphinxtitleref{‘1k’} for normalization
with value at frequency 1 kHz or \sphinxtitleref{‘max’} for normalization with
maximal value. Use \sphinxtitleref{None} for no normalization. Default: \sphinxtitleref{None}.

\sphinxlineitem{\sphinxstylestrong{show\_info\_box}}{[}bool, optional{]}
\sphinxAtStartPar
Shows an information box on the plot. Default: \sphinxtitleref{True}.

\sphinxlineitem{\sphinxstylestrong{returns}}{[}bool, optional{]}
\sphinxAtStartPar
When \sphinxtitleref{True} figure and axis are returned. Default: \sphinxtitleref{False}.

\end{description}

\sphinxlineitem{Returns}\begin{description}
\sphinxlineitem{fig, ax}
\sphinxAtStartPar
Returned only when \sphinxtitleref{returns=True}.

\end{description}

\end{description}\end{quote}

\end{fulllineitems}

\index{plot\_phase() (dsptoolbox.classes.filter\_class.Filter method)@\spxentry{plot\_phase()}\spxextra{dsptoolbox.classes.filter\_class.Filter method}}

\begin{fulllineitems}
\phantomsection\label{\detokenize{classes:dsptoolbox.classes.filter_class.Filter.plot_phase}}
\pysigstartsignatures
\pysiglinewithargsret{\sphinxbfcode{\sphinxupquote{plot\_phase}}}{\emph{\DUrole{n}{length\_samples}\DUrole{p}{:}\DUrole{w}{  }\DUrole{n}{\sphinxhref{https://docs.python.org/3/library/functions.html\#int}{int}}\DUrole{w}{  }\DUrole{o}{=}\DUrole{w}{  }\DUrole{default_value}{512}}, \emph{\DUrole{n}{range\_hz}\DUrole{o}{=}\DUrole{default_value}{{[}20, 20000.0{]}}}, \emph{\DUrole{n}{unwrap}\DUrole{p}{:}\DUrole{w}{  }\DUrole{n}{\sphinxhref{https://docs.python.org/3/library/functions.html\#bool}{bool}}\DUrole{w}{  }\DUrole{o}{=}\DUrole{w}{  }\DUrole{default_value}{False}}, \emph{\DUrole{n}{show\_info\_box}\DUrole{p}{:}\DUrole{w}{  }\DUrole{n}{\sphinxhref{https://docs.python.org/3/library/functions.html\#bool}{bool}}\DUrole{w}{  }\DUrole{o}{=}\DUrole{w}{  }\DUrole{default_value}{False}}, \emph{\DUrole{n}{returns}\DUrole{p}{:}\DUrole{w}{  }\DUrole{n}{\sphinxhref{https://docs.python.org/3/library/functions.html\#bool}{bool}}\DUrole{w}{  }\DUrole{o}{=}\DUrole{w}{  }\DUrole{default_value}{False}}}{}
\pysigstopsignatures
\sphinxAtStartPar
Plots phase spectrum.
\begin{quote}\begin{description}
\sphinxlineitem{Parameters}\begin{description}
\sphinxlineitem{\sphinxstylestrong{length\_samples}}{[}int, optional{]}
\sphinxAtStartPar
Length of ir for magnitude plot. Default: 512.

\sphinxlineitem{\sphinxstylestrong{range\_hz}}{[}array\sphinxhyphen{}like with length 2, optional{]}
\sphinxAtStartPar
Range for which to plot the magnitude response.
Default: {[}20, 20000{]}.

\sphinxlineitem{\sphinxstylestrong{unwrap}}{[}bool, optional{]}
\sphinxAtStartPar
Unwraps the phase to show. Default: \sphinxtitleref{False}.

\sphinxlineitem{\sphinxstylestrong{show\_info\_box}}{[}bool, optional{]}
\sphinxAtStartPar
Shows an information box on the plot. Default: \sphinxtitleref{False}.

\sphinxlineitem{\sphinxstylestrong{returns}}{[}bool, optional{]}
\sphinxAtStartPar
When \sphinxtitleref{True} figure and axis are returned. Default: \sphinxtitleref{False}.

\end{description}

\sphinxlineitem{Returns}\begin{description}
\sphinxlineitem{fig, ax}
\sphinxAtStartPar
Returned only when \sphinxtitleref{returns=True}.

\end{description}

\end{description}\end{quote}

\end{fulllineitems}

\index{plot\_zp() (dsptoolbox.classes.filter\_class.Filter method)@\spxentry{plot\_zp()}\spxextra{dsptoolbox.classes.filter\_class.Filter method}}

\begin{fulllineitems}
\phantomsection\label{\detokenize{classes:dsptoolbox.classes.filter_class.Filter.plot_zp}}
\pysigstartsignatures
\pysiglinewithargsret{\sphinxbfcode{\sphinxupquote{plot\_zp}}}{\emph{\DUrole{n}{show\_info\_box}\DUrole{p}{:}\DUrole{w}{  }\DUrole{n}{\sphinxhref{https://docs.python.org/3/library/functions.html\#bool}{bool}}\DUrole{w}{  }\DUrole{o}{=}\DUrole{w}{  }\DUrole{default_value}{False}}, \emph{\DUrole{n}{returns}\DUrole{p}{:}\DUrole{w}{  }\DUrole{n}{\sphinxhref{https://docs.python.org/3/library/functions.html\#bool}{bool}}\DUrole{w}{  }\DUrole{o}{=}\DUrole{w}{  }\DUrole{default_value}{False}}}{}
\pysigstopsignatures
\sphinxAtStartPar
Plots zeros and poles with the unit circle.
\begin{quote}\begin{description}
\sphinxlineitem{Parameters}\begin{description}
\sphinxlineitem{\sphinxstylestrong{returns}}{[}bool, optional{]}
\sphinxAtStartPar
When \sphinxtitleref{True} figure and axis are returned. Default: \sphinxtitleref{False}.

\sphinxlineitem{\sphinxstylestrong{show\_info\_box}}{[}bool, optional{]}
\sphinxAtStartPar
Shows an information box on the plot. Default: \sphinxtitleref{False}.

\end{description}

\sphinxlineitem{Returns}\begin{description}
\sphinxlineitem{fig, ax}
\sphinxAtStartPar
Returned only when \sphinxtitleref{returns=True}.

\end{description}

\end{description}\end{quote}

\end{fulllineitems}

\index{sampling\_rate\_hz (dsptoolbox.classes.filter\_class.Filter property)@\spxentry{sampling\_rate\_hz}\spxextra{dsptoolbox.classes.filter\_class.Filter property}}

\begin{fulllineitems}
\phantomsection\label{\detokenize{classes:dsptoolbox.classes.filter_class.Filter.sampling_rate_hz}}
\pysigstartsignatures
\pysigline{\sphinxbfcode{\sphinxupquote{property\DUrole{w}{  }}}\sphinxbfcode{\sphinxupquote{sampling\_rate\_hz}}}
\pysigstopsignatures
\end{fulllineitems}

\index{save\_filter() (dsptoolbox.classes.filter\_class.Filter method)@\spxentry{save\_filter()}\spxextra{dsptoolbox.classes.filter\_class.Filter method}}

\begin{fulllineitems}
\phantomsection\label{\detokenize{classes:dsptoolbox.classes.filter_class.Filter.save_filter}}
\pysigstartsignatures
\pysiglinewithargsret{\sphinxbfcode{\sphinxupquote{save\_filter}}}{\emph{\DUrole{n}{path}\DUrole{p}{:}\DUrole{w}{  }\DUrole{n}{\sphinxhref{https://docs.python.org/3/library/stdtypes.html\#str}{str}}\DUrole{w}{  }\DUrole{o}{=}\DUrole{w}{  }\DUrole{default_value}{\textquotesingle{}filter\textquotesingle{}}}}{}
\pysigstopsignatures
\sphinxAtStartPar
Saves the Filter object as a pickle.
\begin{quote}\begin{description}
\sphinxlineitem{Parameters}\begin{description}
\sphinxlineitem{\sphinxstylestrong{path}}{[}str, optional{]}
\sphinxAtStartPar
Path for the filter to be saved. Use only folder1/folder2/name
(without format). Default: \sphinxtitleref{‘filter’}
(local folder, object named filter).

\end{description}

\end{description}\end{quote}

\end{fulllineitems}

\index{set\_filter\_parameters() (dsptoolbox.classes.filter\_class.Filter method)@\spxentry{set\_filter\_parameters()}\spxextra{dsptoolbox.classes.filter\_class.Filter method}}

\begin{fulllineitems}
\phantomsection\label{\detokenize{classes:dsptoolbox.classes.filter_class.Filter.set_filter_parameters}}
\pysigstartsignatures
\pysiglinewithargsret{\sphinxbfcode{\sphinxupquote{set\_filter\_parameters}}}{\emph{\DUrole{n}{filter\_type}\DUrole{p}{:}\DUrole{w}{  }\DUrole{n}{\sphinxhref{https://docs.python.org/3/library/stdtypes.html\#str}{str}}}, \emph{\DUrole{n}{filter\_configuration}\DUrole{p}{:}\DUrole{w}{  }\DUrole{n}{\sphinxhref{https://docs.python.org/3/library/stdtypes.html\#dict}{dict}}}}{}
\pysigstopsignatures
\end{fulllineitems}

\index{show\_info() (dsptoolbox.classes.filter\_class.Filter method)@\spxentry{show\_info()}\spxextra{dsptoolbox.classes.filter\_class.Filter method}}

\begin{fulllineitems}
\phantomsection\label{\detokenize{classes:dsptoolbox.classes.filter_class.Filter.show_info}}
\pysigstartsignatures
\pysiglinewithargsret{\sphinxbfcode{\sphinxupquote{show\_info}}}{}{}
\pysigstopsignatures
\sphinxAtStartPar
Prints all the filter parameters to the console.

\end{fulllineitems}


\end{fulllineitems}



\section{Filterbank}
\label{\detokenize{classes:module-dsptoolbox.classes.filterbank}}\label{\detokenize{classes:filterbank}}\index{module@\spxentry{module}!dsptoolbox.classes.filterbank@\spxentry{dsptoolbox.classes.filterbank}}\index{dsptoolbox.classes.filterbank@\spxentry{dsptoolbox.classes.filterbank}!module@\spxentry{module}}\index{FilterBank (class in dsptoolbox.classes.filterbank)@\spxentry{FilterBank}\spxextra{class in dsptoolbox.classes.filterbank}}

\begin{fulllineitems}
\phantomsection\label{\detokenize{classes:dsptoolbox.classes.filterbank.FilterBank}}
\pysigstartsignatures
\pysiglinewithargsret{\sphinxbfcode{\sphinxupquote{class\DUrole{w}{  }}}\sphinxcode{\sphinxupquote{dsptoolbox.classes.filterbank.}}\sphinxbfcode{\sphinxupquote{FilterBank}}}{\emph{\DUrole{n}{filters}\DUrole{o}{=}\DUrole{default_value}{{[}{]}}}, \emph{\DUrole{n}{same\_sampling\_rate}\DUrole{p}{:}\DUrole{w}{  }\DUrole{n}{\sphinxhref{https://docs.python.org/3/library/functions.html\#bool}{bool}}\DUrole{w}{  }\DUrole{o}{=}\DUrole{w}{  }\DUrole{default_value}{True}}, \emph{\DUrole{n}{info}\DUrole{p}{:}\DUrole{w}{  }\DUrole{n}{\sphinxhref{https://docs.python.org/3/library/stdtypes.html\#dict}{dict}}\DUrole{w}{  }\DUrole{o}{=}\DUrole{w}{  }\DUrole{default_value}{\{\}}}}{}
\pysigstopsignatures
\sphinxAtStartPar
Bases: \sphinxhref{https://docs.python.org/3/library/functions.html\#object}{\sphinxcode{\sphinxupquote{object}}}

\sphinxAtStartPar
Standard template for filter banks containing filters, filters’ initial
values, metadata and some useful plotting methods.
\subsubsection*{Methods}


\begin{savenotes}\sphinxattablestart
\sphinxthistablewithglobalstyle
\sphinxthistablewithnovlinesstyle
\centering
\begin{tabulary}{\linewidth}[t]{\X{1}{2}\X{1}{2}}
\sphinxtoprule
\sphinxtableatstartofbodyhook
\sphinxAtStartPar
{\hyperref[\detokenize{classes:dsptoolbox.classes.filterbank.FilterBank.add_filter}]{\sphinxcrossref{\sphinxcode{\sphinxupquote{add\_filter}}}}}(filt{[}, index{]})
&
\sphinxAtStartPar
Adds a new filter at the end of the filters dictionary.
\\
\sphinxhline
\sphinxAtStartPar
{\hyperref[\detokenize{classes:dsptoolbox.classes.filterbank.FilterBank.copy}]{\sphinxcrossref{\sphinxcode{\sphinxupquote{copy}}}}}()
&
\sphinxAtStartPar
Returns a copy of the object.
\\
\sphinxhline
\sphinxAtStartPar
{\hyperref[\detokenize{classes:dsptoolbox.classes.filterbank.FilterBank.filter_signal}]{\sphinxcrossref{\sphinxcode{\sphinxupquote{filter\_signal}}}}}(signal{[}, mode, activate\_zi, ...{]})
&
\sphinxAtStartPar
Applies the filter bank to a signal and returns a multiband signal or a \sphinxtitleref{Signal} object.
\\
\sphinxhline
\sphinxAtStartPar
{\hyperref[\detokenize{classes:dsptoolbox.classes.filterbank.FilterBank.initialize_zi}]{\sphinxcrossref{\sphinxcode{\sphinxupquote{initialize\_zi}}}}}({[}number\_of\_channels{]})
&
\sphinxAtStartPar
Initiates the zi of the filters for the given number of channels.
\\
\sphinxhline
\sphinxAtStartPar
{\hyperref[\detokenize{classes:dsptoolbox.classes.filterbank.FilterBank.plot_group_delay}]{\sphinxcrossref{\sphinxcode{\sphinxupquote{plot\_group\_delay}}}}}({[}mode, range\_hz, test\_zi, ...{]})
&
\sphinxAtStartPar
Plots the phase response of each filter.
\\
\sphinxhline
\sphinxAtStartPar
{\hyperref[\detokenize{classes:dsptoolbox.classes.filterbank.FilterBank.plot_magnitude}]{\sphinxcrossref{\sphinxcode{\sphinxupquote{plot\_magnitude}}}}}({[}mode, range\_hz, test\_zi, ...{]})
&
\sphinxAtStartPar
Plots the magnitude response of each filter.
\\
\sphinxhline
\sphinxAtStartPar
{\hyperref[\detokenize{classes:dsptoolbox.classes.filterbank.FilterBank.plot_phase}]{\sphinxcrossref{\sphinxcode{\sphinxupquote{plot\_phase}}}}}({[}mode, range\_hz, test\_zi, ...{]})
&
\sphinxAtStartPar
Plots the phase response of each filter.
\\
\sphinxhline
\sphinxAtStartPar
{\hyperref[\detokenize{classes:dsptoolbox.classes.filterbank.FilterBank.remove_filter}]{\sphinxcrossref{\sphinxcode{\sphinxupquote{remove\_filter}}}}}({[}index, return\_filter{]})
&
\sphinxAtStartPar
Removes a filter from the filter bank.
\\
\sphinxhline
\sphinxAtStartPar
{\hyperref[\detokenize{classes:dsptoolbox.classes.filterbank.FilterBank.save_filterbank}]{\sphinxcrossref{\sphinxcode{\sphinxupquote{save\_filterbank}}}}}({[}path{]})
&
\sphinxAtStartPar
Saves the FilterBank object as a pickle.
\\
\sphinxhline
\sphinxAtStartPar
{\hyperref[\detokenize{classes:dsptoolbox.classes.filterbank.FilterBank.show_info}]{\sphinxcrossref{\sphinxcode{\sphinxupquote{show\_info}}}}}({[}show\_filter\_info{]})
&
\sphinxAtStartPar
Show information about the filter bank.
\\
\sphinxbottomrule
\end{tabulary}
\sphinxtableafterendhook\par
\sphinxattableend\end{savenotes}
\index{\_\_init\_\_() (dsptoolbox.classes.filterbank.FilterBank method)@\spxentry{\_\_init\_\_()}\spxextra{dsptoolbox.classes.filterbank.FilterBank method}}

\begin{fulllineitems}
\phantomsection\label{\detokenize{classes:dsptoolbox.classes.filterbank.FilterBank.__init__}}
\pysigstartsignatures
\pysiglinewithargsret{\sphinxbfcode{\sphinxupquote{\_\_init\_\_}}}{\emph{\DUrole{n}{filters}\DUrole{o}{=}\DUrole{default_value}{{[}{]}}}, \emph{\DUrole{n}{same\_sampling\_rate}\DUrole{p}{:}\DUrole{w}{  }\DUrole{n}{\sphinxhref{https://docs.python.org/3/library/functions.html\#bool}{bool}}\DUrole{w}{  }\DUrole{o}{=}\DUrole{w}{  }\DUrole{default_value}{True}}, \emph{\DUrole{n}{info}\DUrole{p}{:}\DUrole{w}{  }\DUrole{n}{\sphinxhref{https://docs.python.org/3/library/stdtypes.html\#dict}{dict}}\DUrole{w}{  }\DUrole{o}{=}\DUrole{w}{  }\DUrole{default_value}{\{\}}}}{}
\pysigstopsignatures
\sphinxAtStartPar
FilterBank object saves multiple filters and some metadata.
It also allows for easy filtering with multiple filters.
Since the digital filters that are supported are linear systems,
the order in which they are saved and applied to a signal is not
relevant.
\begin{quote}\begin{description}
\sphinxlineitem{Parameters}\begin{description}
\sphinxlineitem{\sphinxstylestrong{filters}}{[}list or tuple, optional{]}
\sphinxAtStartPar
List or tuple containing filters.

\sphinxlineitem{\sphinxstylestrong{same\_sampling\_rate}}{[}bool, optional{]}
\sphinxAtStartPar
When \sphinxtitleref{True}, every Filter should have the same sampling rate.
Set to \sphinxtitleref{False} for a multirate system. Default: \sphinxtitleref{True}.

\sphinxlineitem{\sphinxstylestrong{info}}{[}dict, optional{]}
\sphinxAtStartPar
Dictionary containing general information about the filter bank.
Some parameters of the filter bank are automatically read from
the filters dictionary.

\end{description}

\end{description}\end{quote}
\subsubsection*{Methods}


\begin{savenotes}\sphinxattablestart
\sphinxthistablewithglobalstyle
\centering
\begin{tabulary}{\linewidth}[t]{|T|T|}
\sphinxtoprule
\sphinxtableatstartofbodyhook
\sphinxAtStartPar
\sphinxstylestrong{General}
&
\sphinxAtStartPar
add\_filter, remove\_filter, save\_filterbank.
\\
\sphinxhline
\sphinxAtStartPar
\sphinxstylestrong{Prints and plots}
&
\sphinxAtStartPar
plot\_magnitude, plot\_phase, plot\_group\_delay, show\_info.
\\
\sphinxbottomrule
\end{tabulary}
\sphinxtableafterendhook\par
\sphinxattableend\end{savenotes}

\end{fulllineitems}

\index{add\_filter() (dsptoolbox.classes.filterbank.FilterBank method)@\spxentry{add\_filter()}\spxextra{dsptoolbox.classes.filterbank.FilterBank method}}

\begin{fulllineitems}
\phantomsection\label{\detokenize{classes:dsptoolbox.classes.filterbank.FilterBank.add_filter}}
\pysigstartsignatures
\pysiglinewithargsret{\sphinxbfcode{\sphinxupquote{add\_filter}}}{\emph{\DUrole{n}{filt}\DUrole{p}{:}\DUrole{w}{  }\DUrole{n}{{\hyperref[\detokenize{classes:dsptoolbox.classes.filter_class.Filter}]{\sphinxcrossref{Filter}}}}}, \emph{\DUrole{n}{index}\DUrole{p}{:}\DUrole{w}{  }\DUrole{n}{\sphinxhref{https://docs.python.org/3/library/functions.html\#int}{int}}\DUrole{w}{  }\DUrole{o}{=}\DUrole{w}{  }\DUrole{default_value}{\sphinxhyphen{}1}}}{}
\pysigstopsignatures
\sphinxAtStartPar
Adds a new filter at the end of the filters dictionary.
\begin{quote}\begin{description}
\sphinxlineitem{Parameters}\begin{description}
\sphinxlineitem{\sphinxstylestrong{filt}}{[}\sphinxtitleref{Filter}{]}
\sphinxAtStartPar
Filter to be added to the FilterBank.

\sphinxlineitem{\sphinxstylestrong{index}}{[}int, optional{]}
\sphinxAtStartPar
Index at which to insert the new Filter. Default: \sphinxhyphen{}1.

\end{description}

\end{description}\end{quote}

\end{fulllineitems}

\index{copy() (dsptoolbox.classes.filterbank.FilterBank method)@\spxentry{copy()}\spxextra{dsptoolbox.classes.filterbank.FilterBank method}}

\begin{fulllineitems}
\phantomsection\label{\detokenize{classes:dsptoolbox.classes.filterbank.FilterBank.copy}}
\pysigstartsignatures
\pysiglinewithargsret{\sphinxbfcode{\sphinxupquote{copy}}}{}{}
\pysigstopsignatures
\sphinxAtStartPar
Returns a copy of the object.
\begin{quote}\begin{description}
\sphinxlineitem{Returns}\begin{description}
\sphinxlineitem{\sphinxstylestrong{new\_sig}}{[}\sphinxtitleref{FilterBank}{]}
\sphinxAtStartPar
Copy of filter bank.

\end{description}

\end{description}\end{quote}

\end{fulllineitems}

\index{filter\_signal() (dsptoolbox.classes.filterbank.FilterBank method)@\spxentry{filter\_signal()}\spxextra{dsptoolbox.classes.filterbank.FilterBank method}}

\begin{fulllineitems}
\phantomsection\label{\detokenize{classes:dsptoolbox.classes.filterbank.FilterBank.filter_signal}}
\pysigstartsignatures
\pysiglinewithargsret{\sphinxbfcode{\sphinxupquote{filter\_signal}}}{\emph{\DUrole{n}{signal}\DUrole{p}{:}\DUrole{w}{  }\DUrole{n}{{\hyperref[\detokenize{classes:dsptoolbox.classes.signal_class.Signal}]{\sphinxcrossref{Signal}}}}}, \emph{\DUrole{n}{mode}\DUrole{p}{:}\DUrole{w}{  }\DUrole{n}{\sphinxhref{https://docs.python.org/3/library/stdtypes.html\#str}{str}}\DUrole{w}{  }\DUrole{o}{=}\DUrole{w}{  }\DUrole{default_value}{\textquotesingle{}parallel\textquotesingle{}}}, \emph{\DUrole{n}{activate\_zi}\DUrole{p}{:}\DUrole{w}{  }\DUrole{n}{\sphinxhref{https://docs.python.org/3/library/functions.html\#bool}{bool}}\DUrole{w}{  }\DUrole{o}{=}\DUrole{w}{  }\DUrole{default_value}{False}}, \emph{\DUrole{n}{zero\_phase}\DUrole{p}{:}\DUrole{w}{  }\DUrole{n}{\sphinxhref{https://docs.python.org/3/library/functions.html\#bool}{bool}}\DUrole{w}{  }\DUrole{o}{=}\DUrole{w}{  }\DUrole{default_value}{False}}}{}
\pysigstopsignatures
\sphinxAtStartPar
Applies the filter bank to a signal and returns a multiband signal
or a \sphinxtitleref{Signal} object.
\sphinxtitleref{‘parallel’}: returns a \sphinxtitleref{MultiBandSignal} object where each band is
the output of each filter.
\sphinxtitleref{‘sequential’}: applies each filter to the given Signal in a sequential
manner and returns output with same dimension.
\sphinxtitleref{‘summed’}: applies every filter as parallel and then summs the outputs
returning same dimensional output as input.
\begin{quote}\begin{description}
\sphinxlineitem{Parameters}\begin{description}
\sphinxlineitem{\sphinxstylestrong{signal}}{[}\sphinxtitleref{Signal}{]}
\sphinxAtStartPar
Signal to be filtered.

\sphinxlineitem{\sphinxstylestrong{mode}}{[}str, optional{]}
\sphinxAtStartPar
Way to apply filter bank to the signal. Supported modes are:
\sphinxtitleref{‘parallel’}, \sphinxtitleref{‘sequential’}, \sphinxtitleref{‘summed’}. Default: \sphinxtitleref{‘parallel’}.

\sphinxlineitem{\sphinxstylestrong{activate\_zi}}{[}bool, optional{]}
\sphinxAtStartPar
Takes in the filter initial values and updates them for
streaming purposes. Default: \sphinxtitleref{False}.

\sphinxlineitem{\sphinxstylestrong{zero\_phase}}{[}bool, optional{]}
\sphinxAtStartPar
Activates zero\_phase filtering for the filter bank. It cannot be
used at the same time with \sphinxtitleref{zi=True}. Default: \sphinxtitleref{False}.

\end{description}

\sphinxlineitem{Returns}\begin{description}
\sphinxlineitem{\sphinxstylestrong{new\_sig}}{[}\sphinxtitleref{‘sequential’} or \sphinxtitleref{‘summed’} \sphinxhyphen{}\textgreater{} \sphinxtitleref{Signal}.{]}\begin{quote}

\sphinxAtStartPar
\sphinxtitleref{‘parallel’} \sphinxhyphen{}\textgreater{} \sphinxtitleref{MultiBandSignal}.
\end{quote}

\sphinxAtStartPar
New signal after filtering.

\end{description}

\end{description}\end{quote}

\end{fulllineitems}

\index{initialize\_zi() (dsptoolbox.classes.filterbank.FilterBank method)@\spxentry{initialize\_zi()}\spxextra{dsptoolbox.classes.filterbank.FilterBank method}}

\begin{fulllineitems}
\phantomsection\label{\detokenize{classes:dsptoolbox.classes.filterbank.FilterBank.initialize_zi}}
\pysigstartsignatures
\pysiglinewithargsret{\sphinxbfcode{\sphinxupquote{initialize\_zi}}}{\emph{\DUrole{n}{number\_of\_channels}\DUrole{p}{:}\DUrole{w}{  }\DUrole{n}{\sphinxhref{https://docs.python.org/3/library/functions.html\#int}{int}}\DUrole{w}{  }\DUrole{o}{=}\DUrole{w}{  }\DUrole{default_value}{1}}}{}
\pysigstopsignatures
\sphinxAtStartPar
Initiates the zi of the filters for the given number of channels.
\begin{quote}\begin{description}
\sphinxlineitem{Parameters}\begin{description}
\sphinxlineitem{\sphinxstylestrong{number\_of\_channels}}{[}int, optional{]}
\sphinxAtStartPar
Number of channels is needed for the number of filters’ zi’s.
Default: 1.

\end{description}

\end{description}\end{quote}

\end{fulllineitems}

\index{plot\_group\_delay() (dsptoolbox.classes.filterbank.FilterBank method)@\spxentry{plot\_group\_delay()}\spxextra{dsptoolbox.classes.filterbank.FilterBank method}}

\begin{fulllineitems}
\phantomsection\label{\detokenize{classes:dsptoolbox.classes.filterbank.FilterBank.plot_group_delay}}
\pysigstartsignatures
\pysiglinewithargsret{\sphinxbfcode{\sphinxupquote{plot\_group\_delay}}}{\emph{\DUrole{n}{mode}\DUrole{p}{:}\DUrole{w}{  }\DUrole{n}{\sphinxhref{https://docs.python.org/3/library/stdtypes.html\#str}{str}}\DUrole{w}{  }\DUrole{o}{=}\DUrole{w}{  }\DUrole{default_value}{\textquotesingle{}parallel\textquotesingle{}}}, \emph{\DUrole{n}{range\_hz}\DUrole{o}{=}\DUrole{default_value}{{[}20, 20000.0{]}}}, \emph{\DUrole{n}{test\_zi}\DUrole{p}{:}\DUrole{w}{  }\DUrole{n}{\sphinxhref{https://docs.python.org/3/library/functions.html\#bool}{bool}}\DUrole{w}{  }\DUrole{o}{=}\DUrole{w}{  }\DUrole{default_value}{False}}, \emph{\DUrole{n}{returns}\DUrole{p}{:}\DUrole{w}{  }\DUrole{n}{\sphinxhref{https://docs.python.org/3/library/functions.html\#bool}{bool}}\DUrole{w}{  }\DUrole{o}{=}\DUrole{w}{  }\DUrole{default_value}{False}}}{}
\pysigstopsignatures
\sphinxAtStartPar
Plots the phase response of each filter.
\begin{quote}\begin{description}
\sphinxlineitem{Parameters}\begin{description}
\sphinxlineitem{\sphinxstylestrong{mode}}{[}str, optional{]}
\sphinxAtStartPar
Type of plot. \sphinxtitleref{‘parallel’} plots every filter’s frequency response,
\sphinxtitleref{‘sequential’} plots the frequency response after having filtered
one impulse with every filter in the FilterBank. \sphinxtitleref{‘summed’}
sums up every filter output. Default: \sphinxtitleref{‘parallel’}.

\sphinxlineitem{\sphinxstylestrong{range\_hz}}{[}array\sphinxhyphen{}like, optional{]}
\sphinxAtStartPar
Range of Hz to plot. Default: {[}20, 20e3{]}.

\sphinxlineitem{\sphinxstylestrong{test\_zi}}{[}bool, optional{]}
\sphinxAtStartPar
Uses the zi’s of each filter to test the FilterBank’s output.
Default: \sphinxtitleref{False}.

\sphinxlineitem{\sphinxstylestrong{returns}}{[}bool, optional{]}
\sphinxAtStartPar
When \sphinxtitleref{True}, the figure and axis are returned. Default: \sphinxtitleref{False}.

\end{description}

\sphinxlineitem{Returns}\begin{description}
\sphinxlineitem{fig, ax}
\sphinxAtStartPar
Returned only when \sphinxtitleref{returns=True}.

\end{description}

\end{description}\end{quote}

\end{fulllineitems}

\index{plot\_magnitude() (dsptoolbox.classes.filterbank.FilterBank method)@\spxentry{plot\_magnitude()}\spxextra{dsptoolbox.classes.filterbank.FilterBank method}}

\begin{fulllineitems}
\phantomsection\label{\detokenize{classes:dsptoolbox.classes.filterbank.FilterBank.plot_magnitude}}
\pysigstartsignatures
\pysiglinewithargsret{\sphinxbfcode{\sphinxupquote{plot\_magnitude}}}{\emph{\DUrole{n}{mode}\DUrole{p}{:}\DUrole{w}{  }\DUrole{n}{\sphinxhref{https://docs.python.org/3/library/stdtypes.html\#str}{str}}\DUrole{w}{  }\DUrole{o}{=}\DUrole{w}{  }\DUrole{default_value}{\textquotesingle{}parallel\textquotesingle{}}}, \emph{\DUrole{n}{range\_hz}\DUrole{o}{=}\DUrole{default_value}{{[}20, 20000.0{]}}}, \emph{\DUrole{n}{test\_zi}\DUrole{p}{:}\DUrole{w}{  }\DUrole{n}{\sphinxhref{https://docs.python.org/3/library/functions.html\#bool}{bool}}\DUrole{w}{  }\DUrole{o}{=}\DUrole{w}{  }\DUrole{default_value}{False}}, \emph{\DUrole{n}{returns}\DUrole{p}{:}\DUrole{w}{  }\DUrole{n}{\sphinxhref{https://docs.python.org/3/library/functions.html\#bool}{bool}}\DUrole{w}{  }\DUrole{o}{=}\DUrole{w}{  }\DUrole{default_value}{False}}}{}
\pysigstopsignatures
\sphinxAtStartPar
Plots the magnitude response of each filter.
\begin{quote}\begin{description}
\sphinxlineitem{Parameters}\begin{description}
\sphinxlineitem{\sphinxstylestrong{mode}}{[}str, optional{]}
\sphinxAtStartPar
Type of plot. \sphinxtitleref{‘parallel’} plots every filter’s frequency response,
\sphinxtitleref{‘sequential’} plots the frequency response after having filtered
one impulse with every filter in the FilterBank. \sphinxtitleref{‘summed’}
sums up every frequency response. Default: \sphinxtitleref{‘parallel’}.

\sphinxlineitem{\sphinxstylestrong{range\_hz}}{[}array\sphinxhyphen{}like, optional{]}
\sphinxAtStartPar
Range of Hz to plot. Default: {[}20, 20e3{]}.

\sphinxlineitem{\sphinxstylestrong{test\_zi}}{[}bool, optional{]}
\sphinxAtStartPar
Uses the zi’s of each filter to test the FilterBank’s output.
Default: \sphinxtitleref{False}.

\sphinxlineitem{\sphinxstylestrong{returns}}{[}bool, optional{]}
\sphinxAtStartPar
When \sphinxtitleref{True}, the figure and axis are returned. Default: \sphinxtitleref{False}.

\end{description}

\sphinxlineitem{Returns}\begin{description}
\sphinxlineitem{fig, ax}
\sphinxAtStartPar
Returned only when \sphinxtitleref{returns=True}.

\end{description}

\end{description}\end{quote}

\end{fulllineitems}

\index{plot\_phase() (dsptoolbox.classes.filterbank.FilterBank method)@\spxentry{plot\_phase()}\spxextra{dsptoolbox.classes.filterbank.FilterBank method}}

\begin{fulllineitems}
\phantomsection\label{\detokenize{classes:dsptoolbox.classes.filterbank.FilterBank.plot_phase}}
\pysigstartsignatures
\pysiglinewithargsret{\sphinxbfcode{\sphinxupquote{plot\_phase}}}{\emph{\DUrole{n}{mode}\DUrole{p}{:}\DUrole{w}{  }\DUrole{n}{\sphinxhref{https://docs.python.org/3/library/stdtypes.html\#str}{str}}\DUrole{w}{  }\DUrole{o}{=}\DUrole{w}{  }\DUrole{default_value}{\textquotesingle{}parallel\textquotesingle{}}}, \emph{\DUrole{n}{range\_hz}\DUrole{o}{=}\DUrole{default_value}{{[}20, 20000.0{]}}}, \emph{\DUrole{n}{test\_zi}\DUrole{p}{:}\DUrole{w}{  }\DUrole{n}{\sphinxhref{https://docs.python.org/3/library/functions.html\#bool}{bool}}\DUrole{w}{  }\DUrole{o}{=}\DUrole{w}{  }\DUrole{default_value}{False}}, \emph{\DUrole{n}{unwrap}\DUrole{p}{:}\DUrole{w}{  }\DUrole{n}{\sphinxhref{https://docs.python.org/3/library/functions.html\#bool}{bool}}\DUrole{w}{  }\DUrole{o}{=}\DUrole{w}{  }\DUrole{default_value}{False}}, \emph{\DUrole{n}{returns}\DUrole{p}{:}\DUrole{w}{  }\DUrole{n}{\sphinxhref{https://docs.python.org/3/library/functions.html\#bool}{bool}}\DUrole{w}{  }\DUrole{o}{=}\DUrole{w}{  }\DUrole{default_value}{False}}}{}
\pysigstopsignatures
\sphinxAtStartPar
Plots the phase response of each filter.
\begin{quote}\begin{description}
\sphinxlineitem{Parameters}\begin{description}
\sphinxlineitem{\sphinxstylestrong{mode}}{[}str, optional{]}
\sphinxAtStartPar
Type of plot. \sphinxtitleref{‘parallel’} plots every filter’s frequency response,
\sphinxtitleref{‘sequential’} plots the frequency response after having filtered
one impulse with every filter in the FilterBank. \sphinxtitleref{‘summed’}
sums up every filter output. Default: \sphinxtitleref{‘parallel’}.

\sphinxlineitem{\sphinxstylestrong{range\_hz}}{[}array\sphinxhyphen{}like, optional{]}
\sphinxAtStartPar
Range of Hz to plot. Default: {[}20, 20e3{]}.

\sphinxlineitem{\sphinxstylestrong{test\_zi}}{[}bool, optional{]}
\sphinxAtStartPar
Uses the zi’s of each filter to test the FilterBank’s output.
Default: \sphinxtitleref{False}.

\sphinxlineitem{\sphinxstylestrong{unwrap}}{[}bool, optional{]}
\sphinxAtStartPar
When \sphinxtitleref{True}, unwrapped phase is plotted. Default: \sphinxtitleref{False}.

\sphinxlineitem{\sphinxstylestrong{returns}}{[}bool, optional{]}
\sphinxAtStartPar
When \sphinxtitleref{True}, the figure and axis are returned. Default: \sphinxtitleref{False}.

\end{description}

\sphinxlineitem{Returns}\begin{description}
\sphinxlineitem{fig, ax}
\sphinxAtStartPar
Returned only when \sphinxtitleref{returns=True}.

\end{description}

\end{description}\end{quote}

\end{fulllineitems}

\index{remove\_filter() (dsptoolbox.classes.filterbank.FilterBank method)@\spxentry{remove\_filter()}\spxextra{dsptoolbox.classes.filterbank.FilterBank method}}

\begin{fulllineitems}
\phantomsection\label{\detokenize{classes:dsptoolbox.classes.filterbank.FilterBank.remove_filter}}
\pysigstartsignatures
\pysiglinewithargsret{\sphinxbfcode{\sphinxupquote{remove\_filter}}}{\emph{\DUrole{n}{index}\DUrole{p}{:}\DUrole{w}{  }\DUrole{n}{\sphinxhref{https://docs.python.org/3/library/functions.html\#int}{int}}\DUrole{w}{  }\DUrole{o}{=}\DUrole{w}{  }\DUrole{default_value}{\sphinxhyphen{}1}}, \emph{\DUrole{n}{return\_filter}\DUrole{p}{:}\DUrole{w}{  }\DUrole{n}{\sphinxhref{https://docs.python.org/3/library/functions.html\#bool}{bool}}\DUrole{w}{  }\DUrole{o}{=}\DUrole{w}{  }\DUrole{default_value}{False}}}{}
\pysigstopsignatures
\sphinxAtStartPar
Removes a filter from the filter bank.
\begin{quote}\begin{description}
\sphinxlineitem{Parameters}\begin{description}
\sphinxlineitem{\sphinxstylestrong{index}}{[}int, optional{]}
\sphinxAtStartPar
This is the index from the filters list at which the filter
will be erased. When \sphinxhyphen{}1, last filter is erased.
Default: \sphinxhyphen{}1.

\sphinxlineitem{\sphinxstylestrong{return\_filter}}{[}bool, optional{]}
\sphinxAtStartPar
When \sphinxtitleref{True}, the erased filter is returned. Default: \sphinxtitleref{False}.

\end{description}

\end{description}\end{quote}

\end{fulllineitems}

\index{save\_filterbank() (dsptoolbox.classes.filterbank.FilterBank method)@\spxentry{save\_filterbank()}\spxextra{dsptoolbox.classes.filterbank.FilterBank method}}

\begin{fulllineitems}
\phantomsection\label{\detokenize{classes:dsptoolbox.classes.filterbank.FilterBank.save_filterbank}}
\pysigstartsignatures
\pysiglinewithargsret{\sphinxbfcode{\sphinxupquote{save\_filterbank}}}{\emph{\DUrole{n}{path}\DUrole{p}{:}\DUrole{w}{  }\DUrole{n}{\sphinxhref{https://docs.python.org/3/library/stdtypes.html\#str}{str}}\DUrole{w}{  }\DUrole{o}{=}\DUrole{w}{  }\DUrole{default_value}{\textquotesingle{}filterbank\textquotesingle{}}}}{}
\pysigstopsignatures
\sphinxAtStartPar
Saves the FilterBank object as a pickle.
\begin{quote}\begin{description}
\sphinxlineitem{Parameters}\begin{description}
\sphinxlineitem{\sphinxstylestrong{path}}{[}str, optional{]}
\sphinxAtStartPar
Path for the filterbank to be saved. Use only folder1/folder2/name
(without format). Default: \sphinxtitleref{‘filterbank’}
(local folder, object named filterbank).

\end{description}

\end{description}\end{quote}

\end{fulllineitems}

\index{show\_info() (dsptoolbox.classes.filterbank.FilterBank method)@\spxentry{show\_info()}\spxextra{dsptoolbox.classes.filterbank.FilterBank method}}

\begin{fulllineitems}
\phantomsection\label{\detokenize{classes:dsptoolbox.classes.filterbank.FilterBank.show_info}}
\pysigstartsignatures
\pysiglinewithargsret{\sphinxbfcode{\sphinxupquote{show\_info}}}{\emph{\DUrole{n}{show\_filter\_info}\DUrole{p}{:}\DUrole{w}{  }\DUrole{n}{\sphinxhref{https://docs.python.org/3/library/functions.html\#bool}{bool}}\DUrole{w}{  }\DUrole{o}{=}\DUrole{w}{  }\DUrole{default_value}{True}}}{}
\pysigstopsignatures
\sphinxAtStartPar
Show information about the filter bank.
\begin{quote}\begin{description}
\sphinxlineitem{Parameters}\begin{description}
\sphinxlineitem{\sphinxstylestrong{show\_filters\_info}}{[}bool, optional{]}
\sphinxAtStartPar
When \sphinxtitleref{True}, a longer message is printed with all available
information regarding each filter in the filter bank.
Default: \sphinxtitleref{True}.

\end{description}

\end{description}\end{quote}

\end{fulllineitems}


\end{fulllineitems}


\sphinxstepscope


\chapter{Modules in dsptoolbox}
\label{\detokenize{modules:modules-in-dsptoolbox}}\label{\detokenize{modules::doc}}
\sphinxAtStartPar
The modules and functions of dsptoolbox are listed down below.

\sphinxstepscope


\section{Distances (dsptoolbox.distances)}
\label{\detokenize{modules/dsptoolbox.distances:module-dsptoolbox.distances}}\label{\detokenize{modules/dsptoolbox.distances:distances-dsptoolbox-distances}}\label{\detokenize{modules/dsptoolbox.distances::doc}}\index{module@\spxentry{module}!dsptoolbox.distances@\spxentry{dsptoolbox.distances}}\index{dsptoolbox.distances@\spxentry{dsptoolbox.distances}!module@\spxentry{module}}\index{itakura\_saito() (in module dsptoolbox.distances)@\spxentry{itakura\_saito()}\spxextra{in module dsptoolbox.distances}}

\begin{fulllineitems}
\phantomsection\label{\detokenize{modules/dsptoolbox.distances:dsptoolbox.distances.itakura_saito}}
\pysigstartsignatures
\pysiglinewithargsret{\sphinxcode{\sphinxupquote{dsptoolbox.distances.}}\sphinxbfcode{\sphinxupquote{itakura\_saito}}}{\emph{\DUrole{n}{insig1}\DUrole{p}{:}\DUrole{w}{  }\DUrole{n}{{\hyperref[\detokenize{classes:dsptoolbox.classes.signal_class.Signal}]{\sphinxcrossref{Signal}}}}}, \emph{\DUrole{n}{insig2}\DUrole{p}{:}\DUrole{w}{  }\DUrole{n}{{\hyperref[\detokenize{classes:dsptoolbox.classes.signal_class.Signal}]{\sphinxcrossref{Signal}}}}}, \emph{\DUrole{n}{method}\DUrole{p}{:}\DUrole{w}{  }\DUrole{n}{\sphinxhref{https://docs.python.org/3/library/stdtypes.html\#str}{str}}\DUrole{w}{  }\DUrole{o}{=}\DUrole{w}{  }\DUrole{default_value}{\textquotesingle{}welch\textquotesingle{}}}, \emph{\DUrole{n}{f\_range\_hz}\DUrole{o}{=}\DUrole{default_value}{{[}20, 20000{]}}}, \emph{\DUrole{o}{**}\DUrole{n}{kwargs}}}{}
\pysigstopsignatures
\sphinxAtStartPar
Computes itakura\sphinxhyphen{}saito measure between two signals. Beware that this
measure is not symmetric (x, y) != (y, x).
\begin{quote}\begin{description}
\sphinxlineitem{Parameters}\begin{description}
\sphinxlineitem{\sphinxstylestrong{insig1}}{[}Signal{]}
\sphinxAtStartPar
Signal 1.

\sphinxlineitem{\sphinxstylestrong{insig2}}{[}Signal{]}
\sphinxAtStartPar
Signal 2.

\sphinxlineitem{\sphinxstylestrong{f\_range\_hz}}{[}array, optional{]}
\sphinxAtStartPar
Range of frequencies in which to compute the distance. When \sphinxtitleref{None},
it is computed in all frequencies. Default: {[}20, 20000{]}.

\end{description}

\end{description}\end{quote}

\end{fulllineitems}

\index{log\_spectral() (in module dsptoolbox.distances)@\spxentry{log\_spectral()}\spxextra{in module dsptoolbox.distances}}

\begin{fulllineitems}
\phantomsection\label{\detokenize{modules/dsptoolbox.distances:dsptoolbox.distances.log_spectral}}
\pysigstartsignatures
\pysiglinewithargsret{\sphinxcode{\sphinxupquote{dsptoolbox.distances.}}\sphinxbfcode{\sphinxupquote{log\_spectral}}}{\emph{\DUrole{n}{insig1}\DUrole{p}{:}\DUrole{w}{  }\DUrole{n}{{\hyperref[\detokenize{classes:dsptoolbox.classes.signal_class.Signal}]{\sphinxcrossref{Signal}}}}}, \emph{\DUrole{n}{insig2}\DUrole{p}{:}\DUrole{w}{  }\DUrole{n}{{\hyperref[\detokenize{classes:dsptoolbox.classes.signal_class.Signal}]{\sphinxcrossref{Signal}}}}}, \emph{\DUrole{n}{method}\DUrole{p}{:}\DUrole{w}{  }\DUrole{n}{\sphinxhref{https://docs.python.org/3/library/stdtypes.html\#str}{str}}\DUrole{w}{  }\DUrole{o}{=}\DUrole{w}{  }\DUrole{default_value}{\textquotesingle{}welch\textquotesingle{}}}, \emph{\DUrole{n}{f\_range\_hz}\DUrole{o}{=}\DUrole{default_value}{{[}20, 20000{]}}}, \emph{\DUrole{o}{**}\DUrole{n}{kwargs}}}{}
\pysigstopsignatures
\sphinxAtStartPar
Computes log spectral distance between two signals.
\begin{quote}\begin{description}
\sphinxlineitem{Parameters}\begin{description}
\sphinxlineitem{\sphinxstylestrong{insig1}}{[}Signal{]}
\sphinxAtStartPar
Signal 1.

\sphinxlineitem{\sphinxstylestrong{insig2}}{[}Signal{]}
\sphinxAtStartPar
Signal 2.

\sphinxlineitem{\sphinxstylestrong{f\_range\_hz}}{[}array, optional{]}
\sphinxAtStartPar
Range of frequencies in which to compute the distance. When \sphinxtitleref{None},
it is computed in all frequencies. Default: {[}20, 20000{]}.

\end{description}

\end{description}\end{quote}

\end{fulllineitems}


\sphinxstepscope


\section{Filterbanks (dsptoolbox.filterbanks)}
\label{\detokenize{modules/dsptoolbox.filterbanks:module-dsptoolbox.filterbanks}}\label{\detokenize{modules/dsptoolbox.filterbanks:filterbanks-dsptoolbox-filterbanks}}\label{\detokenize{modules/dsptoolbox.filterbanks::doc}}\index{module@\spxentry{module}!dsptoolbox.filterbanks@\spxentry{dsptoolbox.filterbanks}}\index{dsptoolbox.filterbanks@\spxentry{dsptoolbox.filterbanks}!module@\spxentry{module}}\index{linkwitz\_riley\_crossovers() (in module dsptoolbox.filterbanks)@\spxentry{linkwitz\_riley\_crossovers()}\spxextra{in module dsptoolbox.filterbanks}}

\begin{fulllineitems}
\phantomsection\label{\detokenize{modules/dsptoolbox.filterbanks:dsptoolbox.filterbanks.linkwitz_riley_crossovers}}
\pysigstartsignatures
\pysiglinewithargsret{\sphinxcode{\sphinxupquote{dsptoolbox.filterbanks.}}\sphinxbfcode{\sphinxupquote{linkwitz\_riley\_crossovers}}}{\emph{\DUrole{n}{freqs}}, \emph{\DUrole{n}{order}}, \emph{\DUrole{n}{sampling\_rate\_hz}\DUrole{p}{:}\DUrole{w}{  }\DUrole{n}{\sphinxhref{https://docs.python.org/3/library/functions.html\#int}{int}}\DUrole{w}{  }\DUrole{o}{=}\DUrole{w}{  }\DUrole{default_value}{48000}}}{}
\pysigstopsignatures
\sphinxAtStartPar
Returns a linkwitz\sphinxhyphen{}riley crossovers filter bank.
\begin{quote}\begin{description}
\sphinxlineitem{Parameters}\begin{description}
\sphinxlineitem{\sphinxstylestrong{freqs}}{[}array\sphinxhyphen{}like{]}
\sphinxAtStartPar
Frequencies at which to set the crossovers.

\sphinxlineitem{\sphinxstylestrong{order}}{[}array\sphinxhyphen{}like{]}
\sphinxAtStartPar
Order of the crossovers. The higher, the steeper.

\sphinxlineitem{\sphinxstylestrong{sampling\_rate\_hz}}{[}int, optional{]}
\sphinxAtStartPar
Sampling rate for the filterbank. Default: 48000.

\end{description}

\sphinxlineitem{Returns}\begin{description}
\sphinxlineitem{\sphinxstylestrong{fb}}{[}LRFilterBank{]}
\sphinxAtStartPar
Filter bank in form of LRFilterBank class which contains the same
methods as the FilterBank class but is generated with different
internal methods.

\end{description}

\end{description}\end{quote}

\end{fulllineitems}

\index{reconstructing\_fractional\_octave\_bands() (in module dsptoolbox.filterbanks)@\spxentry{reconstructing\_fractional\_octave\_bands()}\spxextra{in module dsptoolbox.filterbanks}}

\begin{fulllineitems}
\phantomsection\label{\detokenize{modules/dsptoolbox.filterbanks:dsptoolbox.filterbanks.reconstructing_fractional_octave_bands}}
\pysigstartsignatures
\pysiglinewithargsret{\sphinxcode{\sphinxupquote{dsptoolbox.filterbanks.}}\sphinxbfcode{\sphinxupquote{reconstructing\_fractional\_octave\_bands}}}{\emph{\DUrole{n}{num\_fractions}\DUrole{p}{:}\DUrole{w}{  }\DUrole{n}{\sphinxhref{https://docs.python.org/3/library/functions.html\#int}{int}}\DUrole{w}{  }\DUrole{o}{=}\DUrole{w}{  }\DUrole{default_value}{1}}, \emph{\DUrole{n}{frequency\_range}\DUrole{o}{=}\DUrole{default_value}{{[}63, 16000{]}}}, \emph{\DUrole{n}{overlap}\DUrole{p}{:}\DUrole{w}{  }\DUrole{n}{\sphinxhref{https://docs.python.org/3/library/functions.html\#float}{float}}\DUrole{w}{  }\DUrole{o}{=}\DUrole{w}{  }\DUrole{default_value}{1}}, \emph{\DUrole{n}{slope}\DUrole{p}{:}\DUrole{w}{  }\DUrole{n}{\sphinxhref{https://docs.python.org/3/library/functions.html\#int}{int}}\DUrole{w}{  }\DUrole{o}{=}\DUrole{w}{  }\DUrole{default_value}{0}}, \emph{\DUrole{n}{n\_samples}\DUrole{p}{:}\DUrole{w}{  }\DUrole{n}{\sphinxhref{https://docs.python.org/3/library/functions.html\#int}{int}}\DUrole{w}{  }\DUrole{o}{=}\DUrole{w}{  }\DUrole{default_value}{4096}}, \emph{\DUrole{n}{sampling\_rate\_hz}\DUrole{p}{:}\DUrole{w}{  }\DUrole{n}{\sphinxhref{https://docs.python.org/3/library/functions.html\#int}{int}}\DUrole{w}{  }\DUrole{o}{=}\DUrole{w}{  }\DUrole{default_value}{48000}}}{}
\pysigstopsignatures
\sphinxAtStartPar
Create and/or apply an amplitude preserving fractional octave filter
bank. This implementation is taken from the pyfar package.
See references for more information about it.
\begin{quote}\begin{description}
\sphinxlineitem{Parameters}\begin{description}
\sphinxlineitem{\sphinxstylestrong{num\_fractions}}{[}int, optional{]}
\sphinxAtStartPar
Octave fraction, e.g., \sphinxcode{\sphinxupquote{3}} for third\sphinxhyphen{}octave bands. The default is
\sphinxcode{\sphinxupquote{1}}.

\sphinxlineitem{\sphinxstylestrong{frequency\_range}}{[}tuple, optional{]}
\sphinxAtStartPar
Frequency range for fractional octave in Hz. The default is
\sphinxcode{\sphinxupquote{(63, 16000)}}

\sphinxlineitem{\sphinxstylestrong{overlap}}{[}float{]}
\sphinxAtStartPar
Band overlap of the filter slopes between 0 and 1. Smaller values yield
wider pass\sphinxhyphen{}bands and steeper filter slopes. The default is \sphinxcode{\sphinxupquote{1}}.

\sphinxlineitem{\sphinxstylestrong{slope}}{[}int, optional{]}
\sphinxAtStartPar
Number \textgreater{} 0 that defines the width and steepness of the filter slopes.
Larger values yield wider pass\sphinxhyphen{}bands and steeper filter slopes. The
default is \sphinxcode{\sphinxupquote{0}}.

\sphinxlineitem{\sphinxstylestrong{n\_samples}}{[}int, optional{]}
\sphinxAtStartPar
Length of the filter in samples. Longer filters yield more exact
filters. The default is \sphinxcode{\sphinxupquote{2**12}}.

\sphinxlineitem{\sphinxstylestrong{sampling\_rate}}{[}int{]}
\sphinxAtStartPar
Sampling frequency in Hz. The default is \sphinxcode{\sphinxupquote{None}}. Only required if
\sphinxcode{\sphinxupquote{signal=None}}.

\end{description}

\sphinxlineitem{Returns}\begin{description}
\sphinxlineitem{\sphinxstylestrong{signal}}{[}Signal{]}
\sphinxAtStartPar
The filtered signal. Only returned if \sphinxcode{\sphinxupquote{sampling\_rate = None}}.

\sphinxlineitem{\sphinxstylestrong{filter}}{[}FilterFIR{]}
\sphinxAtStartPar
FIR Filter object. Only returned if \sphinxcode{\sphinxupquote{signal = None}}.

\sphinxlineitem{\sphinxstylestrong{frequencies}}{[}np.ndarray{]}
\sphinxAtStartPar
Center frequencies of the filters.

\end{description}

\end{description}\end{quote}
\subsubsection*{References}
\begin{itemize}
\item {} 
\sphinxAtStartPar
\sphinxurl{https://pubmed.ncbi.nlm.nih.gov/20136211/}

\item {} 
\sphinxAtStartPar
\sphinxurl{https://github.com/pyfar/pyfar}

\end{itemize}

\end{fulllineitems}


\sphinxstepscope


\section{Generators (dsptoolbox.generators)}
\label{\detokenize{modules/dsptoolbox.generators:module-dsptoolbox.generators}}\label{\detokenize{modules/dsptoolbox.generators:generators-dsptoolbox-generators}}\label{\detokenize{modules/dsptoolbox.generators::doc}}\index{module@\spxentry{module}!dsptoolbox.generators@\spxentry{dsptoolbox.generators}}\index{dsptoolbox.generators@\spxentry{dsptoolbox.generators}!module@\spxentry{module}}\index{chirp() (in module dsptoolbox.generators)@\spxentry{chirp()}\spxextra{in module dsptoolbox.generators}}

\begin{fulllineitems}
\phantomsection\label{\detokenize{modules/dsptoolbox.generators:dsptoolbox.generators.chirp}}
\pysigstartsignatures
\pysiglinewithargsret{\sphinxcode{\sphinxupquote{dsptoolbox.generators.}}\sphinxbfcode{\sphinxupquote{chirp}}}{\emph{\DUrole{n}{type\_of\_chirp}\DUrole{p}{:}\DUrole{w}{  }\DUrole{n}{\sphinxhref{https://docs.python.org/3/library/stdtypes.html\#str}{str}}\DUrole{w}{  }\DUrole{o}{=}\DUrole{w}{  }\DUrole{default_value}{\textquotesingle{}log\textquotesingle{}}}, \emph{\DUrole{n}{range\_hz}\DUrole{o}{=}\DUrole{default_value}{None}}, \emph{\DUrole{n}{length\_seconds}\DUrole{p}{:}\DUrole{w}{  }\DUrole{n}{\sphinxhref{https://docs.python.org/3/library/functions.html\#float}{float}}\DUrole{w}{  }\DUrole{o}{=}\DUrole{w}{  }\DUrole{default_value}{1}}, \emph{\DUrole{n}{sampling\_rate\_hz}\DUrole{p}{:}\DUrole{w}{  }\DUrole{n}{\sphinxhref{https://docs.python.org/3/library/functions.html\#int}{int}}\DUrole{w}{  }\DUrole{o}{=}\DUrole{w}{  }\DUrole{default_value}{48000}}, \emph{\DUrole{n}{peak\_level\_dbfs}\DUrole{p}{:}\DUrole{w}{  }\DUrole{n}{\sphinxhref{https://docs.python.org/3/library/functions.html\#float}{float}}\DUrole{w}{  }\DUrole{o}{=}\DUrole{w}{  }\DUrole{default_value}{\sphinxhyphen{}10}}, \emph{\DUrole{n}{number\_of\_channels}\DUrole{p}{:}\DUrole{w}{  }\DUrole{n}{\sphinxhref{https://docs.python.org/3/library/functions.html\#int}{int}}\DUrole{w}{  }\DUrole{o}{=}\DUrole{w}{  }\DUrole{default_value}{1}}, \emph{\DUrole{n}{fade}\DUrole{p}{:}\DUrole{w}{  }\DUrole{n}{\sphinxhref{https://docs.python.org/3/library/stdtypes.html\#str}{str}}\DUrole{w}{  }\DUrole{o}{=}\DUrole{w}{  }\DUrole{default_value}{\textquotesingle{}log\textquotesingle{}}}, \emph{\DUrole{n}{padding\_end\_seconds}\DUrole{p}{:}\DUrole{w}{  }\DUrole{n}{\sphinxhref{https://docs.python.org/3/library/typing.html\#typing.Optional}{Optional}\DUrole{p}{{[}}\sphinxhref{https://docs.python.org/3/library/functions.html\#float}{float}\DUrole{p}{{]}}}\DUrole{w}{  }\DUrole{o}{=}\DUrole{w}{  }\DUrole{default_value}{None}}}{}
\pysigstopsignatures
\sphinxAtStartPar
Creates a sweep signal.
\begin{quote}\begin{description}
\sphinxlineitem{Parameters}\begin{description}
\sphinxlineitem{\sphinxstylestrong{type\_of\_chirp}}{[}str, optional{]}
\sphinxAtStartPar
Choose from \sphinxtitleref{‘lin’}, \sphinxtitleref{‘log’}.
Default: \sphinxtitleref{‘log’}.

\sphinxlineitem{\sphinxstylestrong{range\_hz}}{[}array\sphinxhyphen{}like with length 2{]}
\sphinxAtStartPar
Define range of chirp in Hz. When \sphinxtitleref{None}, all frequencies between
1 and nyquist are taken. Default: \sphinxtitleref{None}.

\sphinxlineitem{\sphinxstylestrong{length\_seconds}}{[}float, optional{]}
\sphinxAtStartPar
Length of the generated signal in seconds. Default: 1.

\sphinxlineitem{\sphinxstylestrong{sampling\_rate\_hz}}{[}int, optional{]}
\sphinxAtStartPar
Sampling rate in Hz. Default: 48000.

\sphinxlineitem{\sphinxstylestrong{peak\_level\_dbfs}}{[}float, optional{]}
\sphinxAtStartPar
Peak level of the signal in dBFS. Default: \sphinxhyphen{}10.

\sphinxlineitem{\sphinxstylestrong{number\_of\_channels}}{[}int, optional{]}
\sphinxAtStartPar
Number of channels (with the same chirp) to be created. Default: 1.

\sphinxlineitem{\sphinxstylestrong{fade}}{[}str, optional{]}
\sphinxAtStartPar
Type of fade done on the generated signal. Options are \sphinxtitleref{‘exp’},
\sphinxtitleref{‘lin’}, \sphinxtitleref{‘log’}. Pass \sphinxtitleref{None} for no fading. Default: \sphinxtitleref{‘log’}.

\sphinxlineitem{\sphinxstylestrong{padding\_end\_seconds}}{[}float, optional{]}
\sphinxAtStartPar
Padding at the end of signal. Use \sphinxtitleref{None} to avoid any padding.
Default: \sphinxtitleref{None}.

\end{description}

\sphinxlineitem{Returns}\begin{description}
\sphinxlineitem{\sphinxstylestrong{chirp\_sig}}{[}Signal{]}
\sphinxAtStartPar
Chirp Signal object.

\end{description}

\end{description}\end{quote}
\subsubsection*{References}

\sphinxAtStartPar
\sphinxurl{https://de.wikipedia.org/wiki/Chirp}

\end{fulllineitems}

\index{dirac() (in module dsptoolbox.generators)@\spxentry{dirac()}\spxextra{in module dsptoolbox.generators}}

\begin{fulllineitems}
\phantomsection\label{\detokenize{modules/dsptoolbox.generators:dsptoolbox.generators.dirac}}
\pysigstartsignatures
\pysiglinewithargsret{\sphinxcode{\sphinxupquote{dsptoolbox.generators.}}\sphinxbfcode{\sphinxupquote{dirac}}}{\emph{\DUrole{n}{length\_samples}\DUrole{p}{:}\DUrole{w}{  }\DUrole{n}{\sphinxhref{https://docs.python.org/3/library/functions.html\#int}{int}}\DUrole{w}{  }\DUrole{o}{=}\DUrole{w}{  }\DUrole{default_value}{512}}, \emph{\DUrole{n}{number\_of\_channels}\DUrole{p}{:}\DUrole{w}{  }\DUrole{n}{\sphinxhref{https://docs.python.org/3/library/functions.html\#int}{int}}\DUrole{w}{  }\DUrole{o}{=}\DUrole{w}{  }\DUrole{default_value}{1}}, \emph{\DUrole{n}{sampling\_rate\_hz}\DUrole{p}{:}\DUrole{w}{  }\DUrole{n}{\sphinxhref{https://docs.python.org/3/library/functions.html\#int}{int}}\DUrole{w}{  }\DUrole{o}{=}\DUrole{w}{  }\DUrole{default_value}{48000}}}{}
\pysigstopsignatures
\sphinxAtStartPar
Generates a dirac impulse Signal with the specified length and
sampling rate.
\begin{quote}\begin{description}
\sphinxlineitem{Parameters}\begin{description}
\sphinxlineitem{\sphinxstylestrong{length\_samples}}{[}int, optional{]}
\sphinxAtStartPar
Length in samples. Default: 512.

\sphinxlineitem{\sphinxstylestrong{number\_of\_channels}}{[}int, optional{]}
\sphinxAtStartPar
Number of channels to be generated with the same impulse. Default: 1.

\sphinxlineitem{\sphinxstylestrong{sampling\_rate\_hz}}{[}int, optional{]}
\sphinxAtStartPar
Sampling rate to be used. Default: 480000.

\end{description}

\sphinxlineitem{Returns}\begin{description}
\sphinxlineitem{\sphinxstylestrong{imp}}{[}Signal{]}
\sphinxAtStartPar
Signal with dirac impulse.

\end{description}

\end{description}\end{quote}

\end{fulllineitems}

\index{noise() (in module dsptoolbox.generators)@\spxentry{noise()}\spxextra{in module dsptoolbox.generators}}

\begin{fulllineitems}
\phantomsection\label{\detokenize{modules/dsptoolbox.generators:dsptoolbox.generators.noise}}
\pysigstartsignatures
\pysiglinewithargsret{\sphinxcode{\sphinxupquote{dsptoolbox.generators.}}\sphinxbfcode{\sphinxupquote{noise}}}{\emph{\DUrole{n}{type\_of\_noise}\DUrole{p}{:}\DUrole{w}{  }\DUrole{n}{\sphinxhref{https://docs.python.org/3/library/stdtypes.html\#str}{str}}\DUrole{w}{  }\DUrole{o}{=}\DUrole{w}{  }\DUrole{default_value}{\textquotesingle{}white\textquotesingle{}}}, \emph{\DUrole{n}{length\_seconds}\DUrole{p}{:}\DUrole{w}{  }\DUrole{n}{\sphinxhref{https://docs.python.org/3/library/functions.html\#float}{float}}\DUrole{w}{  }\DUrole{o}{=}\DUrole{w}{  }\DUrole{default_value}{1}}, \emph{\DUrole{n}{sampling\_rate\_hz}\DUrole{p}{:}\DUrole{w}{  }\DUrole{n}{\sphinxhref{https://docs.python.org/3/library/functions.html\#int}{int}}\DUrole{w}{  }\DUrole{o}{=}\DUrole{w}{  }\DUrole{default_value}{48000}}, \emph{\DUrole{n}{peak\_level\_dbfs}\DUrole{p}{:}\DUrole{w}{  }\DUrole{n}{\sphinxhref{https://docs.python.org/3/library/functions.html\#float}{float}}\DUrole{w}{  }\DUrole{o}{=}\DUrole{w}{  }\DUrole{default_value}{\sphinxhyphen{}10}}, \emph{\DUrole{n}{number\_of\_channels}\DUrole{p}{:}\DUrole{w}{  }\DUrole{n}{\sphinxhref{https://docs.python.org/3/library/functions.html\#int}{int}}\DUrole{w}{  }\DUrole{o}{=}\DUrole{w}{  }\DUrole{default_value}{1}}, \emph{\DUrole{n}{fade}\DUrole{p}{:}\DUrole{w}{  }\DUrole{n}{\sphinxhref{https://docs.python.org/3/library/stdtypes.html\#str}{str}}\DUrole{w}{  }\DUrole{o}{=}\DUrole{w}{  }\DUrole{default_value}{\textquotesingle{}log\textquotesingle{}}}, \emph{\DUrole{n}{padding\_end\_seconds}\DUrole{p}{:}\DUrole{w}{  }\DUrole{n}{\sphinxhref{https://docs.python.org/3/library/typing.html\#typing.Optional}{Optional}\DUrole{p}{{[}}\sphinxhref{https://docs.python.org/3/library/functions.html\#float}{float}\DUrole{p}{{]}}}\DUrole{w}{  }\DUrole{o}{=}\DUrole{w}{  }\DUrole{default_value}{None}}}{}
\pysigstopsignatures
\sphinxAtStartPar
Creates a noise signal.
\begin{quote}\begin{description}
\sphinxlineitem{Parameters}\begin{description}
\sphinxlineitem{\sphinxstylestrong{type\_of\_noise}}{[}str, optional{]}
\sphinxAtStartPar
Choose from \sphinxtitleref{‘white’}, \sphinxtitleref{‘pink’}, \sphinxtitleref{‘red’}, \sphinxtitleref{‘blue’}, \sphinxtitleref{‘violet’}.
Default: \sphinxtitleref{‘white’}.

\sphinxlineitem{\sphinxstylestrong{length\_seconds}}{[}float, optional{]}
\sphinxAtStartPar
Length of the generated signal in seconds. Default: 1.

\sphinxlineitem{\sphinxstylestrong{sampling\_rate\_hz}}{[}int, optional{]}
\sphinxAtStartPar
Sampling rate in Hz. Default: 48000.

\sphinxlineitem{\sphinxstylestrong{peak\_level\_dbfs}}{[}float, optional{]}
\sphinxAtStartPar
Peak level of the signal in dBFS. Default: \sphinxhyphen{}10.

\sphinxlineitem{\sphinxstylestrong{number\_of\_channels}}{[}int, optional{]}
\sphinxAtStartPar
Number of channels (with different noise signals) to be created.
Default: 1.

\sphinxlineitem{\sphinxstylestrong{fade}}{[}str, optional{]}
\sphinxAtStartPar
Type of fade done on the generated signal. Options are \sphinxtitleref{‘exp’},
\sphinxtitleref{‘lin’}, \sphinxtitleref{‘log’}. Pass \sphinxtitleref{None} for no fading. Default: \sphinxtitleref{‘log’}.

\sphinxlineitem{\sphinxstylestrong{padding\_end\_seconds}}{[}float, optional{]}
\sphinxAtStartPar
Padding at the end of signal. Use \sphinxtitleref{None} to avoid any padding.
Default: \sphinxtitleref{None}.

\end{description}

\sphinxlineitem{Returns}\begin{description}
\sphinxlineitem{\sphinxstylestrong{noise\_sig}}{[}Signal{]}
\sphinxAtStartPar
Noise Signal object.

\end{description}

\end{description}\end{quote}
\subsubsection*{References}

\sphinxAtStartPar
\sphinxurl{https://en.wikipedia.org/wiki/Colors\_of\_noise}

\end{fulllineitems}


\sphinxstepscope


\section{Measure (dsptoolbox.measure)}
\label{\detokenize{modules/dsptoolbox.measure:module-dsptoolbox.measure}}\label{\detokenize{modules/dsptoolbox.measure:measure-dsptoolbox-measure}}\label{\detokenize{modules/dsptoolbox.measure::doc}}\index{module@\spxentry{module}!dsptoolbox.measure@\spxentry{dsptoolbox.measure}}\index{dsptoolbox.measure@\spxentry{dsptoolbox.measure}!module@\spxentry{module}}\index{play() (in module dsptoolbox.measure)@\spxentry{play()}\spxextra{in module dsptoolbox.measure}}

\begin{fulllineitems}
\phantomsection\label{\detokenize{modules/dsptoolbox.measure:dsptoolbox.measure.play}}
\pysigstartsignatures
\pysiglinewithargsret{\sphinxcode{\sphinxupquote{dsptoolbox.measure.}}\sphinxbfcode{\sphinxupquote{play}}}{\emph{\DUrole{n}{signal}\DUrole{p}{:}\DUrole{w}{  }\DUrole{n}{{\hyperref[\detokenize{classes:dsptoolbox.classes.signal_class.Signal}]{\sphinxcrossref{Signal}}}}}, \emph{\DUrole{n}{duration\_seconds}\DUrole{p}{:}\DUrole{w}{  }\DUrole{n}{\sphinxhref{https://docs.python.org/3/library/typing.html\#typing.Optional}{Optional}\DUrole{p}{{[}}\sphinxhref{https://docs.python.org/3/library/functions.html\#float}{float}\DUrole{p}{{]}}}\DUrole{w}{  }\DUrole{o}{=}\DUrole{w}{  }\DUrole{default_value}{None}}, \emph{\DUrole{n}{normalized\_dbfs}\DUrole{p}{:}\DUrole{w}{  }\DUrole{n}{\sphinxhref{https://docs.python.org/3/library/functions.html\#float}{float}}\DUrole{w}{  }\DUrole{o}{=}\DUrole{w}{  }\DUrole{default_value}{\sphinxhyphen{}6}}, \emph{\DUrole{n}{device}\DUrole{p}{:}\DUrole{w}{  }\DUrole{n}{\sphinxhref{https://docs.python.org/3/library/typing.html\#typing.Optional}{Optional}\DUrole{p}{{[}}\sphinxhref{https://docs.python.org/3/library/stdtypes.html\#str}{str}\DUrole{p}{{]}}}\DUrole{w}{  }\DUrole{o}{=}\DUrole{w}{  }\DUrole{default_value}{None}}, \emph{\DUrole{n}{play\_channels}\DUrole{o}{=}\DUrole{default_value}{None}}}{}
\pysigstopsignatures
\sphinxAtStartPar
Play some available device. Note that the channel numbers
start here with 1.
\begin{quote}\begin{description}
\sphinxlineitem{Parameters}\begin{description}
\sphinxlineitem{\sphinxstylestrong{signal}}{[}Signal{]}
\sphinxAtStartPar
Signal to be reproduced. Its channel number must match the the length
of the play\_channels vector.

\sphinxlineitem{\sphinxstylestrong{duration\_seconds}}{[}float, optional{]}
\sphinxAtStartPar
If \sphinxtitleref{None}, the whole signal is played, otherwise it is trimmed to the
given length. Default: \sphinxtitleref{None}.

\sphinxlineitem{\sphinxstylestrong{normalized\_dbfs: float, optional}}
\sphinxAtStartPar
Normalizes the signal (dBFS peak level) before playing it.
Set to \sphinxtitleref{None} to ignore normalization. Default: \sphinxhyphen{}6.

\sphinxlineitem{\sphinxstylestrong{device}}{[}str, optional{]}
\sphinxAtStartPar
I/O device to be used. If \sphinxtitleref{None}, the default device is used.
Default: \sphinxtitleref{None}.

\sphinxlineitem{\sphinxstylestrong{play\_channels}}{[}int or array\sphinxhyphen{}like, optional{]}
\sphinxAtStartPar
Output channels that will play the signal. The number of channels
should match the number of channels in signal. When \sphinxtitleref{None}, the
channels are automatically set. Default: \sphinxtitleref{None}.

\end{description}

\end{description}\end{quote}

\end{fulllineitems}

\index{play\_and\_record() (in module dsptoolbox.measure)@\spxentry{play\_and\_record()}\spxextra{in module dsptoolbox.measure}}

\begin{fulllineitems}
\phantomsection\label{\detokenize{modules/dsptoolbox.measure:dsptoolbox.measure.play_and_record}}
\pysigstartsignatures
\pysiglinewithargsret{\sphinxcode{\sphinxupquote{dsptoolbox.measure.}}\sphinxbfcode{\sphinxupquote{play\_and\_record}}}{\emph{\DUrole{n}{signal}\DUrole{p}{:}\DUrole{w}{  }\DUrole{n}{{\hyperref[\detokenize{classes:dsptoolbox.classes.signal_class.Signal}]{\sphinxcrossref{Signal}}}}}, \emph{\DUrole{n}{duration\_seconds}\DUrole{p}{:}\DUrole{w}{  }\DUrole{n}{\sphinxhref{https://docs.python.org/3/library/typing.html\#typing.Optional}{Optional}\DUrole{p}{{[}}\sphinxhref{https://docs.python.org/3/library/functions.html\#float}{float}\DUrole{p}{{]}}}\DUrole{w}{  }\DUrole{o}{=}\DUrole{w}{  }\DUrole{default_value}{None}}, \emph{\DUrole{n}{normalized\_dbfs}\DUrole{p}{:}\DUrole{w}{  }\DUrole{n}{\sphinxhref{https://docs.python.org/3/library/functions.html\#float}{float}}\DUrole{w}{  }\DUrole{o}{=}\DUrole{w}{  }\DUrole{default_value}{\sphinxhyphen{}6}}, \emph{\DUrole{n}{device}\DUrole{p}{:}\DUrole{w}{  }\DUrole{n}{\sphinxhref{https://docs.python.org/3/library/typing.html\#typing.Optional}{Optional}\DUrole{p}{{[}}\sphinxhref{https://docs.python.org/3/library/stdtypes.html\#str}{str}\DUrole{p}{{]}}}\DUrole{w}{  }\DUrole{o}{=}\DUrole{w}{  }\DUrole{default_value}{None}}, \emph{\DUrole{n}{play\_channels}\DUrole{o}{=}\DUrole{default_value}{None}}, \emph{\DUrole{n}{rec\_channels}\DUrole{o}{=}\DUrole{default_value}{{[}1{]}}}}{}
\pysigstopsignatures
\sphinxAtStartPar
Play and record using some available device. Note that the channel
numbers start here with 1.
\begin{quote}\begin{description}
\sphinxlineitem{Parameters}\begin{description}
\sphinxlineitem{\sphinxstylestrong{signal}}{[}Signal{]}
\sphinxAtStartPar
Signal object to be played. The number of channels has to match the
total length and order of play\_channels. The sampling rate of signal
will define the sampling rate of the recorded signals.

\sphinxlineitem{\sphinxstylestrong{duration\_seconds}}{[}float, optional{]}
\sphinxAtStartPar
If \sphinxtitleref{None}, the whole signal is played, otherwise it is trimmed to the
given length. Default: \sphinxtitleref{None}.

\sphinxlineitem{\sphinxstylestrong{normalized\_dbfs: float, optional}}
\sphinxAtStartPar
Normalizes the signal (dBFS peak level) before playing it.
Set to \sphinxtitleref{None} to ignore normalization. Default: \sphinxhyphen{}6.

\sphinxlineitem{\sphinxstylestrong{device}}{[}str, optional{]}
\sphinxAtStartPar
I/O device to be used. If \sphinxtitleref{None}, the default device is used.
Default: \sphinxtitleref{None}.

\sphinxlineitem{\sphinxstylestrong{play\_channels}}{[}int or array\sphinxhyphen{}like, optional{]}
\sphinxAtStartPar
Output channels that will play the signal. The number of channels
should match the number of channels in signal. When \sphinxtitleref{None}, the
channels are automatically set. Default: \sphinxtitleref{None}.

\sphinxlineitem{\sphinxstylestrong{rec\_channels}}{[}int or array\sphinxhyphen{}like, optional{]}
\sphinxAtStartPar
Channel numbers that will be recorded. Default: {[}1{]}.

\end{description}

\sphinxlineitem{Returns}\begin{description}
\sphinxlineitem{\sphinxstylestrong{rec\_sig}}{[}Signal{]}
\sphinxAtStartPar
Recorded signal.

\end{description}

\end{description}\end{quote}

\end{fulllineitems}

\index{print\_device\_info() (in module dsptoolbox.measure)@\spxentry{print\_device\_info()}\spxextra{in module dsptoolbox.measure}}

\begin{fulllineitems}
\phantomsection\label{\detokenize{modules/dsptoolbox.measure:dsptoolbox.measure.print_device_info}}
\pysigstartsignatures
\pysiglinewithargsret{\sphinxcode{\sphinxupquote{dsptoolbox.measure.}}\sphinxbfcode{\sphinxupquote{print\_device\_info}}}{\emph{\DUrole{n}{device\_number}\DUrole{p}{:}\DUrole{w}{  }\DUrole{n}{\sphinxhref{https://docs.python.org/3/library/typing.html\#typing.Optional}{Optional}\DUrole{p}{{[}}\sphinxhref{https://docs.python.org/3/library/functions.html\#int}{int}\DUrole{p}{{]}}}\DUrole{w}{  }\DUrole{o}{=}\DUrole{w}{  }\DUrole{default_value}{None}}}{}
\pysigstopsignatures
\sphinxAtStartPar
Prints available audio devices or information about a certain device
when the device number is given.
\begin{quote}\begin{description}
\sphinxlineitem{Parameters}\begin{description}
\sphinxlineitem{\sphinxstylestrong{device\_number}}{[}int, optional{]}
\sphinxAtStartPar
Prints information about the specific device and returns it as
a dictionary. Use \sphinxtitleref{None} to ignore. Default: \sphinxtitleref{None}.

\end{description}

\sphinxlineitem{Returns}\begin{description}
\sphinxlineitem{\sphinxstylestrong{d}}{[}dict{]}
\sphinxAtStartPar
Only when \sphinxtitleref{device\_number is not None}.

\end{description}

\end{description}\end{quote}

\end{fulllineitems}

\index{record() (in module dsptoolbox.measure)@\spxentry{record()}\spxextra{in module dsptoolbox.measure}}

\begin{fulllineitems}
\phantomsection\label{\detokenize{modules/dsptoolbox.measure:dsptoolbox.measure.record}}
\pysigstartsignatures
\pysiglinewithargsret{\sphinxcode{\sphinxupquote{dsptoolbox.measure.}}\sphinxbfcode{\sphinxupquote{record}}}{\emph{\DUrole{n}{duration\_seconds}\DUrole{p}{:}\DUrole{w}{  }\DUrole{n}{\sphinxhref{https://docs.python.org/3/library/functions.html\#float}{float}}\DUrole{w}{  }\DUrole{o}{=}\DUrole{w}{  }\DUrole{default_value}{5}}, \emph{\DUrole{n}{sampling\_rate\_hz}\DUrole{p}{:}\DUrole{w}{  }\DUrole{n}{\sphinxhref{https://docs.python.org/3/library/functions.html\#int}{int}}\DUrole{w}{  }\DUrole{o}{=}\DUrole{w}{  }\DUrole{default_value}{48000}}, \emph{\DUrole{n}{device}\DUrole{p}{:}\DUrole{w}{  }\DUrole{n}{\sphinxhref{https://docs.python.org/3/library/typing.html\#typing.Optional}{Optional}\DUrole{p}{{[}}\sphinxhref{https://docs.python.org/3/library/stdtypes.html\#str}{str}\DUrole{p}{{]}}}\DUrole{w}{  }\DUrole{o}{=}\DUrole{w}{  }\DUrole{default_value}{None}}, \emph{\DUrole{n}{rec\_channels}\DUrole{o}{=}\DUrole{default_value}{{[}1{]}}}}{}
\pysigstopsignatures
\sphinxAtStartPar
Record using some available device. Note that the channel numbers
start here with 1.
\begin{quote}\begin{description}
\sphinxlineitem{Parameters}\begin{description}
\sphinxlineitem{\sphinxstylestrong{duration\_seconds}}{[}float, optional{]}
\sphinxAtStartPar
Duration of recording in seconds. Default: 5.

\sphinxlineitem{\sphinxstylestrong{sampling\_rate\_hz}}{[}int, optional{]}
\sphinxAtStartPar
Sampling rate used for recording. Default: 48000.

\sphinxlineitem{\sphinxstylestrong{device}}{[}str, optional{]}
\sphinxAtStartPar
I/O device to be used. If \sphinxtitleref{None}, the default device is used.
Default: \sphinxtitleref{None}.

\sphinxlineitem{\sphinxstylestrong{rec\_channels}}{[}int or array\sphinxhyphen{}like, optional{]}
\sphinxAtStartPar
Number that will be recorded. Default: {[}1{]}.

\end{description}

\sphinxlineitem{Returns}\begin{description}
\sphinxlineitem{\sphinxstylestrong{rec\_sig}}{[}Signal{]}
\sphinxAtStartPar
Recorded signal.

\end{description}

\end{description}\end{quote}

\end{fulllineitems}

\index{set\_device() (in module dsptoolbox.measure)@\spxentry{set\_device()}\spxextra{in module dsptoolbox.measure}}

\begin{fulllineitems}
\phantomsection\label{\detokenize{modules/dsptoolbox.measure:dsptoolbox.measure.set_device}}
\pysigstartsignatures
\pysiglinewithargsret{\sphinxcode{\sphinxupquote{dsptoolbox.measure.}}\sphinxbfcode{\sphinxupquote{set\_device}}}{\emph{\DUrole{n}{device\_number}\DUrole{p}{:}\DUrole{w}{  }\DUrole{n}{\sphinxhref{https://docs.python.org/3/library/typing.html\#typing.Optional}{Optional}\DUrole{p}{{[}}\sphinxhref{https://docs.python.org/3/library/functions.html\#int}{int}\DUrole{p}{{]}}}\DUrole{w}{  }\DUrole{o}{=}\DUrole{w}{  }\DUrole{default_value}{None}}}{}
\pysigstopsignatures
\sphinxAtStartPar
Takes in a device number to set it as the default. If \sphinxtitleref{None} is passed,
the available devices are first shown and then the user is asked for
input to set the device.
\begin{quote}\begin{description}
\sphinxlineitem{Parameters}\begin{description}
\sphinxlineitem{\sphinxstylestrong{device\_number}}{[}int, optional{]}
\sphinxAtStartPar
Sets the device as default. Use \sphinxtitleref{None} to ignore. Default: \sphinxtitleref{None}.

\end{description}

\end{description}\end{quote}

\end{fulllineitems}


\sphinxstepscope


\section{Plots (dsptoolbox.plots)}
\label{\detokenize{modules/dsptoolbox.plots:module-dsptoolbox.plots}}\label{\detokenize{modules/dsptoolbox.plots:plots-dsptoolbox-plots}}\label{\detokenize{modules/dsptoolbox.plots::doc}}\index{module@\spxentry{module}!dsptoolbox.plots@\spxentry{dsptoolbox.plots}}\index{dsptoolbox.plots@\spxentry{dsptoolbox.plots}!module@\spxentry{module}}\index{general\_matrix\_plot() (in module dsptoolbox.plots)@\spxentry{general\_matrix\_plot()}\spxextra{in module dsptoolbox.plots}}

\begin{fulllineitems}
\phantomsection\label{\detokenize{modules/dsptoolbox.plots:dsptoolbox.plots.general_matrix_plot}}
\pysigstartsignatures
\pysiglinewithargsret{\sphinxcode{\sphinxupquote{dsptoolbox.plots.}}\sphinxbfcode{\sphinxupquote{general\_matrix\_plot}}}{\emph{\DUrole{n}{matrix}}, \emph{\DUrole{n}{range\_x}\DUrole{o}{=}\DUrole{default_value}{None}}, \emph{\DUrole{n}{range\_y}\DUrole{o}{=}\DUrole{default_value}{None}}, \emph{\DUrole{n}{range\_z}\DUrole{o}{=}\DUrole{default_value}{None}}, \emph{\DUrole{n}{xlabel}\DUrole{p}{:}\DUrole{w}{  }\DUrole{n}{\sphinxhref{https://docs.python.org/3/library/typing.html\#typing.Optional}{Optional}\DUrole{p}{{[}}\sphinxhref{https://docs.python.org/3/library/stdtypes.html\#str}{str}\DUrole{p}{{]}}}\DUrole{w}{  }\DUrole{o}{=}\DUrole{w}{  }\DUrole{default_value}{None}}, \emph{\DUrole{n}{ylabel}\DUrole{p}{:}\DUrole{w}{  }\DUrole{n}{\sphinxhref{https://docs.python.org/3/library/typing.html\#typing.Optional}{Optional}\DUrole{p}{{[}}\sphinxhref{https://docs.python.org/3/library/stdtypes.html\#str}{str}\DUrole{p}{{]}}}\DUrole{w}{  }\DUrole{o}{=}\DUrole{w}{  }\DUrole{default_value}{None}}, \emph{\DUrole{n}{zlabel}\DUrole{p}{:}\DUrole{w}{  }\DUrole{n}{\sphinxhref{https://docs.python.org/3/library/typing.html\#typing.Optional}{Optional}\DUrole{p}{{[}}\sphinxhref{https://docs.python.org/3/library/stdtypes.html\#str}{str}\DUrole{p}{{]}}}\DUrole{w}{  }\DUrole{o}{=}\DUrole{w}{  }\DUrole{default_value}{None}}, \emph{\DUrole{n}{xlog}\DUrole{p}{:}\DUrole{w}{  }\DUrole{n}{\sphinxhref{https://docs.python.org/3/library/functions.html\#bool}{bool}}\DUrole{w}{  }\DUrole{o}{=}\DUrole{w}{  }\DUrole{default_value}{False}}, \emph{\DUrole{n}{ylog}\DUrole{p}{:}\DUrole{w}{  }\DUrole{n}{\sphinxhref{https://docs.python.org/3/library/functions.html\#bool}{bool}}\DUrole{w}{  }\DUrole{o}{=}\DUrole{w}{  }\DUrole{default_value}{False}}, \emph{\DUrole{n}{colorbar}\DUrole{p}{:}\DUrole{w}{  }\DUrole{n}{\sphinxhref{https://docs.python.org/3/library/functions.html\#bool}{bool}}\DUrole{w}{  }\DUrole{o}{=}\DUrole{w}{  }\DUrole{default_value}{True}}, \emph{\DUrole{n}{cmap}\DUrole{p}{:}\DUrole{w}{  }\DUrole{n}{\sphinxhref{https://docs.python.org/3/library/stdtypes.html\#str}{str}}\DUrole{w}{  }\DUrole{o}{=}\DUrole{w}{  }\DUrole{default_value}{\textquotesingle{}magma\textquotesingle{}}}, \emph{\DUrole{n}{returns}\DUrole{p}{:}\DUrole{w}{  }\DUrole{n}{\sphinxhref{https://docs.python.org/3/library/functions.html\#bool}{bool}}\DUrole{w}{  }\DUrole{o}{=}\DUrole{w}{  }\DUrole{default_value}{False}}}{}
\pysigstopsignatures
\sphinxAtStartPar
Generic plot template for a matrix’s heatmap.
\begin{quote}\begin{description}
\sphinxlineitem{Parameters}\begin{description}
\sphinxlineitem{\sphinxstylestrong{matrix}}{[}\sphinxtitleref{np.ndarray}{]}
\sphinxAtStartPar
Matrix with data to plot.

\sphinxlineitem{\sphinxstylestrong{range\_x}}{[}array\sphinxhyphen{}like, optional{]}
\sphinxAtStartPar
Range to show for x axis. Default: None.

\sphinxlineitem{\sphinxstylestrong{range\_y}}{[}array\sphinxhyphen{}like, optional{]}
\sphinxAtStartPar
Range to show for y axis. Default: None.

\sphinxlineitem{\sphinxstylestrong{xlabel}}{[}str, optional{]}
\sphinxAtStartPar
Label for x axis. Default: None.

\sphinxlineitem{\sphinxstylestrong{ylabel}}{[}str, optional{]}
\sphinxAtStartPar
Label for y axis. Default: None.

\sphinxlineitem{\sphinxstylestrong{zlabel}}{[}str, optional{]}
\sphinxAtStartPar
Label for z axis. Default: None.

\sphinxlineitem{\sphinxstylestrong{xlog}}{[}bool, optional{]}
\sphinxAtStartPar
Show x axis as logarithmic. Default: \sphinxtitleref{False}.

\sphinxlineitem{\sphinxstylestrong{ylog}}{[}bool, optional{]}
\sphinxAtStartPar
Show y axis as logarithmic. Default: \sphinxtitleref{False}.

\sphinxlineitem{\sphinxstylestrong{colorbar}}{[}bool, optional{]}
\sphinxAtStartPar
When \sphinxtitleref{True}, a colorbar for zaxis is shown. Default: \sphinxtitleref{True}.

\sphinxlineitem{\sphinxstylestrong{cmap}}{[}str, optional{]}
\sphinxAtStartPar
Type of colormap to use from matplotlib.
See \sphinxurl{https://matplotlib.org/stable/tutorials/colors/colormaps.html}.
Default: \sphinxtitleref{‘magma’}.

\sphinxlineitem{\sphinxstylestrong{returns}}{[}bool, optional{]}
\sphinxAtStartPar
When \sphinxtitleref{True}, the figure and axis are returned. Default: \sphinxtitleref{False}.

\end{description}

\sphinxlineitem{Returns}\begin{description}
\sphinxlineitem{fig, ax}
\sphinxAtStartPar
Returned only when \sphinxtitleref{returns=True}.

\end{description}

\end{description}\end{quote}

\end{fulllineitems}

\index{general\_plot() (in module dsptoolbox.plots)@\spxentry{general\_plot()}\spxextra{in module dsptoolbox.plots}}

\begin{fulllineitems}
\phantomsection\label{\detokenize{modules/dsptoolbox.plots:dsptoolbox.plots.general_plot}}
\pysigstartsignatures
\pysiglinewithargsret{\sphinxcode{\sphinxupquote{dsptoolbox.plots.}}\sphinxbfcode{\sphinxupquote{general\_plot}}}{\emph{\DUrole{n}{x}}, \emph{\DUrole{n}{matrix}}, \emph{\DUrole{n}{range\_x}\DUrole{o}{=}\DUrole{default_value}{None}}, \emph{\DUrole{n}{range\_y}\DUrole{o}{=}\DUrole{default_value}{None}}, \emph{\DUrole{n}{log}\DUrole{p}{:}\DUrole{w}{  }\DUrole{n}{\sphinxhref{https://docs.python.org/3/library/functions.html\#bool}{bool}}\DUrole{w}{  }\DUrole{o}{=}\DUrole{w}{  }\DUrole{default_value}{True}}, \emph{\DUrole{n}{labels}\DUrole{o}{=}\DUrole{default_value}{None}}, \emph{\DUrole{n}{xlabel}\DUrole{p}{:}\DUrole{w}{  }\DUrole{n}{\sphinxhref{https://docs.python.org/3/library/stdtypes.html\#str}{str}}\DUrole{w}{  }\DUrole{o}{=}\DUrole{w}{  }\DUrole{default_value}{\textquotesingle{}Frequency / Hz\textquotesingle{}}}, \emph{\DUrole{n}{ylabel}\DUrole{p}{:}\DUrole{w}{  }\DUrole{n}{\sphinxhref{https://docs.python.org/3/library/typing.html\#typing.Optional}{Optional}\DUrole{p}{{[}}\sphinxhref{https://docs.python.org/3/library/stdtypes.html\#str}{str}\DUrole{p}{{]}}}\DUrole{w}{  }\DUrole{o}{=}\DUrole{w}{  }\DUrole{default_value}{None}}, \emph{\DUrole{n}{info\_box}\DUrole{p}{:}\DUrole{w}{  }\DUrole{n}{\sphinxhref{https://docs.python.org/3/library/typing.html\#typing.Optional}{Optional}\DUrole{p}{{[}}\sphinxhref{https://docs.python.org/3/library/stdtypes.html\#str}{str}\DUrole{p}{{]}}}\DUrole{w}{  }\DUrole{o}{=}\DUrole{w}{  }\DUrole{default_value}{None}}, \emph{\DUrole{n}{returns}\DUrole{p}{:}\DUrole{w}{  }\DUrole{n}{\sphinxhref{https://docs.python.org/3/library/functions.html\#bool}{bool}}\DUrole{w}{  }\DUrole{o}{=}\DUrole{w}{  }\DUrole{default_value}{False}}}{}
\pysigstopsignatures
\sphinxAtStartPar
Generic plot template.
\begin{quote}\begin{description}
\sphinxlineitem{Parameters}\begin{description}
\sphinxlineitem{\sphinxstylestrong{x}}{[}array\sphinxhyphen{}like{]}
\sphinxAtStartPar
Vector for x axis.

\sphinxlineitem{\sphinxstylestrong{matrix}}{[}\sphinxtitleref{np.ndarray}{]}
\sphinxAtStartPar
Matrix with data to plot.

\sphinxlineitem{\sphinxstylestrong{range\_x}}{[}array\sphinxhyphen{}like, optional{]}
\sphinxAtStartPar
Range to show for x axis. Default: None.

\sphinxlineitem{\sphinxstylestrong{range\_y}}{[}array\sphinxhyphen{}like, optional{]}
\sphinxAtStartPar
Range to show for y axis. Default: None.

\sphinxlineitem{\sphinxstylestrong{log}}{[}bool, optional{]}
\sphinxAtStartPar
Show x axis as logarithmic. Default: \sphinxtitleref{True}.

\sphinxlineitem{\sphinxstylestrong{xlabel}}{[}str, optional{]}
\sphinxAtStartPar
Label for x axis. Default: None.

\sphinxlineitem{\sphinxstylestrong{ylabel}}{[}str, optional{]}
\sphinxAtStartPar
Label for y axis. Default: None.

\sphinxlineitem{\sphinxstylestrong{info\_box}}{[}str, optional{]}
\sphinxAtStartPar
String containing extra information to be shown in a info box on the
plot. Default: None.

\sphinxlineitem{\sphinxstylestrong{returns}}{[}bool, optional{]}
\sphinxAtStartPar
When \sphinxtitleref{True}, the figure and axis are returned. Default: \sphinxtitleref{False}.

\end{description}

\sphinxlineitem{Returns}\begin{description}
\sphinxlineitem{fig, ax}
\sphinxAtStartPar
Returned only when \sphinxtitleref{returns=True}.

\end{description}

\end{description}\end{quote}

\end{fulllineitems}

\index{general\_subplots\_line() (in module dsptoolbox.plots)@\spxentry{general\_subplots\_line()}\spxextra{in module dsptoolbox.plots}}

\begin{fulllineitems}
\phantomsection\label{\detokenize{modules/dsptoolbox.plots:dsptoolbox.plots.general_subplots_line}}
\pysigstartsignatures
\pysiglinewithargsret{\sphinxcode{\sphinxupquote{dsptoolbox.plots.}}\sphinxbfcode{\sphinxupquote{general\_subplots\_line}}}{\emph{\DUrole{n}{x}}, \emph{\DUrole{n}{matrix}}, \emph{\DUrole{n}{column}\DUrole{p}{:}\DUrole{w}{  }\DUrole{n}{\sphinxhref{https://docs.python.org/3/library/functions.html\#bool}{bool}}\DUrole{w}{  }\DUrole{o}{=}\DUrole{w}{  }\DUrole{default_value}{True}}, \emph{\DUrole{n}{sharex}\DUrole{p}{:}\DUrole{w}{  }\DUrole{n}{\sphinxhref{https://docs.python.org/3/library/functions.html\#bool}{bool}}\DUrole{w}{  }\DUrole{o}{=}\DUrole{w}{  }\DUrole{default_value}{True}}, \emph{\DUrole{n}{sharey}\DUrole{p}{:}\DUrole{w}{  }\DUrole{n}{\sphinxhref{https://docs.python.org/3/library/functions.html\#bool}{bool}}\DUrole{w}{  }\DUrole{o}{=}\DUrole{w}{  }\DUrole{default_value}{False}}, \emph{\DUrole{n}{log}\DUrole{p}{:}\DUrole{w}{  }\DUrole{n}{\sphinxhref{https://docs.python.org/3/library/functions.html\#bool}{bool}}\DUrole{w}{  }\DUrole{o}{=}\DUrole{w}{  }\DUrole{default_value}{False}}, \emph{\DUrole{n}{xlabels}\DUrole{o}{=}\DUrole{default_value}{None}}, \emph{\DUrole{n}{ylabels}\DUrole{o}{=}\DUrole{default_value}{None}}, \emph{\DUrole{n}{range\_x}\DUrole{o}{=}\DUrole{default_value}{None}}, \emph{\DUrole{n}{range\_y}\DUrole{o}{=}\DUrole{default_value}{None}}, \emph{\DUrole{n}{returns}\DUrole{p}{:}\DUrole{w}{  }\DUrole{n}{\sphinxhref{https://docs.python.org/3/library/functions.html\#bool}{bool}}\DUrole{w}{  }\DUrole{o}{=}\DUrole{w}{  }\DUrole{default_value}{False}}}{}
\pysigstopsignatures
\sphinxAtStartPar
Generic plot template with subplots in one column or row.
\begin{quote}\begin{description}
\sphinxlineitem{Parameters}\begin{description}
\sphinxlineitem{\sphinxstylestrong{x}}{[}array\sphinxhyphen{}like{]}
\sphinxAtStartPar
Vector for x axis.

\sphinxlineitem{\sphinxstylestrong{matrix}}{[}\sphinxtitleref{np.ndarray}{]}
\sphinxAtStartPar
Matrix with data to plot.

\sphinxlineitem{\sphinxstylestrong{column}}{[}bool, optional{]}
\sphinxAtStartPar
When \sphinxtitleref{True}, the subplots are organized in one column. Default: \sphinxtitleref{True}.

\sphinxlineitem{\sphinxstylestrong{sharex}}{[}bool, optional{]}
\sphinxAtStartPar
When \sphinxtitleref{True}, all subplots share the same values for the x axis.
Default: \sphinxtitleref{True}.

\sphinxlineitem{\sphinxstylestrong{sharey}}{[}bool, optional{]}
\sphinxAtStartPar
When \sphinxtitleref{True}, all subplots share the same values for the y axis.
Default: \sphinxtitleref{False}.

\sphinxlineitem{\sphinxstylestrong{log}}{[}bool, optional{]}
\sphinxAtStartPar
Show x axis as logarithmic. Default: \sphinxtitleref{False}.

\sphinxlineitem{\sphinxstylestrong{xlabels}}{[}array\_like, optional{]}
\sphinxAtStartPar
Labels for x axis. Default: None.

\sphinxlineitem{\sphinxstylestrong{ylabels}}{[}array\_like, optional{]}
\sphinxAtStartPar
Labels for y axis. Default: None.

\sphinxlineitem{\sphinxstylestrong{range\_x}}{[}array\sphinxhyphen{}like, optional{]}
\sphinxAtStartPar
Range to show for x axis. Default: None.

\sphinxlineitem{\sphinxstylestrong{range\_y}}{[}array\sphinxhyphen{}like, optional{]}
\sphinxAtStartPar
Range to show for y axis. Default: None.

\sphinxlineitem{\sphinxstylestrong{returns}}{[}bool, optional{]}
\sphinxAtStartPar
When \sphinxtitleref{True}, the figure and axis are returned. Default: \sphinxtitleref{False}.

\end{description}

\sphinxlineitem{Returns}\begin{description}
\sphinxlineitem{fig, ax}
\sphinxAtStartPar
Returned only when \sphinxtitleref{returns=True}.

\end{description}

\end{description}\end{quote}

\end{fulllineitems}

\index{show() (in module dsptoolbox.plots)@\spxentry{show()}\spxextra{in module dsptoolbox.plots}}

\begin{fulllineitems}
\phantomsection\label{\detokenize{modules/dsptoolbox.plots:dsptoolbox.plots.show}}
\pysigstartsignatures
\pysiglinewithargsret{\sphinxcode{\sphinxupquote{dsptoolbox.plots.}}\sphinxbfcode{\sphinxupquote{show}}}{}{}
\pysigstopsignatures
\sphinxAtStartPar
Wrapper around matplotlib’s show.

\end{fulllineitems}


\sphinxstepscope


\section{Room acoustics (dsptoolbox.room\_acoustics)}
\label{\detokenize{modules/dsptoolbox.room_acoustics:module-dsptoolbox.room_acoustics}}\label{\detokenize{modules/dsptoolbox.room_acoustics:room-acoustics-dsptoolbox-room-acoustics}}\label{\detokenize{modules/dsptoolbox.room_acoustics::doc}}\index{module@\spxentry{module}!dsptoolbox.room\_acoustics@\spxentry{dsptoolbox.room\_acoustics}}\index{dsptoolbox.room\_acoustics@\spxentry{dsptoolbox.room\_acoustics}!module@\spxentry{module}}\index{convolve\_rir\_on\_signal() (in module dsptoolbox.room\_acoustics)@\spxentry{convolve\_rir\_on\_signal()}\spxextra{in module dsptoolbox.room\_acoustics}}

\begin{fulllineitems}
\phantomsection\label{\detokenize{modules/dsptoolbox.room_acoustics:dsptoolbox.room_acoustics.convolve_rir_on_signal}}
\pysigstartsignatures
\pysiglinewithargsret{\sphinxcode{\sphinxupquote{dsptoolbox.room\_acoustics.}}\sphinxbfcode{\sphinxupquote{convolve\_rir\_on\_signal}}}{\emph{\DUrole{n}{signal}\DUrole{p}{:}\DUrole{w}{  }\DUrole{n}{{\hyperref[\detokenize{classes:dsptoolbox.classes.signal_class.Signal}]{\sphinxcrossref{Signal}}}}}, \emph{\DUrole{n}{rir}\DUrole{p}{:}\DUrole{w}{  }\DUrole{n}{{\hyperref[\detokenize{classes:dsptoolbox.classes.signal_class.Signal}]{\sphinxcrossref{Signal}}}}}, \emph{\DUrole{n}{keep\_peak\_level}\DUrole{p}{:}\DUrole{w}{  }\DUrole{n}{\sphinxhref{https://docs.python.org/3/library/functions.html\#bool}{bool}}\DUrole{w}{  }\DUrole{o}{=}\DUrole{w}{  }\DUrole{default_value}{True}}, \emph{\DUrole{n}{keep\_length}\DUrole{p}{:}\DUrole{w}{  }\DUrole{n}{\sphinxhref{https://docs.python.org/3/library/functions.html\#bool}{bool}}\DUrole{w}{  }\DUrole{o}{=}\DUrole{w}{  }\DUrole{default_value}{True}}}{}
\pysigstopsignatures
\sphinxAtStartPar
Applies an RIR to a given signal. The RIR should also be a signal object
with a single channel containing the RIR time data. Signal type should
also be set to IR or RIR. By default, all channels are convolved with
the RIR.
\begin{quote}\begin{description}
\sphinxlineitem{Parameters}\begin{description}
\sphinxlineitem{\sphinxstylestrong{signal}}{[}Signal{]}
\sphinxAtStartPar
Signal to which the RIR is applied. All channels are affected.

\sphinxlineitem{\sphinxstylestrong{rir}}{[}Signal{]}
\sphinxAtStartPar
Single\sphinxhyphen{}channel Signal object containing the RIR.

\sphinxlineitem{\sphinxstylestrong{keep\_peak\_level}}{[}bool, optional{]}
\sphinxAtStartPar
When \sphinxtitleref{True}, output signal is normalized to the peak level of
the original signal. Default: \sphinxtitleref{True}.

\sphinxlineitem{\sphinxstylestrong{keep\_length}}{[}bool, optional{]}
\sphinxAtStartPar
When \sphinxtitleref{True}, the original length is kept after convolution, otherwise
the output signal is longer than the input one. Default: \sphinxtitleref{True}.

\end{description}

\sphinxlineitem{Returns}\begin{description}
\sphinxlineitem{\sphinxstylestrong{new\_sig}}{[}Signal{]}
\sphinxAtStartPar
Convolved signal with RIR.

\end{description}

\end{description}\end{quote}

\end{fulllineitems}

\index{find\_modes() (in module dsptoolbox.room\_acoustics)@\spxentry{find\_modes()}\spxextra{in module dsptoolbox.room\_acoustics}}

\begin{fulllineitems}
\phantomsection\label{\detokenize{modules/dsptoolbox.room_acoustics:dsptoolbox.room_acoustics.find_modes}}
\pysigstartsignatures
\pysiglinewithargsret{\sphinxcode{\sphinxupquote{dsptoolbox.room\_acoustics.}}\sphinxbfcode{\sphinxupquote{find\_modes}}}{\emph{\DUrole{n}{signal}\DUrole{p}{:}\DUrole{w}{  }\DUrole{n}{{\hyperref[\detokenize{classes:dsptoolbox.classes.signal_class.Signal}]{\sphinxcrossref{Signal}}}}}, \emph{\DUrole{n}{f\_range\_hz}\DUrole{o}{=}\DUrole{default_value}{{[}50, 200{]}}}, \emph{\DUrole{n}{proximity\_effect}\DUrole{p}{:}\DUrole{w}{  }\DUrole{n}{\sphinxhref{https://docs.python.org/3/library/functions.html\#bool}{bool}}\DUrole{w}{  }\DUrole{o}{=}\DUrole{w}{  }\DUrole{default_value}{False}}, \emph{\DUrole{n}{dist\_hz}\DUrole{p}{:}\DUrole{w}{  }\DUrole{n}{\sphinxhref{https://docs.python.org/3/library/functions.html\#float}{float}}\DUrole{w}{  }\DUrole{o}{=}\DUrole{w}{  }\DUrole{default_value}{5}}}{}
\pysigstopsignatures
\sphinxAtStartPar
This metod is NOT validated. It might not be sufficient to find all
modes in the given range.

\sphinxAtStartPar
Computes the room modes of a set of RIR using different criteria:
Complex mode indication function, sum of magnitude responses and group
delay peaks of the first RIR.
\begin{quote}\begin{description}
\sphinxlineitem{Parameters}\begin{description}
\sphinxlineitem{\sphinxstylestrong{signal}}{[}\sphinxtitleref{Signal}{]}
\sphinxAtStartPar
Signal containing the RIR’S from which to find the modes.

\sphinxlineitem{\sphinxstylestrong{f\_range\_hz}}{[}array\sphinxhyphen{}like, optional{]}
\sphinxAtStartPar
Vector setting range for mode search. Default: {[}50, 200{]}.

\sphinxlineitem{\sphinxstylestrong{proximity\_effect}}{[}bool, optional{]}
\sphinxAtStartPar
When \sphinxtitleref{True}, only group delay criteria is used for finding modes
up until 200 Hz. This is done since a gradient transducer will not
easily see peaks in its magnitude response in low frequencies
due to near\sphinxhyphen{}field effects.
There is also an assessment that the modes are not dips of
the magnitude response. Default: \sphinxtitleref{False}.

\sphinxlineitem{\sphinxstylestrong{dist\_hz}}{[}float, optional{]}
\sphinxAtStartPar
Minimum distance (in Hz) between modes. Default: 5.

\end{description}

\sphinxlineitem{Returns}\begin{description}
\sphinxlineitem{\sphinxstylestrong{f\_modes}}{[}\sphinxtitleref{np.ndarray}{]}
\sphinxAtStartPar
Vector containing frequencies where modes have been localized.

\end{description}

\end{description}\end{quote}
\subsubsection*{References}

\sphinxAtStartPar
\sphinxurl{http://papers.vibetech.com/Paper17-CMIF.pdf}

\end{fulllineitems}

\index{reverb\_time() (in module dsptoolbox.room\_acoustics)@\spxentry{reverb\_time()}\spxextra{in module dsptoolbox.room\_acoustics}}

\begin{fulllineitems}
\phantomsection\label{\detokenize{modules/dsptoolbox.room_acoustics:dsptoolbox.room_acoustics.reverb_time}}
\pysigstartsignatures
\pysiglinewithargsret{\sphinxcode{\sphinxupquote{dsptoolbox.room\_acoustics.}}\sphinxbfcode{\sphinxupquote{reverb\_time}}}{\emph{\DUrole{n}{signal}\DUrole{p}{:}\DUrole{w}{  }\DUrole{n}{{\hyperref[\detokenize{classes:dsptoolbox.classes.signal_class.Signal}]{\sphinxcrossref{Signal}}}}}, \emph{\DUrole{n}{mode}\DUrole{p}{:}\DUrole{w}{  }\DUrole{n}{\sphinxhref{https://docs.python.org/3/library/stdtypes.html\#str}{str}}\DUrole{w}{  }\DUrole{o}{=}\DUrole{w}{  }\DUrole{default_value}{\textquotesingle{}T20\textquotesingle{}}}}{}
\pysigstopsignatures
\sphinxAtStartPar
Computes reverberation time. T20, T30, T60 and EDT.
\begin{quote}\begin{description}
\sphinxlineitem{Parameters}\begin{description}
\sphinxlineitem{\sphinxstylestrong{signal}}{[}Signal{]}
\sphinxAtStartPar
Signal for which to compute reverberation times. It must be type
\sphinxtitleref{‘ir’} or \sphinxtitleref{‘rir’}.

\sphinxlineitem{\sphinxstylestrong{mode}}{[}str, optional{]}
\sphinxAtStartPar
Reverberation time mode. Options are \sphinxtitleref{‘T20’}, \sphinxtitleref{‘T30’}, \sphinxtitleref{‘T60’} or
\sphinxtitleref{‘EDT’}. Default: \sphinxtitleref{‘T20’}.

\end{description}

\sphinxlineitem{Returns}\begin{description}
\sphinxlineitem{\sphinxstylestrong{reverberation\_times}}{[}\sphinxtitleref{np.ndarray}{]}
\sphinxAtStartPar
Reverberation times for each channel.

\end{description}

\end{description}\end{quote}
\subsubsection*{References}
\begin{itemize}
\item {} 
\sphinxAtStartPar
DIN 3382

\item {} 
\sphinxAtStartPar
ISO 3382\sphinxhyphen{}1:2009\sphinxhyphen{}10, Acoustics \sphinxhyphen{} Measurement of the reverberation time of

\end{itemize}

\sphinxAtStartPar
rooms with reference to other acoustical parameters. pp. 22.

\end{fulllineitems}


\sphinxstepscope


\section{Special (dsptoolbox.special)}
\label{\detokenize{modules/dsptoolbox.special:module-dsptoolbox.special}}\label{\detokenize{modules/dsptoolbox.special:special-dsptoolbox-special}}\label{\detokenize{modules/dsptoolbox.special::doc}}\index{module@\spxentry{module}!dsptoolbox.special@\spxentry{dsptoolbox.special}}\index{dsptoolbox.special@\spxentry{dsptoolbox.special}!module@\spxentry{module}}\index{cepstrum() (in module dsptoolbox.special)@\spxentry{cepstrum()}\spxextra{in module dsptoolbox.special}}

\begin{fulllineitems}
\phantomsection\label{\detokenize{modules/dsptoolbox.special:dsptoolbox.special.cepstrum}}
\pysigstartsignatures
\pysiglinewithargsret{\sphinxcode{\sphinxupquote{dsptoolbox.special.}}\sphinxbfcode{\sphinxupquote{cepstrum}}}{\emph{\DUrole{n}{signal}\DUrole{p}{:}\DUrole{w}{  }\DUrole{n}{{\hyperref[\detokenize{classes:dsptoolbox.classes.signal_class.Signal}]{\sphinxcrossref{Signal}}}}}, \emph{\DUrole{n}{mode}\DUrole{o}{=}\DUrole{default_value}{\textquotesingle{}power\textquotesingle{}}}}{}
\pysigstopsignatures
\sphinxAtStartPar
Returns the cepstrum of a given signal in the Quefrency domain.
\begin{quote}\begin{description}
\sphinxlineitem{Parameters}\begin{description}
\sphinxlineitem{\sphinxstylestrong{signal}}{[}Signal{]}
\sphinxAtStartPar
Signal to compute the cepstrum from.

\sphinxlineitem{\sphinxstylestrong{mode}}{[}str, optional{]}
\sphinxAtStartPar
Type of cepstrum. Supported modes are \sphinxtitleref{‘power’}, \sphinxtitleref{‘real’} and
\sphinxtitleref{‘complex’}. Default: \sphinxtitleref{‘power’}.

\end{description}

\sphinxlineitem{Returns}\begin{description}
\sphinxlineitem{\sphinxstylestrong{que}}{[}\sphinxtitleref{np.ndarray}{]}
\sphinxAtStartPar
Quefrency.

\sphinxlineitem{\sphinxstylestrong{ceps}}{[}\sphinxtitleref{np.ndarray}{]}
\sphinxAtStartPar
Cepstrum.

\end{description}

\end{description}\end{quote}
\subsubsection*{References}

\sphinxAtStartPar
\sphinxurl{https://de.wikipedia.org/wiki/Cepstrum}

\end{fulllineitems}


\sphinxstepscope


\section{Standard functions (dsptoolbox.*)}
\label{\detokenize{modules/dsptoolbox.standard_functions:module-dsptoolbox.standard_functions}}\label{\detokenize{modules/dsptoolbox.standard_functions:standard-functions-dsptoolbox}}\label{\detokenize{modules/dsptoolbox.standard_functions::doc}}\index{module@\spxentry{module}!dsptoolbox.standard\_functions@\spxentry{dsptoolbox.standard\_functions}}\index{dsptoolbox.standard\_functions@\spxentry{dsptoolbox.standard\_functions}!module@\spxentry{module}}
\sphinxAtStartPar
Standard functions in DSP processes
\index{excess\_group\_delay() (in module dsptoolbox.standard\_functions)@\spxentry{excess\_group\_delay()}\spxextra{in module dsptoolbox.standard\_functions}}

\begin{fulllineitems}
\phantomsection\label{\detokenize{modules/dsptoolbox.standard_functions:dsptoolbox.standard_functions.excess_group_delay}}
\pysigstartsignatures
\pysiglinewithargsret{\sphinxcode{\sphinxupquote{dsptoolbox.standard\_functions.}}\sphinxbfcode{\sphinxupquote{excess\_group\_delay}}}{\emph{\DUrole{n}{signal}\DUrole{p}{:}\DUrole{w}{  }\DUrole{n}{{\hyperref[\detokenize{classes:dsptoolbox.classes.signal_class.Signal}]{\sphinxcrossref{Signal}}}}}}{}
\pysigstopsignatures
\sphinxAtStartPar
Computes excess group delay.
\begin{quote}\begin{description}
\sphinxlineitem{Parameters}\begin{description}
\sphinxlineitem{\sphinxstylestrong{signal}}{[}Signal{]}
\sphinxAtStartPar
Signal object for which to compute minimal group delay.

\end{description}

\sphinxlineitem{Returns}\begin{description}
\sphinxlineitem{\sphinxstylestrong{f}}{[}\sphinxtitleref{np.ndarray}{]}
\sphinxAtStartPar
Frequency vector.

\sphinxlineitem{\sphinxstylestrong{ex\_gd}}{[}\sphinxtitleref{np.ndarray}{]}
\sphinxAtStartPar
Excess group delays in seconds.

\end{description}

\end{description}\end{quote}

\end{fulllineitems}

\index{group\_delay() (in module dsptoolbox.standard\_functions)@\spxentry{group\_delay()}\spxextra{in module dsptoolbox.standard\_functions}}

\begin{fulllineitems}
\phantomsection\label{\detokenize{modules/dsptoolbox.standard_functions:dsptoolbox.standard_functions.group_delay}}
\pysigstartsignatures
\pysiglinewithargsret{\sphinxcode{\sphinxupquote{dsptoolbox.standard\_functions.}}\sphinxbfcode{\sphinxupquote{group\_delay}}}{\emph{\DUrole{n}{signal}\DUrole{p}{:}\DUrole{w}{  }\DUrole{n}{{\hyperref[\detokenize{classes:dsptoolbox.classes.signal_class.Signal}]{\sphinxcrossref{Signal}}}}}, \emph{\DUrole{n}{method}\DUrole{o}{=}\DUrole{default_value}{\textquotesingle{}direct\textquotesingle{}}}}{}
\pysigstopsignatures
\sphinxAtStartPar
Computation of group delay.
\begin{quote}\begin{description}
\sphinxlineitem{Parameters}\begin{description}
\sphinxlineitem{\sphinxstylestrong{signal}}{[}Signal{]}
\sphinxAtStartPar
Signal for which to compute group delay.

\sphinxlineitem{\sphinxstylestrong{method}}{[}str, optional{]}
\sphinxAtStartPar
\sphinxtitleref{‘direct’} uses gradient with unwrapped phase. \sphinxtitleref{‘matlab’} uses
this implementation:
\sphinxurl{https://www.dsprelated.com/freebooks/filters/Phase\_Group\_Delay.html}

\end{description}

\sphinxlineitem{Returns}\begin{description}
\sphinxlineitem{\sphinxstylestrong{freqs}}{[}\sphinxtitleref{np.ndarray}{]}
\sphinxAtStartPar
Frequency vector in Hz.

\sphinxlineitem{\sphinxstylestrong{group\_delays}}{[}\sphinxtitleref{np.ndarray}{]}
\sphinxAtStartPar
Matrix containing group delays in seconds.

\end{description}

\end{description}\end{quote}

\end{fulllineitems}

\index{latency() (in module dsptoolbox.standard\_functions)@\spxentry{latency()}\spxextra{in module dsptoolbox.standard\_functions}}

\begin{fulllineitems}
\phantomsection\label{\detokenize{modules/dsptoolbox.standard_functions:dsptoolbox.standard_functions.latency}}
\pysigstartsignatures
\pysiglinewithargsret{\sphinxcode{\sphinxupquote{dsptoolbox.standard\_functions.}}\sphinxbfcode{\sphinxupquote{latency}}}{\emph{\DUrole{n}{in1}\DUrole{p}{:}\DUrole{w}{  }\DUrole{n}{{\hyperref[\detokenize{classes:dsptoolbox.classes.signal_class.Signal}]{\sphinxcrossref{Signal}}}}}, \emph{\DUrole{n}{in2}\DUrole{p}{:}\DUrole{w}{  }\DUrole{n}{\sphinxhref{https://docs.python.org/3/library/typing.html\#typing.Optional}{Optional}\DUrole{p}{{[}}{\hyperref[\detokenize{classes:dsptoolbox.classes.signal_class.Signal}]{\sphinxcrossref{Signal}}}\DUrole{p}{{]}}}\DUrole{w}{  }\DUrole{o}{=}\DUrole{w}{  }\DUrole{default_value}{None}}}{}
\pysigstopsignatures
\sphinxAtStartPar
Computes latency between two signals using the correlation method.
If there is no second signal, the latency between the first and the other
channels of the is computed.
\begin{quote}\begin{description}
\sphinxlineitem{Parameters}\begin{description}
\sphinxlineitem{\sphinxstylestrong{in1}}{[}Signal{]}
\sphinxAtStartPar
First signal.

\sphinxlineitem{\sphinxstylestrong{in2}}{[}Signal, optional{]}
\sphinxAtStartPar
Second signal. Default: \sphinxtitleref{None}.

\end{description}

\sphinxlineitem{Returns}\begin{description}
\sphinxlineitem{\sphinxstylestrong{latency\_per\_channel\_samples}}{[}\sphinxtitleref{np.ndarray}{]}
\sphinxAtStartPar
Array with latency between two signals in samples per channel.

\end{description}

\end{description}\end{quote}

\end{fulllineitems}

\index{merge\_filterbanks() (in module dsptoolbox.standard\_functions)@\spxentry{merge\_filterbanks()}\spxextra{in module dsptoolbox.standard\_functions}}

\begin{fulllineitems}
\phantomsection\label{\detokenize{modules/dsptoolbox.standard_functions:dsptoolbox.standard_functions.merge_filterbanks}}
\pysigstartsignatures
\pysiglinewithargsret{\sphinxcode{\sphinxupquote{dsptoolbox.standard\_functions.}}\sphinxbfcode{\sphinxupquote{merge\_filterbanks}}}{\emph{\DUrole{n}{fb1}\DUrole{p}{:}\DUrole{w}{  }\DUrole{n}{{\hyperref[\detokenize{classes:dsptoolbox.classes.filterbank.FilterBank}]{\sphinxcrossref{FilterBank}}}}}, \emph{\DUrole{n}{fb2}\DUrole{p}{:}\DUrole{w}{  }\DUrole{n}{{\hyperref[\detokenize{classes:dsptoolbox.classes.filterbank.FilterBank}]{\sphinxcrossref{FilterBank}}}}}}{}
\pysigstopsignatures
\sphinxAtStartPar
Merges two filterbanks.
\begin{quote}\begin{description}
\sphinxlineitem{Parameters}\begin{description}
\sphinxlineitem{\sphinxstylestrong{fb1}}{[}\sphinxtitleref{FilterBank}{]}
\sphinxAtStartPar
First filterbank.

\sphinxlineitem{\sphinxstylestrong{fb1}}{[}\sphinxtitleref{FilterBank}{]}
\sphinxAtStartPar
Second filterbank.

\end{description}

\sphinxlineitem{Returns}\begin{description}
\sphinxlineitem{\sphinxstylestrong{new\_fb}}{[}\sphinxtitleref{FilterBank}{]}
\sphinxAtStartPar
New filterbank with merged filters

\end{description}

\end{description}\end{quote}

\end{fulllineitems}

\index{merge\_signals() (in module dsptoolbox.standard\_functions)@\spxentry{merge\_signals()}\spxextra{in module dsptoolbox.standard\_functions}}

\begin{fulllineitems}
\phantomsection\label{\detokenize{modules/dsptoolbox.standard_functions:dsptoolbox.standard_functions.merge_signals}}
\pysigstartsignatures
\pysiglinewithargsret{\sphinxcode{\sphinxupquote{dsptoolbox.standard\_functions.}}\sphinxbfcode{\sphinxupquote{merge\_signals}}}{\emph{\DUrole{n}{in1}}, \emph{\DUrole{n}{in2}}, \emph{\DUrole{n}{trimming\_at\_end}\DUrole{p}{:}\DUrole{w}{  }\DUrole{n}{\sphinxhref{https://docs.python.org/3/library/functions.html\#bool}{bool}}\DUrole{w}{  }\DUrole{o}{=}\DUrole{w}{  }\DUrole{default_value}{True}}}{}
\pysigstopsignatures
\sphinxAtStartPar
Merges two signals by appending the channels of the second one to the
first. If the length of in2 is not the same, trimming or padding is
applied at the end.
\begin{quote}\begin{description}
\sphinxlineitem{Parameters}\begin{description}
\sphinxlineitem{\sphinxstylestrong{in1}}{[}\sphinxtitleref{Signal} or \sphinxtitleref{MultiBandSignal}{]}
\sphinxAtStartPar
First signal.

\sphinxlineitem{\sphinxstylestrong{in2}}{[}\sphinxtitleref{Signal} or \sphinxtitleref{MultiBandSignal}{]}
\sphinxAtStartPar
Second signal.

\sphinxlineitem{\sphinxstylestrong{trimming\_at\_end}}{[}bool, optional{]}
\sphinxAtStartPar
If the signals do not have the same length, the second one is padded
or trimmed. When \sphinxtitleref{True}, padding/trimming is done at the end.
Default: \sphinxtitleref{True}.

\end{description}

\sphinxlineitem{Returns}\begin{description}
\sphinxlineitem{\sphinxstylestrong{new\_sig}}{[}\sphinxtitleref{Signal}{]}
\sphinxAtStartPar
New merged signal.

\end{description}

\end{description}\end{quote}

\end{fulllineitems}

\index{minimal\_group\_delay() (in module dsptoolbox.standard\_functions)@\spxentry{minimal\_group\_delay()}\spxextra{in module dsptoolbox.standard\_functions}}

\begin{fulllineitems}
\phantomsection\label{\detokenize{modules/dsptoolbox.standard_functions:dsptoolbox.standard_functions.minimal_group_delay}}
\pysigstartsignatures
\pysiglinewithargsret{\sphinxcode{\sphinxupquote{dsptoolbox.standard\_functions.}}\sphinxbfcode{\sphinxupquote{minimal\_group\_delay}}}{\emph{\DUrole{n}{signal}\DUrole{p}{:}\DUrole{w}{  }\DUrole{n}{{\hyperref[\detokenize{classes:dsptoolbox.classes.signal_class.Signal}]{\sphinxcrossref{Signal}}}}}}{}
\pysigstopsignatures
\sphinxAtStartPar
Computes minimal group delay
\begin{quote}\begin{description}
\sphinxlineitem{Parameters}\begin{description}
\sphinxlineitem{\sphinxstylestrong{signal}}{[}Signal{]}
\sphinxAtStartPar
Signal object for which to compute minimal group delay.

\end{description}

\sphinxlineitem{Returns}\begin{description}
\sphinxlineitem{\sphinxstylestrong{f}}{[}\sphinxtitleref{np.ndarray}{]}
\sphinxAtStartPar
Frequency vector.

\sphinxlineitem{\sphinxstylestrong{min\_gd}}{[}\sphinxtitleref{np.ndarray}{]}
\sphinxAtStartPar
Minimal group delays in seconds as matrix.

\end{description}

\end{description}\end{quote}

\end{fulllineitems}

\index{minimal\_phase() (in module dsptoolbox.standard\_functions)@\spxentry{minimal\_phase()}\spxextra{in module dsptoolbox.standard\_functions}}

\begin{fulllineitems}
\phantomsection\label{\detokenize{modules/dsptoolbox.standard_functions:dsptoolbox.standard_functions.minimal_phase}}
\pysigstartsignatures
\pysiglinewithargsret{\sphinxcode{\sphinxupquote{dsptoolbox.standard\_functions.}}\sphinxbfcode{\sphinxupquote{minimal\_phase}}}{\emph{\DUrole{n}{signal}\DUrole{p}{:}\DUrole{w}{  }\DUrole{n}{{\hyperref[\detokenize{classes:dsptoolbox.classes.signal_class.Signal}]{\sphinxcrossref{Signal}}}}}}{}
\pysigstopsignatures
\sphinxAtStartPar
Gives back a matrix containing the minimal phase for every channel.
\begin{quote}\begin{description}
\sphinxlineitem{Parameters}\begin{description}
\sphinxlineitem{\sphinxstylestrong{signal}}{[}Signal{]}
\sphinxAtStartPar
Signal for which to compute the minimal phase.

\end{description}

\sphinxlineitem{Returns}\begin{description}
\sphinxlineitem{\sphinxstylestrong{f}}{[}\sphinxtitleref{np.ndarray}{]}
\sphinxAtStartPar
Frequency vector.

\sphinxlineitem{\sphinxstylestrong{min\_phases}}{[}\sphinxtitleref{np.ndarray}{]}
\sphinxAtStartPar
Minimal phases as matrix.

\end{description}

\end{description}\end{quote}

\end{fulllineitems}

\index{pad\_trim() (in module dsptoolbox.standard\_functions)@\spxentry{pad\_trim()}\spxextra{in module dsptoolbox.standard\_functions}}

\begin{fulllineitems}
\phantomsection\label{\detokenize{modules/dsptoolbox.standard_functions:dsptoolbox.standard_functions.pad_trim}}
\pysigstartsignatures
\pysiglinewithargsret{\sphinxcode{\sphinxupquote{dsptoolbox.standard\_functions.}}\sphinxbfcode{\sphinxupquote{pad\_trim}}}{\emph{\DUrole{n}{signal}\DUrole{p}{:}\DUrole{w}{  }\DUrole{n}{{\hyperref[\detokenize{classes:dsptoolbox.classes.signal_class.Signal}]{\sphinxcrossref{Signal}}}}}, \emph{\DUrole{n}{desired\_length\_samples}\DUrole{p}{:}\DUrole{w}{  }\DUrole{n}{\sphinxhref{https://docs.python.org/3/library/functions.html\#int}{int}}}, \emph{\DUrole{n}{in\_the\_end}\DUrole{p}{:}\DUrole{w}{  }\DUrole{n}{\sphinxhref{https://docs.python.org/3/library/functions.html\#bool}{bool}}\DUrole{w}{  }\DUrole{o}{=}\DUrole{w}{  }\DUrole{default_value}{True}}}{}
\pysigstopsignatures
\sphinxAtStartPar
Returns a copy of the signal with padded or trimmed time data.
\begin{quote}\begin{description}
\sphinxlineitem{Parameters}\begin{description}
\sphinxlineitem{\sphinxstylestrong{signal}}{[}Signal{]}
\sphinxAtStartPar
Signal to be padded or trimmed.

\sphinxlineitem{\sphinxstylestrong{desired\_length\_samples}}{[}int{]}
\sphinxAtStartPar
Length of resulting signal.

\sphinxlineitem{\sphinxstylestrong{in\_the\_end}}{[}bool, optional{]}
\sphinxAtStartPar
Defines if padding or trimming should be done in the beginning or
in the end of the signal. Default: \sphinxtitleref{True}.

\end{description}

\sphinxlineitem{Returns}\begin{description}
\sphinxlineitem{\sphinxstylestrong{new\_signal}}{[}Signal{]}
\sphinxAtStartPar
New padded signal.

\end{description}

\end{description}\end{quote}

\end{fulllineitems}


\sphinxstepscope


\section{Transfer functions (dsptoolbox.transfer\_functions)}
\label{\detokenize{modules/dsptoolbox.transfer_functions:module-dsptoolbox.transfer_functions}}\label{\detokenize{modules/dsptoolbox.transfer_functions:transfer-functions-dsptoolbox-transfer-functions}}\label{\detokenize{modules/dsptoolbox.transfer_functions::doc}}\index{module@\spxentry{module}!dsptoolbox.transfer\_functions@\spxentry{dsptoolbox.transfer\_functions}}\index{dsptoolbox.transfer\_functions@\spxentry{dsptoolbox.transfer\_functions}!module@\spxentry{module}}\index{compute\_transfer\_function() (in module dsptoolbox.transfer\_functions)@\spxentry{compute\_transfer\_function()}\spxextra{in module dsptoolbox.transfer\_functions}}

\begin{fulllineitems}
\phantomsection\label{\detokenize{modules/dsptoolbox.transfer_functions:dsptoolbox.transfer_functions.compute_transfer_function}}
\pysigstartsignatures
\pysiglinewithargsret{\sphinxcode{\sphinxupquote{dsptoolbox.transfer\_functions.}}\sphinxbfcode{\sphinxupquote{compute\_transfer\_function}}}{\emph{\DUrole{n}{output}\DUrole{p}{:}\DUrole{w}{  }\DUrole{n}{{\hyperref[\detokenize{classes:dsptoolbox.classes.signal_class.Signal}]{\sphinxcrossref{Signal}}}}}, \emph{\DUrole{n}{input}\DUrole{p}{:}\DUrole{w}{  }\DUrole{n}{{\hyperref[\detokenize{classes:dsptoolbox.classes.signal_class.Signal}]{\sphinxcrossref{Signal}}}}}, \emph{\DUrole{n}{mode}\DUrole{o}{=}\DUrole{default_value}{\textquotesingle{}h2\textquotesingle{}}}, \emph{\DUrole{n}{multichannel}\DUrole{p}{:}\DUrole{w}{  }\DUrole{n}{\sphinxhref{https://docs.python.org/3/library/functions.html\#bool}{bool}}\DUrole{w}{  }\DUrole{o}{=}\DUrole{w}{  }\DUrole{default_value}{True}}, \emph{\DUrole{n}{window\_length\_samples}\DUrole{p}{:}\DUrole{w}{  }\DUrole{n}{\sphinxhref{https://docs.python.org/3/library/functions.html\#int}{int}}\DUrole{w}{  }\DUrole{o}{=}\DUrole{w}{  }\DUrole{default_value}{1024}}, \emph{\DUrole{o}{**}\DUrole{n}{kwargs}}}{}
\pysigstopsignatures
\sphinxAtStartPar
Gets transfer function H1, H2 or H3.
H1: for noise in the output signal. \sphinxtitleref{Gxy/Gxx}.
H2: for noise in the input signal. \sphinxtitleref{Gyy/Gyx}.
H3: for noise in both signals. \sphinxtitleref{G\_xy / np.abs(G\_xy) * (G\_yy/G\_xx)**0.5}.
\begin{quote}\begin{description}
\sphinxlineitem{Parameters}\begin{description}
\sphinxlineitem{\sphinxstylestrong{output}}{[}Signal{]}
\sphinxAtStartPar
Signal with output channels.

\sphinxlineitem{\sphinxstylestrong{input}}{[}Signal{]}
\sphinxAtStartPar
Signal with input channels.

\sphinxlineitem{\sphinxstylestrong{mode}}{[}str, optional{]}
\sphinxAtStartPar
Type of transfer function. \sphinxtitleref{‘h1’}, \sphinxtitleref{‘h2’} and \sphinxtitleref{‘h3’} are available.
Default: \sphinxtitleref{‘h2’}.

\sphinxlineitem{\sphinxstylestrong{multichannel}}{[}bool, optional{]}
\sphinxAtStartPar
When \sphinxtitleref{True},  only the first channel of input is used for all output
channels. Default: \sphinxtitleref{True}.

\sphinxlineitem{\sphinxstylestrong{window\_length\_samples}}{[}int, optional{]}
\sphinxAtStartPar
Window length for the IR. Spectrum has the length
window\_length\_samples//2 + 1. Default: 1024.

\sphinxlineitem{\sphinxstylestrong{**kwargs}}{[}dict, optional{]}
\sphinxAtStartPar
Extra parameters for the computation of the cross spectral densities
using welch’s method.

\end{description}

\sphinxlineitem{Returns}\begin{description}
\sphinxlineitem{\sphinxstylestrong{tf}}{[}Signal{]}
\sphinxAtStartPar
Transfer functions. Coherences are also computed and saved in the
Signal object.

\end{description}

\end{description}\end{quote}

\end{fulllineitems}

\index{spectral\_deconvolve() (in module dsptoolbox.transfer\_functions)@\spxentry{spectral\_deconvolve()}\spxextra{in module dsptoolbox.transfer\_functions}}

\begin{fulllineitems}
\phantomsection\label{\detokenize{modules/dsptoolbox.transfer_functions:dsptoolbox.transfer_functions.spectral_deconvolve}}
\pysigstartsignatures
\pysiglinewithargsret{\sphinxcode{\sphinxupquote{dsptoolbox.transfer\_functions.}}\sphinxbfcode{\sphinxupquote{spectral\_deconvolve}}}{\emph{\DUrole{n}{num}\DUrole{p}{:}\DUrole{w}{  }\DUrole{n}{{\hyperref[\detokenize{classes:dsptoolbox.classes.signal_class.Signal}]{\sphinxcrossref{Signal}}}}}, \emph{\DUrole{n}{denum}\DUrole{p}{:}\DUrole{w}{  }\DUrole{n}{{\hyperref[\detokenize{classes:dsptoolbox.classes.signal_class.Signal}]{\sphinxcrossref{Signal}}}}}, \emph{\DUrole{n}{multichannel}\DUrole{o}{=}\DUrole{default_value}{False}}, \emph{\DUrole{n}{mode}\DUrole{o}{=}\DUrole{default_value}{\textquotesingle{}regularized\textquotesingle{}}}, \emph{\DUrole{n}{start\_stop\_hz}\DUrole{o}{=}\DUrole{default_value}{None}}, \emph{\DUrole{n}{threshold\_db}\DUrole{o}{=}\DUrole{default_value}{\sphinxhyphen{}30}}, \emph{\DUrole{n}{padding}\DUrole{p}{:}\DUrole{w}{  }\DUrole{n}{\sphinxhref{https://docs.python.org/3/library/functions.html\#bool}{bool}}\DUrole{w}{  }\DUrole{o}{=}\DUrole{w}{  }\DUrole{default_value}{False}}, \emph{\DUrole{n}{keep\_original\_length}\DUrole{p}{:}\DUrole{w}{  }\DUrole{n}{\sphinxhref{https://docs.python.org/3/library/functions.html\#bool}{bool}}\DUrole{w}{  }\DUrole{o}{=}\DUrole{w}{  }\DUrole{default_value}{False}}}{}
\pysigstopsignatures
\sphinxAtStartPar
Deconvolution by spectral division of two signals.
\begin{quote}\begin{description}
\sphinxlineitem{Parameters}\begin{description}
\sphinxlineitem{\sphinxstylestrong{num}}{[}\sphinxtitleref{Signal}{]}
\sphinxAtStartPar
Signal to deconvolve from.

\sphinxlineitem{\sphinxstylestrong{denum}}{[}\sphinxtitleref{Signal}{]}
\sphinxAtStartPar
Signal to deconvolve.

\sphinxlineitem{\sphinxstylestrong{multichannel}}{[}bool, optional{]}
\sphinxAtStartPar
When \sphinxtitleref{True}, the first channel of denum is used for all channels
in num. Default: \sphinxtitleref{False}.

\sphinxlineitem{\sphinxstylestrong{mode}}{[}str, optional{]}
\sphinxAtStartPar
\sphinxtitleref{‘window’} uses a spectral window in the numerator. \sphinxtitleref{‘regularized’}
uses a regularized inversion. \sphinxtitleref{‘standard’} uses direct deconvolution.
Default: \sphinxtitleref{‘regularized’}.

\sphinxlineitem{\sphinxstylestrong{start\_stop\_hz}}{[}array, None, optional{]}
\sphinxAtStartPar
\sphinxtitleref{‘automatic’} uses a threshold dBFS to create a spectral
window for the numerator or regularized inversion. Array of 2 or
4 frequency points can be also manually given. \sphinxtitleref{None} uses no
spectral window.

\sphinxlineitem{\sphinxstylestrong{threshold}}{[}int, optional{]}
\sphinxAtStartPar
Threshold in dBFS for the automatic creation of the window.
Default: \sphinxhyphen{}30.

\sphinxlineitem{\sphinxstylestrong{padding}}{[}bool, optional{]}
\sphinxAtStartPar
Pads the time data with 2 length. Done for separating distortion
in negative time bins when deconvolving sweep measurements.
Default: \sphinxtitleref{False}.

\sphinxlineitem{\sphinxstylestrong{keep\_original\_length}}{[}bool, optional{]}
\sphinxAtStartPar
Only regarded when padding is \sphinxtitleref{True}. It trims the newly deconvolved
data to its original length. Default: \sphinxtitleref{False}.

\end{description}

\sphinxlineitem{Returns}\begin{description}
\sphinxlineitem{\sphinxstylestrong{new\_sig}}{[}\sphinxtitleref{Signal}{]}
\sphinxAtStartPar
Deconvolved signal.

\end{description}

\end{description}\end{quote}

\end{fulllineitems}

\index{window\_ir() (in module dsptoolbox.transfer\_functions)@\spxentry{window\_ir()}\spxextra{in module dsptoolbox.transfer\_functions}}

\begin{fulllineitems}
\phantomsection\label{\detokenize{modules/dsptoolbox.transfer_functions:dsptoolbox.transfer_functions.window_ir}}
\pysigstartsignatures
\pysiglinewithargsret{\sphinxcode{\sphinxupquote{dsptoolbox.transfer\_functions.}}\sphinxbfcode{\sphinxupquote{window\_ir}}}{\emph{\DUrole{n}{signal}\DUrole{p}{:}\DUrole{w}{  }\DUrole{n}{{\hyperref[\detokenize{classes:dsptoolbox.classes.signal_class.Signal}]{\sphinxcrossref{Signal}}}}}, \emph{\DUrole{n}{constant\_percentage}\DUrole{o}{=}\DUrole{default_value}{0.75}}, \emph{\DUrole{n}{exp2\_trim}\DUrole{p}{:}\DUrole{w}{  }\DUrole{n}{\sphinxhref{https://docs.python.org/3/library/functions.html\#int}{int}}\DUrole{w}{  }\DUrole{o}{=}\DUrole{w}{  }\DUrole{default_value}{13}}, \emph{\DUrole{n}{window\_type}\DUrole{o}{=}\DUrole{default_value}{\textquotesingle{}hann\textquotesingle{}}}, \emph{\DUrole{n}{at\_start}\DUrole{p}{:}\DUrole{w}{  }\DUrole{n}{\sphinxhref{https://docs.python.org/3/library/functions.html\#bool}{bool}}\DUrole{w}{  }\DUrole{o}{=}\DUrole{w}{  }\DUrole{default_value}{True}}}{}
\pysigstopsignatures
\sphinxAtStartPar
Windows an IR with trimming and selection of constant valued length.
\begin{quote}\begin{description}
\sphinxlineitem{Parameters}\begin{description}
\sphinxlineitem{\sphinxstylestrong{signal: Signal}}
\sphinxAtStartPar
Signal to window

\sphinxlineitem{\sphinxstylestrong{constant\_percentage: float, optional}}
\sphinxAtStartPar
Percentage (between 0 and 1) of the window that should be
constant value. Default: 0.75

\sphinxlineitem{\sphinxstylestrong{exp2\_trim: int, optional}}
\sphinxAtStartPar
Exponent of two defining the length to which the IR should be
trimmed. For avoiding trimming set to \sphinxtitleref{None}. Default: 13.

\sphinxlineitem{\sphinxstylestrong{window\_type: str, optional}}
\sphinxAtStartPar
Window function to be used. Available selection from
scipy.signal.windows: \sphinxtitleref{barthann}, \sphinxtitleref{bartlett}, \sphinxtitleref{blackman},
\sphinxtitleref{boxcar}, \sphinxtitleref{cosine}, \sphinxtitleref{hamming}, \sphinxtitleref{hann}, \sphinxtitleref{flattop}, \sphinxtitleref{nuttall} and
others without extra parameters. Default: \sphinxtitleref{hann}.

\sphinxlineitem{\sphinxstylestrong{at\_start: bool, optional}}
\sphinxAtStartPar
Windows the start with a rising window as well as the end.
Default: \sphinxtitleref{True}.

\end{description}

\sphinxlineitem{Returns}\begin{description}
\sphinxlineitem{\sphinxstylestrong{new\_sig}}{[}Signal{]}
\sphinxAtStartPar
Windowed signal. The used window is also saved under \sphinxtitleref{new\_sig.window}.

\end{description}

\end{description}\end{quote}

\end{fulllineitems}



\chapter{Indices and tables}
\label{\detokenize{index:indices-and-tables}}\begin{itemize}
\item {} 
\sphinxAtStartPar
\DUrole{xref,std,std-ref}{genindex}

\item {} 
\sphinxAtStartPar
\DUrole{xref,std,std-ref}{modindex}

\item {} 
\sphinxAtStartPar
\DUrole{xref,std,std-ref}{search}

\end{itemize}


\renewcommand{\indexname}{Python Module Index}
\begin{sphinxtheindex}
\let\bigletter\sphinxstyleindexlettergroup
\bigletter{d}
\item\relax\sphinxstyleindexentry{dsptoolbox.classes.filter\_class}\sphinxstyleindexpageref{classes:\detokenize{module-dsptoolbox.classes.filter_class}}
\item\relax\sphinxstyleindexentry{dsptoolbox.classes.filterbank}\sphinxstyleindexpageref{classes:\detokenize{module-dsptoolbox.classes.filterbank}}
\item\relax\sphinxstyleindexentry{dsptoolbox.classes.multibandsignal}\sphinxstyleindexpageref{classes:\detokenize{module-dsptoolbox.classes.multibandsignal}}
\item\relax\sphinxstyleindexentry{dsptoolbox.classes.signal\_class}\sphinxstyleindexpageref{classes:\detokenize{module-dsptoolbox.classes.signal_class}}
\item\relax\sphinxstyleindexentry{dsptoolbox.distances}\sphinxstyleindexpageref{modules/dsptoolbox.distances:\detokenize{module-dsptoolbox.distances}}
\item\relax\sphinxstyleindexentry{dsptoolbox.filterbanks}\sphinxstyleindexpageref{modules/dsptoolbox.filterbanks:\detokenize{module-dsptoolbox.filterbanks}}
\item\relax\sphinxstyleindexentry{dsptoolbox.generators}\sphinxstyleindexpageref{modules/dsptoolbox.generators:\detokenize{module-dsptoolbox.generators}}
\item\relax\sphinxstyleindexentry{dsptoolbox.measure}\sphinxstyleindexpageref{modules/dsptoolbox.measure:\detokenize{module-dsptoolbox.measure}}
\item\relax\sphinxstyleindexentry{dsptoolbox.plots}\sphinxstyleindexpageref{modules/dsptoolbox.plots:\detokenize{module-dsptoolbox.plots}}
\item\relax\sphinxstyleindexentry{dsptoolbox.room\_acoustics}\sphinxstyleindexpageref{modules/dsptoolbox.room_acoustics:\detokenize{module-dsptoolbox.room_acoustics}}
\item\relax\sphinxstyleindexentry{dsptoolbox.special}\sphinxstyleindexpageref{modules/dsptoolbox.special:\detokenize{module-dsptoolbox.special}}
\item\relax\sphinxstyleindexentry{dsptoolbox.standard\_functions}\sphinxstyleindexpageref{modules/dsptoolbox.standard_functions:\detokenize{module-dsptoolbox.standard_functions}}
\item\relax\sphinxstyleindexentry{dsptoolbox.transfer\_functions}\sphinxstyleindexpageref{modules/dsptoolbox.transfer_functions:\detokenize{module-dsptoolbox.transfer_functions}}
\end{sphinxtheindex}

\renewcommand{\indexname}{Index}
\printindex
\end{document}